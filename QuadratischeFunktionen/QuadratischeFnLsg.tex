\documentclass{article}%
\usepackage[T1]{fontenc}%
\usepackage[utf8]{inputenc}%
\usepackage{lmodern}%
\usepackage{textcomp}%
\usepackage{lastpage}%
\usepackage{xcolor}%
\usepackage{enumitem}%
%
\title{Quadratische Funktionen - Lösungen \newline a2fad196-ecbe-4644-84ec-e6fdb2eec452}%
\date{\today}%
%
\begin{document}%
\normalsize%
\maketitle%
\section{Normalform zu Scheitelpunktsform}%
\label{sec:NormalformzuScheitelpunktsform}%
Für die Normalform ... ist die Scheitelpunktsform ....%
\begin{enumerate}[label=\alph*)]%
\item%
\newline\vspace{0.5cm}$f(x)=2x^2 + 4\Leftrightarrow f(x)=2(x)^2 +4.0$%
\item%
\newline\vspace{0.5cm}$f(x)=-x^2 + 12x - 4\Leftrightarrow f(x)=-(x-6.0)^2 +32.0$%
\item%
\newline\vspace{0.5cm}$f(x)=-3x^2 + 30x + 2\Leftrightarrow f(x)=-3(x-5.0)^2 +77.0$%
\item%
\newline\vspace{0.5cm}$f(x)=-4x^2 - 64x + 7\Leftrightarrow f(x)=-4(x+8.0)^2 +263.0$%
\item%
\newline\vspace{0.5cm}$f(x)=2x^2 + 12x + 4\Leftrightarrow f(x)=2(x+3.0)^2 -14.0$%
\item%
\newline\vspace{0.5cm}$f(x)=3x^2 - 18x - 8\Leftrightarrow f(x)=3(x-3.0)^2 -35.0$%
\item%
\newline\vspace{0.5cm}$f(x)=x^2 - 12x - 9\Leftrightarrow f(x)=(x-6.0)^2 -45.0$%
\item%
\newline\vspace{0.5cm}$f(x)=-2x^2 + 24x - 8\Leftrightarrow f(x)=-2(x-6.0)^2 +64.0$%
\item%
\newline\vspace{0.5cm}$f(x)=2x^2 - 20x - 5\Leftrightarrow f(x)=2(x-5.0)^2 -55.0$%
\item%
\newline\vspace{0.5cm}$f(x)=3x^2 + 12x - 9\Leftrightarrow f(x)=3(x+2.0)^2 -21.0$%
\end{enumerate}

%
\section{Scheitelpunktsform zu Faktorisierten Form}%
\label{sec:ScheitelpunktsformzuFaktorisiertenForm}%
Für die Scheitelpunktsform ... ist die Faktorisierte Form ....%
\begin{enumerate}[label=\alph*)]%
\item%
\newline\vspace{0.5cm}$f(x)=-(x+1.0)^2 +36.0\Leftrightarrow f(x)=-(x-5)(x+7)$%
\item%
\newline\vspace{0.5cm}$f(x)=2(x-5.0)^2 -8.0\Leftrightarrow f(x)=2(x-3)(x-7)$%
\item%
\newline\vspace{0.5cm}$f(x)=3(x-1.5)^2 -126.75\Leftrightarrow f(x)=3(x+5)(x-8)$%
\item%
\newline\vspace{0.5cm}$f(x)=-4(x-8.0)^2 +16.0\Leftrightarrow f(x)=-4(x-6)(x-10)$%
\item%
\newline\vspace{0.5cm}$f(x)=4(x-5.0)^2 -16.0\Leftrightarrow f(x)=4(x-3)(x-7)$%
\item%
\newline\vspace{0.5cm}$f(x)=3(x)^2 -12.0\Leftrightarrow f(x)=3(x-2)(x+2)$%
\item%
\newline\vspace{0.5cm}$f(x)=(x+3.0)^2 -25.0\Leftrightarrow f(x)=(x+8)(x-2)$%
\item%
\newline\vspace{0.5cm}$f(x)=3(x)^2\Leftrightarrow f(x)=3(x)(x)$%
\item%
\newline\vspace{0.5cm}$f(x)=-3(x-1.0)^2 +75.0\Leftrightarrow f(x)=-3(x+4)(x-6)$%
\item%
\newline\vspace{0.5cm}$f(x)=2(x)^2 -98.0\Leftrightarrow f(x)=2(x-7)(x+7)$%
\end{enumerate}

%
\section{Faktorisierte Form zu Normalform}%
\label{sec:FaktorisierteFormzuNormalform}%
Für die Faktorisierte Form ... ist die Normalform ....%
\begin{enumerate}[label=\alph*)]%
\item%
\newline\vspace{0.5cm}$f(x)=-4(x)(x-9)\Leftrightarrow f(x)=-4x^2 + 36x$%
\item%
\newline\vspace{0.5cm}$f(x)=-3(x+2)(x+4)\Leftrightarrow f(x)=-3x^2 - 18x - 24$%
\item%
\newline\vspace{0.5cm}$f(x)=4(x+10)(x)\Leftrightarrow f(x)=4x^2 + 40x$%
\item%
\newline\vspace{0.5cm}$f(x)=3(x-9)(x)\Leftrightarrow f(x)=3x^2 - 27x$%
\item%
\newline\vspace{0.5cm}$f(x)=2(x+8)(x-8)\Leftrightarrow f(x)=2x^2 - 128$%
\item%
\newline\vspace{0.5cm}$f(x)=-2(x+5)(x-5)\Leftrightarrow f(x)=-2x^2 + 50$%
\item%
\newline\vspace{0.5cm}$f(x)=-2(x-5)(x-2)\Leftrightarrow f(x)=-2x^2 + 14x - 20$%
\item%
\newline\vspace{0.5cm}$f(x)=3(x-9)(x-5)\Leftrightarrow f(x)=3x^2 - 42x + 135$%
\item%
\newline\vspace{0.5cm}$f(x)=-(x-4)(x-4)\Leftrightarrow f(x)=-x^2 + 8x - 16$%
\item%
\newline\vspace{0.5cm}$f(x)=-4(x+9)(x-1)\Leftrightarrow f(x)=-4x^2 - 32x + 36$%
\end{enumerate}

%
\section{Normalform zu Faktorisierter Form}%
\label{sec:NormalformzuFaktorisierterForm}%
Für die Normalform ... ist die Faktorisierte Form ....%
\begin{enumerate}[label=\alph*)]%
\item%
\newline\vspace{0.5cm}$f(x)=4x^2 + 52x + 160\Leftrightarrow f(x)=4(x+8)(x+5)$%
\item%
\newline\vspace{0.5cm}$f(x)=-4x^2 - 56x - 180\Leftrightarrow f(x)=-4(x+5)(x+9)$%
\item%
\newline\vspace{0.5cm}$f(x)=2x^2 - 8x - 42\Leftrightarrow f(x)=2(x+3)(x-7)$%
\item%
\newline\vspace{0.5cm}$f(x)=4x^2 - 80x + 400\Leftrightarrow f(x)=4(x-10)(x-10)$%
\item%
\newline\vspace{0.5cm}$f(x)=2x^2 - 8\Leftrightarrow f(x)=2(x+2)(x-2)$%
\item%
\newline\vspace{0.5cm}$f(x)=3x^2 + 9x\Leftrightarrow f(x)=3(x)(x+3)$%
\item%
\newline\vspace{0.5cm}$f(x)=2x^2 + 6x + 4\Leftrightarrow f(x)=2(x+2)(x+1)$%
\item%
\newline\vspace{0.5cm}$f(x)=-3x^2 - 9x + 162\Leftrightarrow f(x)=-3(x+9)(x-6)$%
\item%
\newline\vspace{0.5cm}$f(x)=x^2 - 4x - 12\Leftrightarrow f(x)=(x-6)(x+2)$%
\item%
\newline\vspace{0.5cm}$f(x)=2x^2 + 16x + 14\Leftrightarrow f(x)=2(x+1)(x+7)$%
\end{enumerate}

%
\section{Scheitelpunktform zu Normalform}%
\label{sec:ScheitelpunktformzuNormalform}%
Für die Scheitelpunktform ... ist die Normalform ....%
\begin{enumerate}[label=\alph*)]%
\item%
\newline\vspace{0.5cm}$f(x)=-3(x)^2 -3\Leftrightarrow f(x)=-3x^2 - 3$%
\item%
\newline\vspace{0.5cm}$f(x)=3(x+1)^2 -7\Leftrightarrow f(x)=3x^2 + 6x - 4$%
\item%
\newline\vspace{0.5cm}$f(x)=-3(x+4)^2 -8\Leftrightarrow f(x)=-3x^2 - 24x - 56$%
\item%
\newline\vspace{0.5cm}$f(x)=-4(x+10)^2 -5\Leftrightarrow f(x)=-4x^2 - 80x - 405$%
\item%
\newline\vspace{0.5cm}$f(x)=(x)^2 +2\Leftrightarrow f(x)=x^2 + 2$%
\item%
\newline\vspace{0.5cm}$f(x)=-3(x)^2 +3\Leftrightarrow f(x)=-3x^2 + 3$%
\item%
\newline\vspace{0.5cm}$f(x)=4(x+8)^2 -3\Leftrightarrow f(x)=4x^2 + 64x + 253$%
\item%
\newline\vspace{0.5cm}$f(x)=-4(x)^2 +2\Leftrightarrow f(x)=-4x^2 + 2$%
\item%
\newline\vspace{0.5cm}$f(x)=-3(x+6)^2 -1\Leftrightarrow f(x)=-3x^2 - 36x - 109$%
\item%
\newline\vspace{0.5cm}$f(x)=-2(x)^2 -2\Leftrightarrow f(x)=-2x^2 - 2$%
\end{enumerate}

%
\section{Faktorisierte Form zu Scheitelpunktform}%
\label{sec:FaktorisierteFormzuScheitelpunktform}%
Für die Faktorisierte Form ... ist die Scheitelpunktform ....%
\begin{enumerate}[label=\alph*)]%
\item%
\newline\vspace{0.5cm}$f(x)=-3(x)(x+2)\Leftrightarrow f(x)=-3(x+1.0)^2 +3.0$%
\item%
\newline\vspace{0.5cm}$f(x)=(x+1)(x-1)\Leftrightarrow f(x)=(x)^2 -1.0$%
\item%
\newline\vspace{0.5cm}$f(x)=-3(x)(x+9)\Leftrightarrow f(x)=-3(x+4.5)^2 +60.75$%
\item%
\newline\vspace{0.5cm}$f(x)=(x-5)(x-5)\Leftrightarrow f(x)=(x-5.0)^2$%
\item%
\newline\vspace{0.5cm}$f(x)=2(x+9)(x-9)\Leftrightarrow f(x)=2(x)^2 -162.0$%
\item%
\newline\vspace{0.5cm}$f(x)=4(x-1)(x+3)\Leftrightarrow f(x)=4(x+1.0)^2 -16.0$%
\item%
\newline\vspace{0.5cm}$f(x)=-4(x+7)(x)\Leftrightarrow f(x)=-4(x+3.5)^2 +49.0$%
\item%
\newline\vspace{0.5cm}$f(x)=-3(x+1)(x+1)\Leftrightarrow f(x)=-3(x+1.0)^2$%
\item%
\newline\vspace{0.5cm}$f(x)=-3(x-5)(x-3)\Leftrightarrow f(x)=-3(x-4.0)^2 +3.0$%
\item%
\newline\vspace{0.5cm}$f(x)=-2(x+4)(x-2)\Leftrightarrow f(x)=-2(x+1.0)^2 +18.0$%
\end{enumerate}

%
\section{Scheitelpunktform: Bestimme den Scheitelpunkt}%
\label{sec:ScheitelpunktformBestimmedenScheitelpunkt}%
Für die Scheitelpunktform ... ist der Scheitelpunkt ....%
\begin{enumerate}[label=\alph*)]%
\item%
\newline\vspace{0.5cm}$f(x)=(x-5)^2 +1 \Rightarrow SP(5|1) $%
\item%
\newline\vspace{0.5cm}$f(x)=3(x+7)^2 +2 \Rightarrow SP(-7|2) $%
\item%
\newline\vspace{0.5cm}$f(x)=-(x-6)^2 +10 \Rightarrow SP(6|10) $%
\item%
\newline\vspace{0.5cm}$f(x)=-4(x-6)^2 +6 \Rightarrow SP(6|6) $%
\item%
\newline\vspace{0.5cm}$f(x)=-3(x-3)^2 -8 \Rightarrow SP(3|-8) $%
\item%
\newline\vspace{0.5cm}$f(x)=2(x)^2 -6 \Rightarrow SP(0|-6) $%
\item%
\newline\vspace{0.5cm}$f(x)=-4(x-1)^2 -10 \Rightarrow SP(1|-10) $%
\item%
\newline\vspace{0.5cm}$f(x)=2(x-6)^2 -8 \Rightarrow SP(6|-8) $%
\item%
\newline\vspace{0.5cm}$f(x)=(x-6)^2 \Rightarrow SP(6|0) $%
\item%
\newline\vspace{0.5cm}$f(x)=-3(x+7)^2 +7 \Rightarrow SP(-7|7) $%
\end{enumerate}

%
\section{Normalform: Bestimme den Scheitelpunkt}%
\label{sec:NormalformBestimmedenScheitelpunkt}%
Für die Normalform ... ist der Scheitelpunkt ....%
\begin{enumerate}[label=\alph*)]%
\item%
\newline\vspace{0.5cm}$f(x)=x^2 + 14x + 51 \Rightarrow SP(-7|2) $%
\item%
\newline\vspace{0.5cm}$f(x)=-4x^2 + 72x - 330 \Rightarrow SP(9|-6) $%
\item%
\newline\vspace{0.5cm}$f(x)=-x^2 - 8x - 14 \Rightarrow SP(-4|2) $%
\item%
\newline\vspace{0.5cm}$f(x)=-x^2 - 6x - 16 \Rightarrow SP(-3|-7) $%
\item%
\newline\vspace{0.5cm}$f(x)=-2x^2 - 8x - 17 \Rightarrow SP(-2|-9) $%
\item%
\newline\vspace{0.5cm}$f(x)=-2x^2 + 12x - 22 \Rightarrow SP(3|-4) $%
\item%
\newline\vspace{0.5cm}$f(x)=-4x^2 - 16x - 25 \Rightarrow SP(-2|-9) $%
\item%
\newline\vspace{0.5cm}$f(x)=-3x^2 + 6x - 7 \Rightarrow SP(1|-4) $%
\item%
\newline\vspace{0.5cm}$f(x)=2x^2 + 12x + 17 \Rightarrow SP(-3|-1) $%
\item%
\newline\vspace{0.5cm}$f(x)=-2x^2 + 24x - 69 \Rightarrow SP(6|3) $%
\end{enumerate}

%
\section{Faktorisierte Form: Bestimme den Scheitelpunkt}%
\label{sec:FaktorisierteFormBestimmedenScheitelpunkt}%
Für die Faktorisierte Form ... ist der Scheitelpunkt ....%
\begin{enumerate}[label=\alph*)]%
\item%
\newline\vspace{0.5cm}$f(x)=(x+8)(x-3) \Rightarrow SP(-2.5|-30.25) $%
\item%
\newline\vspace{0.5cm}$f(x)=3(x+4)(x-5) \Rightarrow SP(0.5|-60.75) $%
\item%
\newline\vspace{0.5cm}$f(x)=-4(x-5)(x-1) \Rightarrow SP(3.0|16.0) $%
\item%
\newline\vspace{0.5cm}$f(x)=2(x-5)(x-2) \Rightarrow SP(3.5|-4.5) $%
\item%
\newline\vspace{0.5cm}$f(x)=4(x+8)(x-2) \Rightarrow SP(-3.0|-100.0) $%
\item%
\newline\vspace{0.5cm}$f(x)=3(x-3)(x+3) \Rightarrow SP(0.0|-27.0) $%
\item%
\newline\vspace{0.5cm}$f(x)=-4(x+5)(x+3) \Rightarrow SP(-4.0|4.0) $%
\item%
\newline\vspace{0.5cm}$f(x)=2(x-4)(x-10) \Rightarrow SP(7.0|-18.0) $%
\item%
\newline\vspace{0.5cm}$f(x)=2(x+8)(x-5) \Rightarrow SP(-1.5|-84.5) $%
\item%
\newline\vspace{0.5cm}$f(x)=-3(x)(x+10) \Rightarrow SP(-5.0|75.0) $%
\end{enumerate}

%
\section{Faktorisierte Form: Bestimme die Nullstellen}%
\label{sec:FaktorisierteFormBestimmedieNullstellen}%
Für die Faktorisierte Form ... sind die Nullstellen ....%
\begin{enumerate}[label=\alph*)]%
\item%
\newline\vspace{0.5cm}$f(x)=-(x+4)(x+2) \Rightarrow $ Zwei Nullstellen $: (-4|0) $ und $ (-2|0) $%
\item%
\newline\vspace{0.5cm}$f(x)=-(x-6)(x-6) \Rightarrow $ Zwei Nullstellen $: (6|0) $ und $ (6|0) $%
\item%
\newline\vspace{0.5cm}$f(x)=(x-4)(x+3) \Rightarrow $ Zwei Nullstellen $: (4|0) $ und $ (-3|0) $%
\item%
\newline\vspace{0.5cm}$f(x)=-2(x-1)(x-7) \Rightarrow $ Zwei Nullstellen $: (1|0) $ und $ (7|0) $%
\item%
\newline\vspace{0.5cm}$f(x)=3(x-5)(x+4) \Rightarrow $ Zwei Nullstellen $: (5|0) $ und $ (-4|0) $%
\item%
\newline\vspace{0.5cm}$f(x)=3(x+2)(x+1) \Rightarrow $ Zwei Nullstellen $: (-2|0) $ und $ (-1|0) $%
\item%
\newline\vspace{0.5cm}$f(x)=4(x+4)(x) \Rightarrow $ Zwei Nullstellen $: (-4|0) $ und $ (0|0) $%
\item%
\newline\vspace{0.5cm}$f(x)=(x-3)(x+6) \Rightarrow $ Zwei Nullstellen $: (3|0) $ und $ (-6|0) $%
\item%
\newline\vspace{0.5cm}$f(x)=3(x+7)(x+3) \Rightarrow $ Zwei Nullstellen $: (-7|0) $ und $ (-3|0) $%
\item%
\newline\vspace{0.5cm}$f(x)=2(x+6)(x+6) \Rightarrow $ Zwei Nullstellen $: (-6|0) $ und $ (-6|0) $%
\end{enumerate}

%
\section{Normalform: Bestimme die Nullstellen}%
\label{sec:NormalformBestimmedieNullstellen}%
Für die Normalform ... sind die Nullstellen ....%
\begin{enumerate}[label=\alph*)]%
\item%
\newline\vspace{0.5cm}$f(x)=3x^2 + 48x + 9 \Rightarrow $ Zwei Nullstellen $: (-0.18975032409334602|0) $ und $ (-15.810249675906654|0) $%
\item%
\newline\vspace{0.5cm}$f(x)=2x^2 - 28x + 1 \Rightarrow $ Zwei Nullstellen $: (13.96419413859206|0) $ und $ (0.0358058614079404|0) $%
\item%
\newline\vspace{0.5cm}$f(x)=4x^2 + 48x + 10 \Rightarrow $ Zwei Nullstellen $: (-0.21208154860488726|0) $ und $ (-11.787918451395113|0) $%
\item%
\newline\vspace{0.5cm}$f(x)=4x^2 - 9 \Rightarrow $ Zwei Nullstellen $: (1.5|0) $ und $ (-1.5|0) $%
\item%
\newline\vspace{0.5cm}$f(x)=2x^2 + 4x + 10 \Rightarrow  $ Keine Lösung/Keine Nullstellen $ $%
\item%
\newline\vspace{0.5cm}$f(x)=-4x^2 - 10 \Rightarrow  $ Keine Lösung/Keine Nullstellen $ $%
\item%
\newline\vspace{0.5cm}$f(x)=-3x^2 - 36x + 10 \Rightarrow $ Zwei Nullstellen $: (0.2716292407422598|0) $ und $ (-12.27162924074226|0) $%
\item%
\newline\vspace{0.5cm}$f(x)=4x^2 + 80x + 3 \Rightarrow $ Zwei Nullstellen $: (-0.037570577414362205|0) $ und $ (-19.962429422585636|0) $%
\item%
\newline\vspace{0.5cm}$f(x)=3x^2 + 24x - 5 \Rightarrow $ Zwei Nullstellen $: (0.20317340430616415|0) $ und $ (-8.203173404306163|0) $%
\item%
\newline\vspace{0.5cm}$f(x)=-4x^2 - 16x + 2 \Rightarrow $ Zwei Nullstellen $: (0.12132034355964239|0) $ und $ (-4.121320343559642|0) $%
\item%
\newline\vspace{0.5cm}$f(x)=-3x^2 + 60x + 4 \Rightarrow $ Zwei Nullstellen $: (20.06644591369433|0) $ und $ (-0.0664459136943325|0) $%
\item%
\newline\vspace{0.5cm}$f(x)=x^2 - 16x - 10 \Rightarrow $ Zwei Nullstellen $: (16.602325267042627|0) $ und $ (-0.6023252670426267|0) $%
\item%
\newline\vspace{0.5cm}$f(x)=-x^2 - 10x + 5 \Rightarrow $ Zwei Nullstellen $: (0.4772255750516612|0) $ und $ (-10.477225575051662|0) $%
\item%
\newline\vspace{0.5cm}$f(x)=x^2 + 8x + 6 \Rightarrow $ Zwei Nullstellen $: (-0.8377223398316205|0) $ und $ (-7.16227766016838|0) $%
\item%
\newline\vspace{0.5cm}$f(x)=-2x^2 - 8x + 1 \Rightarrow $ Zwei Nullstellen $: (0.12132034355964239|0) $ und $ (-4.121320343559642|0) $%
\item%
\newline\vspace{0.5cm}$f(x)=-3x^2 - 12x + 7 \Rightarrow $ Zwei Nullstellen $: (0.5166114784235836|0) $ und $ (-4.516611478423584|0) $%
\item%
\newline\vspace{0.5cm}$f(x)=-4x^2 + 8x \Rightarrow $ Zwei Nullstellen $: (2.0|0) $ und $ (0.0|0) $%
\item%
\newline\vspace{0.5cm}$f(x)=3x^2 + 12x - 9 \Rightarrow $ Zwei Nullstellen $: (0.6457513110645907|0) $ und $ (-4.645751311064591|0) $%
\item%
\newline\vspace{0.5cm}$f(x)=-3x^2 + 24x + 1 \Rightarrow $ Zwei Nullstellen $: (8.04145188432738|0) $ und $ (-0.04145188432738056|0) $%
\item%
\newline\vspace{0.5cm}$f(x)=3x^2 + 30x - 8 \Rightarrow $ Zwei Nullstellen $: (0.25991127935316705|0) $ und $ (-10.259911279353167|0) $%
\end{enumerate}

%
\section{Scheitelpunktform: Bestimme die Nullstellen}%
\label{sec:ScheitelpunktformBestimmedieNullstellen}%
Für die Scheitelpunktform ... sind die Nullstellen ....%
\begin{enumerate}[label=\alph*)]%
\item%
\newline\vspace{0.5cm}$f(x)=-4(x+10)^2 -10 \Rightarrow  $ Keine Lösung/Keine Nullstellen $ $%
\item%
\newline\vspace{0.5cm}$f(x)=-3(x-1)^2 -2 \Rightarrow  $ Keine Lösung/Keine Nullstellen $ $%
\item%
\newline\vspace{0.5cm}$f(x)=-4(x-2)^2 -3 \Rightarrow  $ Keine Lösung/Keine Nullstellen $ $%
\item%
\newline\vspace{0.5cm}$f(x)=3(x+3)^2 -8 \Rightarrow $ Zwei Nullstellen $: (-1.367006838144548|0) $ und $ (-4.6329931618554525|0) $%
\item%
\newline\vspace{0.5cm}$f(x)=-2(x-6)^2 -4 \Rightarrow  $ Keine Lösung/Keine Nullstellen $ $%
\item%
\newline\vspace{0.5cm}$f(x)=4(x+2)^2 +1 \Rightarrow  $ Keine Lösung/Keine Nullstellen $ $%
\item%
\newline\vspace{0.5cm}$f(x)=4(x-10)^2 -8 \Rightarrow $ Zwei Nullstellen $: (11.414213562373096|0) $ und $ (8.585786437626904|0) $%
\item%
\newline\vspace{0.5cm}$f(x)=-2(x+9)^2 -1 \Rightarrow  $ Keine Lösung/Keine Nullstellen $ $%
\item%
\newline\vspace{0.5cm}$f(x)=2(x+10)^2 -6 \Rightarrow $ Zwei Nullstellen $: (-8.267949192431123|0) $ und $ (-11.732050807568877|0) $%
\item%
\newline\vspace{0.5cm}$f(x)=2(x-3)^2 +6 \Rightarrow  $ Keine Lösung/Keine Nullstellen $ $%
\item%
\newline\vspace{0.5cm}$f(x)=2(x+2)^2 -1 \Rightarrow $ Zwei Nullstellen $: (-1.2928932188134525|0) $ und $ (-2.7071067811865475|0) $%
\item%
\newline\vspace{0.5cm}$f(x)=3(x+2)^2 +8 \Rightarrow  $ Keine Lösung/Keine Nullstellen $ $%
\item%
\newline\vspace{0.5cm}$f(x)=-(x+4)^2 +1 \Rightarrow $ Zwei Nullstellen $: (-3.0|0) $ und $ (-5.0|0) $%
\item%
\newline\vspace{0.5cm}$f(x)=2(x-3)^2 -5 \Rightarrow $ Zwei Nullstellen $: (4.58113883008419|0) $ und $ (1.4188611699158102|0) $%
\item%
\newline\vspace{0.5cm}$f(x)=-3(x-5)^2 +5 \Rightarrow $ Zwei Nullstellen $: (6.290994448735805|0) $ und $ (3.7090055512641946|0) $%
\item%
\newline\vspace{0.5cm}$f(x)=-3(x-2)^2 +3 \Rightarrow $ Zwei Nullstellen $: (3.0|0) $ und $ (1.0|0) $%
\item%
\newline\vspace{0.5cm}$f(x)=-4(x+5)^2 +4 \Rightarrow $ Zwei Nullstellen $: (-4.0|0) $ und $ (-6.0|0) $%
\item%
\newline\vspace{0.5cm}$f(x)=-(x-5)^2 -1 \Rightarrow  $ Keine Lösung/Keine Nullstellen $ $%
\item%
\newline\vspace{0.5cm}$f(x)=-2(x-8)^2 -1 \Rightarrow  $ Keine Lösung/Keine Nullstellen $ $%
\item%
\newline\vspace{0.5cm}$f(x)=(x-9)^2 +9 \Rightarrow  $ Keine Lösung/Keine Nullstellen $ $%
\end{enumerate}

%
\section{Normalform: Bestimme den Y{-}Achsenabschnitt}%
\label{sec:NormalformBestimmedenY{-}Achsenabschnitt}%
Für die Normalform ...ist der Y{-}Achsenabschnitt ....%
\begin{enumerate}[label=\alph*)]%
\item%
\newline\vspace{0.5cm}$f(x)=3x^2 + 30x + 6 \Rightarrow $ Y-Achsenabschnitt: $ (0|6) $%
\item%
\newline\vspace{0.5cm}$f(x)=2x^2 - 4x - 6 \Rightarrow $ Y-Achsenabschnitt: $ (0|-6) $%
\item%
\newline\vspace{0.5cm}$f(x)=-x^2 - 5 \Rightarrow $ Y-Achsenabschnitt: $ (0|-5) $%
\item%
\newline\vspace{0.5cm}$f(x)=-3x^2 + 24x - 2 \Rightarrow $ Y-Achsenabschnitt: $ (0|-2) $%
\item%
\newline\vspace{0.5cm}$f(x)=-4x^2 + 80x - 5 \Rightarrow $ Y-Achsenabschnitt: $ (0|-5) $%
\item%
\newline\vspace{0.5cm}$f(x)=-4x^2 + 24x - 3 \Rightarrow $ Y-Achsenabschnitt: $ (0|-3) $%
\item%
\newline\vspace{0.5cm}$f(x)=x^2 + 2x \Rightarrow $ Y-Achsenabschnitt: $ (0|0) $%
\item%
\newline\vspace{0.5cm}$f(x)=4x^2 + 16x - 2 \Rightarrow $ Y-Achsenabschnitt: $ (0|-2) $%
\item%
\newline\vspace{0.5cm}$f(x)=-3x^2 + 18x - 8 \Rightarrow $ Y-Achsenabschnitt: $ (0|-8) $%
\item%
\newline\vspace{0.5cm}$f(x)=-x^2 + 8 \Rightarrow $ Y-Achsenabschnitt: $ (0|8) $%
\end{enumerate}

%
\section{Scheitelpunktform: Bestimme den Y{-}Achsenabschnitt}%
\label{sec:ScheitelpunktformBestimmedenY{-}Achsenabschnitt}%
Für die Scheitelpunktform ...ist der Y{-}Achsenabschnitt ....%
\begin{enumerate}[label=\alph*)]%
\item%
\newline\vspace{0.5cm}$f(x)=-2(x+8)^2 +6 \Rightarrow $ Y-Achsenabschnitt: $ (0|-122) $%
\item%
\newline\vspace{0.5cm}$f(x)=-2(x-3)^2 \Rightarrow $ Y-Achsenabschnitt: $ (0|-18) $%
\item%
\newline\vspace{0.5cm}$f(x)=3(x-8)^2 +10 \Rightarrow $ Y-Achsenabschnitt: $ (0|202) $%
\item%
\newline\vspace{0.5cm}$f(x)=3(x-6)^2 -2 \Rightarrow $ Y-Achsenabschnitt: $ (0|106) $%
\item%
\newline\vspace{0.5cm}$f(x)=(x-2)^2 -4 \Rightarrow $ Y-Achsenabschnitt: $ (0|0) $%
\item%
\newline\vspace{0.5cm}$f(x)=4(x-6)^2 +2 \Rightarrow $ Y-Achsenabschnitt: $ (0|146) $%
\item%
\newline\vspace{0.5cm}$f(x)=-3(x-9)^2 +7 \Rightarrow $ Y-Achsenabschnitt: $ (0|-236) $%
\item%
\newline\vspace{0.5cm}$f(x)=-(x)^2 +4 \Rightarrow $ Y-Achsenabschnitt: $ (0|4) $%
\item%
\newline\vspace{0.5cm}$f(x)=-(x-10)^2 -4 \Rightarrow $ Y-Achsenabschnitt: $ (0|-104) $%
\item%
\newline\vspace{0.5cm}$f(x)=3(x+2)^2 -2 \Rightarrow $ Y-Achsenabschnitt: $ (0|10) $%
\end{enumerate}

%
\section{Faktorisierte Form: Bestimme den Y{-}Achsenabschnitt}%
\label{sec:FaktorisierteFormBestimmedenY{-}Achsenabschnitt}%
Für die Faktorisierte Form ...ist der Y{-}Achsenabschnitt ....%
\begin{enumerate}[label=\alph*)]%
\item%
\newline\vspace{0.5cm}$f(x)=-(x-3)(x+1) \Rightarrow $ Y-Achsenabschnitt: $ (0|3) $%
\item%
\newline\vspace{0.5cm}$f(x)=(x+1)(x+8) \Rightarrow $ Y-Achsenabschnitt: $ (0|8) $%
\item%
\newline\vspace{0.5cm}$f(x)=-4(x+6)(x+9) \Rightarrow $ Y-Achsenabschnitt: $ (0|-216) $%
\item%
\newline\vspace{0.5cm}$f(x)=-2(x+1)(x-2) \Rightarrow $ Y-Achsenabschnitt: $ (0|4) $%
\item%
\newline\vspace{0.5cm}$f(x)=-2(x-3)(x+2) \Rightarrow $ Y-Achsenabschnitt: $ (0|12) $%
\item%
\newline\vspace{0.5cm}$f(x)=-3(x+10)(x+5) \Rightarrow $ Y-Achsenabschnitt: $ (0|-150) $%
\item%
\newline\vspace{0.5cm}$f(x)=2(x-8)(x+7) \Rightarrow $ Y-Achsenabschnitt: $ (0|-112) $%
\item%
\newline\vspace{0.5cm}$f(x)=4(x)(x+2) \Rightarrow $ Y-Achsenabschnitt: $ (0|0) $%
\item%
\newline\vspace{0.5cm}$f(x)=3(x+8)(x) \Rightarrow $ Y-Achsenabschnitt: $ (0|0) $%
\item%
\newline\vspace{0.5cm}$f(x)=3(x+8)(x-8) \Rightarrow $ Y-Achsenabschnitt: $ (0|-192) $%
\end{enumerate}

%
\section{Finde die Funktionsgleichung}%
\label{sec:FindedieFunktionsgleichung}%
Die Funktionsgleichung ist ...%
\begin{enumerate}[label=\alph*)]%
\item%
 Nullstellen $-10$ und $4$ und Scheitelpunkt $(-3.0|-98.0) \Rightarrow f(x)=2x^2 + 12x - 80 ; f(x)=2(x+10)(x-4) ; f(x)=2(x+3.0)^2 -98.0$%
\item%
 Punkt $(28|-777)$ und Scheitelpunkt $(-1.0|64.0) \Rightarrow f(x)=-x^2 - 2x + 63 ; f(x)=-(x+9)(x-7) ; f(x)=-(x+1.0)^2 +64.0$%
\item%
 Punkt Scheitelpunkt $(-10.0|0.0)$ und Y-Achsenabschnitt $-100 \Rightarrow f(x)=-x^2 - 20x - 100 ; f(x)=-(x+10)(x+10) ; f(x)=-(x+10.0)^2$%
\item%
 Die Funktion geht durch den Punkt $(-19|-798)$ und hat die Nullstellen $2$ und $0 \Rightarrow f(x)=-2x^2 + 4x ; f(x)=-2(x-2)(x) ; f(x)=-2(x-1.0)^2 +2.0$%
\item%
 Punkte $(-43|4624),(-44|4900)$ und $(16|2500) \Rightarrow f(x)=4x^2 + 72x + 324 ; f(x)=4(x+9)(x+9) ; f(x)=4(x+9.0)^2$%
\item%
 Punkte $(95|19530),(-76|10296)$ und $(51|5978) \Rightarrow f(x)=2x^2 + 16x - 40 ; f(x)=2(x-2)(x+10) ; f(x)=2(x+4.0)^2 -72.0$%
\item%
 Punkte $(64|-9520),(20|-1248)$ und $(24|-1680) \Rightarrow f(x)=-2x^2 - 20x - 48 ; f(x)=-2(x+4)(x+6) ; f(x)=-2(x+5.0)^2 +2.0$%
\item%
 Nullstellen $-8$ und $0$ und Scheitelpunkt $(-4.0|-32.0) \Rightarrow f(x)=2x^2 + 16x ; f(x)=2(x+8)(x) ; f(x)=2(x+4.0)^2 -32.0$%
\item%
 Punkte $(-34|-1015),(75|-5920)$ und $(-73|-5032) \Rightarrow f(x)=-x^2 - 4x + 5 ; f(x)=-(x-1)(x+5) ; f(x)=-(x+2.0)^2 +9.0$%
\item%
 Punkt Scheitelpunkt $(-5.0|32.0)$ und Y-Achsenabschnitt $-18 \Rightarrow f(x)=-2x^2 - 20x - 18 ; f(x)=-2(x+9)(x+1) ; f(x)=-2(x+5.0)^2 +32.0$%
\item%
 Nullstellen $0$ und $8$ und Scheitelpunkt $(4.0|-48.0) \Rightarrow f(x)=3x^2 - 24x ; f(x)=3(x)(x-8) ; f(x)=3(x-4.0)^2 -48.0$%
\item%
 Nullstellen $-4$ und $-6$ und Scheitelpunkt $(-5.0|1.0) \Rightarrow f(x)=-x^2 - 10x - 24 ; f(x)=-(x+4)(x+6) ; f(x)=-(x+5.0)^2 +1.0$%
\item%
 Nullstellen $0$ und $6$ und Scheitelpunkt $(3.0|-36.0) \Rightarrow f(x)=4x^2 - 24x ; f(x)=4(x)(x-6) ; f(x)=4(x-3.0)^2 -36.0$%
\item%
 Nullstellen $5$ und $9$ und Scheitelpunkt $(7.0|-8.0) \Rightarrow f(x)=2x^2 - 28x + 90 ; f(x)=2(x-5)(x-9) ; f(x)=2(x-7.0)^2 -8.0$%
\item%
 Punkte $(96|39008),(65|18300)$ und $(-26|1920) \Rightarrow f(x)=4x^2 + 24x - 160 ; f(x)=4(x-4)(x+10) ; f(x)=4(x+3.0)^2 -196.0$%
\item%
 Punkt $(56|11200)$ und Scheitelpunkt $(3.0|-36.0) \Rightarrow f(x)=4x^2 - 24x ; f(x)=4(x)(x-6) ; f(x)=4(x-3.0)^2 -36.0$%
\item%
 Punkte $(-47|-3774),(-95|-16830)$ und $(0|80) \Rightarrow f(x)=-2x^2 - 12x + 80 ; f(x)=-2(x-4)(x+10) ; f(x)=-2(x+3.0)^2 +98.0$%
\item%
 Punkte $(82|25920),(44|7072)$ und $(4|-288) \Rightarrow f(x)=4x^2 - 8x - 320 ; f(x)=4(x+8)(x-10) ; f(x)=4(x-1.0)^2 -324.0$%
\item%
 Punkt Scheitelpunkt $(-4.0|25.0)$ und Y-Achsenabschnitt $9 \Rightarrow f(x)=-x^2 - 8x + 9 ; f(x)=-(x+9)(x-1) ; f(x)=-(x+4.0)^2 +25.0$%
\item%
 Punkte $(-52|-2600),(10|-120)$ und $(75|-5775) \Rightarrow f(x)=-x^2 - 2x ; f(x)=-(x+2)(x) ; f(x)=-(x+1.0)^2 +1.0$%
\end{enumerate}

%
\end{document}