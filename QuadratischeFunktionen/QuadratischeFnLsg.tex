\documentclass{article}%
\usepackage[T1]{fontenc}%
\usepackage[utf8]{inputenc}%
\usepackage{lmodern}%
\usepackage{textcomp}%
\usepackage{lastpage}%
\usepackage{xcolor}%
\usepackage{enumitem}%
%
\title{Quadratische Funktionen - Lösungen \newline 4b8f0d73-64db-4785-a782-230215b40f77}%
\date{\today}%
%
\begin{document}%
\normalsize%
\maketitle%
\section{Normalform zu Scheitelpunktsform}%
\label{sec:NormalformzuScheitelpunktsform}%
Für die Normalform ... ist die Scheitelpunktsform ....%
\begin{enumerate}[label=\alph*)]%
\item%
\newline\vspace{0.5cm}$f(x)=-4x^2 + 16x\Leftrightarrow f(x)=-4(x+2.0)^2 +16.0$%
\item%
\newline\vspace{0.5cm}$f(x)=3x^2 - 48x + 10\Leftrightarrow f(x)=3(x+8.0)^2 -182.0$%
\item%
\newline\vspace{0.5cm}$f(x)=-x^2 + 10\Leftrightarrow f(x)=-(x)^2 +10.0$%
\end{enumerate}

%
\section{Scheitelpunktsform zu Faktorisierten Form}%
\label{sec:ScheitelpunktsformzuFaktorisiertenForm}%
Für die Scheitelpunktsform ... ist die Faktorisierte Form ....%
\begin{enumerate}[label=\alph*)]%
\item%
\newline\vspace{0.5cm}$f(x)=-4(x-1.0)^2 -256.0\Leftrightarrow f(x)=-4(x+9)(x-7)$%
\item%
\newline\vspace{0.5cm}$f(x)=3(x+3.5)^2 +36.75\Leftrightarrow f(x)=3(x)(x-7)$%
\item%
\newline\vspace{0.5cm}$f(x)=3(x+4.0)^2 +12.0\Leftrightarrow f(x)=3(x-6)(x-2)$%
\end{enumerate}

%
\section{Faktorisierte Form zu Normalform}%
\label{sec:FaktorisierteFormzuNormalform}%
Für die Faktorisierte Form ... ist die Normalform ....%
\begin{enumerate}[label=\alph*)]%
\item%
\newline\vspace{0.5cm}$f(x)=(x+5)(x+4)\Leftrightarrow f(x)=x^2 + 9x + 20$%
\item%
\newline\vspace{0.5cm}$f(x)=2(x-7)(x+1)\Leftrightarrow f(x)=2x^2 - 12x - 14$%
\item%
\newline\vspace{0.5cm}$f(x)=-2(x+6)(x-9)\Leftrightarrow f(x)=-2x^2 + 6x + 108$%
\end{enumerate}

%
\section{Normalform zu Faktorisierter Form}%
\label{sec:NormalformzuFaktorisierterForm}%
Für die Normalform ... ist die Faktorisierte Form ....%
\begin{enumerate}[label=\alph*)]%
\item%
\newline\vspace{0.5cm}$f(x)=-4x^2 - 60x - 224\Leftrightarrow f(x)=-4(x+8)(x+7)$%
\item%
\newline\vspace{0.5cm}$f(x)=x^2 - 5x + 6\Leftrightarrow f(x)=(x-2)(x-3)$%
\item%
\newline\vspace{0.5cm}$f(x)=-2x^2 + 28x - 90\Leftrightarrow f(x)=-2(x-9)(x-5)$%
\end{enumerate}

%
\section{Scheitelpunktform zu Normalform}%
\label{sec:ScheitelpunktformzuNormalform}%
Für die Scheitelpunktform ... ist die Normalform ....%
\begin{enumerate}[label=\alph*)]%
\item%
\newline\vspace{0.5cm}$f(x)=2(x-10)^2 +2\Leftrightarrow f(x)=2x^2 + 40x + 202$%
\item%
\newline\vspace{0.5cm}$f(x)=-(x+2)^2 +7\Leftrightarrow f(x)=-x^2 + 4x + 3$%
\item%
\newline\vspace{0.5cm}$f(x)=-(x-9)^2 -7\Leftrightarrow f(x)=-x^2 - 18x - 88$%
\end{enumerate}

%
\section{Faktorisierte Form zu Scheitelpunktform}%
\label{sec:FaktorisierteFormzuScheitelpunktform}%
Für die Faktorisierte Form ... ist die Scheitelpunktform ....%
\begin{enumerate}[label=\alph*)]%
\item%
\newline\vspace{0.5cm}$f(x)=2(x-7)(x-7)\Leftrightarrow f(x)=2(x+7.0)^2$%
\item%
\newline\vspace{0.5cm}$f(x)=(x-4)(x+1)\Leftrightarrow f(x)=(x+1.5)^2 +6.25$%
\item%
\newline\vspace{0.5cm}$f(x)=2(x)(x+8)\Leftrightarrow f(x)=2(x-4.0)^2 +32.0$%
\end{enumerate}

%
\section{Scheitelpunktform: Bestimme den Scheitelpunkt}%
\label{sec:ScheitelpunktformBestimmedenScheitelpunkt}%
Für die Scheitelpunktform ... ist der Scheitelpunkt ....%
\begin{enumerate}[label=\alph*)]%
\item%
\newline\vspace{0.5cm}$f(x)=4(x+9)^2 +2 \Rightarrow SP(9|2) $%
\item%
\newline\vspace{0.5cm}$f(x)=-2(x-9)^2 -3 \Rightarrow SP(-9|-3) $%
\item%
\newline\vspace{0.5cm}$f(x)=-2(x-8)^2 \Rightarrow SP(-8|0) $%
\end{enumerate}

%
\section{Normalform: Bestimme den Scheitelpunkt}%
\label{sec:NormalformBestimmedenScheitelpunkt}%
Für die Normalform ... ist der Scheitelpunkt ....%
\begin{enumerate}[label=\alph*)]%
\item%
\newline\vspace{0.5cm}$f(x)=-2x^2 - 4 \Rightarrow SP(0|-4) $%
\item%
\newline\vspace{0.5cm}$f(x)=3x^2 + 18x + 23 \Rightarrow SP(-3|-4) $%
\item%
\newline\vspace{0.5cm}$f(x)=2x^2 - 24x + 65 \Rightarrow SP(6|-7) $%
\end{enumerate}

%
\section{Faktorisierte Form: Bestimme den Scheitelpunkt}%
\label{sec:FaktorisierteFormBestimmedenScheitelpunkt}%
Für die Faktorisierte Form ... ist der Scheitelpunkt ....%
\begin{enumerate}[label=\alph*)]%
\item%
\newline\vspace{0.5cm}$f(x)=-4(x+4)(x-4) \Rightarrow SP(-0.0|-64.0) $%
\item%
\newline\vspace{0.5cm}$f(x)=-2(x+2)(x-5) \Rightarrow SP(1.5|-24.5) $%
\item%
\newline\vspace{0.5cm}$f(x)=-4(x+2)(x+8) \Rightarrow SP(-5.0|-36.0) $%
\end{enumerate}

%
\section{Faktorisierte Form: Bestimme die Nullstellen}%
\label{sec:FaktorisierteFormBestimmedieNullstellen}%
Für die Faktorisierte Form ... sind die Nullstellen ....%
\begin{enumerate}[label=\alph*)]%
\item%
\newline\vspace{0.5cm}$f(x)=-4(x-7)(x-2) \Rightarrow $ Zwei Nullstellen $: (7|0) $ und $ (2|0) $%
\item%
\newline\vspace{0.5cm}$f(x)=-3(x+4)(x-1) \Rightarrow $ Zwei Nullstellen $: (-4|0) $ und $ (1|0) $%
\item%
\newline\vspace{0.5cm}$f(x)=2(x-3)(x+8) \Rightarrow $ Zwei Nullstellen $: (3|0) $ und $ (-8|0) $%
\end{enumerate}

%
\section{Normalform: Bestimme die Nullstellen}%
\label{sec:NormalformBestimmedieNullstellen}%
Für die Normalform ... sind die Nullstellen ....%
\begin{enumerate}[label=\alph*)]%
\item%
\newline\vspace{0.5cm}$f(x)=-x^2 + 16x - 2 \Rightarrow $ Zwei Nullstellen $: (-0.12599212598818887|0) $ und $ (-15.874007874011811|0) $%
\item%
\newline\vspace{0.5cm}$f(x)=2x^2 - 16x + 9 \Rightarrow $ Zwei Nullstellen $: (-0.6088350084373659|0) $ und $ (-7.3911649915626345|0) $%
\item%
\newline\vspace{0.5cm}$f(x)=-x^2 + 8x \Rightarrow $ Zwei Nullstellen $: (0.0|0) $ und $ (-8.0|0) $%
\item%
\newline\vspace{0.5cm}$f(x)=-3x^2 + 24x - 8 \Rightarrow $ Zwei Nullstellen $: (-0.3485162832988924|0) $ und $ (-7.651483716701108|0) $%
\item%
\newline\vspace{0.5cm}$f(x)=3x^2 + 42x + 7 \Rightarrow $ Zwei Nullstellen $: (13.831300510639732|0) $ und $ (0.16869948936026802|0) $%
\item%
\newline\vspace{0.5cm}$f(x)=-4x^2 - 8x - 10 \Rightarrow  $ Keine Lösung/Keine Nullstellen $ $%
\end{enumerate}

%
\section{Scheitelpunktform: Bestimme die Nullstellen}%
\label{sec:ScheitelpunktformBestimmedieNullstellen}%
Für die Scheitelpunktform ... sind die Nullstellen ....%
\begin{enumerate}[label=\alph*)]%
\item%
\newline\vspace{0.5cm}$f(x)=(x+9)^2 -2 \Rightarrow $ Zwei Nullstellen $: (-7.585786437626905|0) $ und $ (-10.414213562373096|0) $%
\item%
\newline\vspace{0.5cm}$f(x)=2(x-2)^2 +2 \Rightarrow  $ Keine Lösung/Keine Nullstellen $ $%
\item%
\newline\vspace{0.5cm}$f(x)=-(x)^2 +9 \Rightarrow $ Zwei Nullstellen $: (3.0|0) $ und $ (-3.0|0) $%
\item%
\newline\vspace{0.5cm}$f(x)=-(x+2)^2 +7 \Rightarrow $ Zwei Nullstellen $: (0.6457513110645907|0) $ und $ (-4.645751311064591|0) $%
\item%
\newline\vspace{0.5cm}$f(x)=(x+5)^2 +5 \Rightarrow  $ Keine Lösung/Keine Nullstellen $ $%
\item%
\newline\vspace{0.5cm}$f(x)=-3(x)^2 -4 \Rightarrow  $ Keine Lösung/Keine Nullstellen $ $%
\end{enumerate}

%
\section{Normalform: Bestimme den Y{-}Achsenabschnitt}%
\label{sec:NormalformBestimmedenY{-}Achsenabschnitt}%
Für die Normalform ...ist der Y{-}Achsenabschnitt ....%
\begin{enumerate}[label=\alph*)]%
\item%
\newline\vspace{0.5cm}$f(x)=4x^2 + 80x + 8 \Rightarrow $ Y-Achsenabschnitt: $ (0|8) $%
\item%
\newline\vspace{0.5cm}$f(x)=2x^2 - 36x - 4 \Rightarrow $ Y-Achsenabschnitt: $ (0|-4) $%
\item%
\newline\vspace{0.5cm}$f(x)=-2x^2 - 12x \Rightarrow $ Y-Achsenabschnitt: $ (0|0) $%
\end{enumerate}

%
\section{Scheitelpunktform: Bestimme den Y{-}Achsenabschnitt}%
\label{sec:ScheitelpunktformBestimmedenY{-}Achsenabschnitt}%
Für die Scheitelpunktform ...ist der Y{-}Achsenabschnitt ....%
\begin{enumerate}[label=\alph*)]%
\item%
\newline\vspace{0.5cm}$f(x)=-3(x)^2 -6 \Rightarrow $ Y-Achsenabschnitt: $ (0|-6) $%
\item%
\newline\vspace{0.5cm}$f(x)=3(x-4)^2 \Rightarrow $ Y-Achsenabschnitt: $ (0|48) $%
\item%
\newline\vspace{0.5cm}$f(x)=-3(x+9)^2 -5 \Rightarrow $ Y-Achsenabschnitt: $ (0|-248) $%
\end{enumerate}

%
\section{Faktorisierte Form: Bestimme den Y{-}Achsenabschnitt}%
\label{sec:FaktorisierteFormBestimmedenY{-}Achsenabschnitt}%
Für die Faktorisierte Form ...ist der Y{-}Achsenabschnitt ....%
\begin{enumerate}[label=\alph*)]%
\item%
\newline\vspace{0.5cm}$f(x)=3(x+6)(x-7) \Rightarrow $ Y-Achsenabschnitt: $ (0|-126) $%
\item%
\newline\vspace{0.5cm}$f(x)=-2(x)(x) \Rightarrow $ Y-Achsenabschnitt: $ (0|0) $%
\item%
\newline\vspace{0.5cm}$f(x)=-(x-8)(x-4) \Rightarrow $ Y-Achsenabschnitt: $ (0|-32) $%
\end{enumerate}

%
\section{Finde die Funktionsgleichung}%
\label{sec:FindedieFunktionsgleichung}%
Die Funktionsgleichung ist ...%
\begin{enumerate}[label=\alph*)]%
\item%
 Punkt $(59|7198)$ und Scheitelpunkt $(-1.0|-2.0) \Rightarrow f(x)=2x^2 + 4x ; f(x)=2(x)(x+2) ; f(x)=2(x-1.0)^2 -2.0$%
\item%
 Punkte $(31|-1353),(65|-5025)$ und $(-42|-1280) \Rightarrow f(x)=-x^2 - 12x - 20 ; f(x)=-(x+2)(x+10) ; f(x)=-(x-6.0)^2 +16.0$%
\item%
 Nullstellen $-1$ und $7$ und Scheitelpunkt $(-3.0|16.0) \Rightarrow f(x)=-x^2 - 6x + 7 ; f(x)=-(x-1)(x+7) ; f(x)=-(x-3.0)^2 +16.0$%
\item%
 Punkt Scheitelpunkt $(-1.0|2.0)$ und Y-Achsenabschnitt $0 \Rightarrow f(x)=-2x^2 - 4x ; f(x)=-2(x+2)(x) ; f(x)=-2(x-1.0)^2 +2.0$%
\item%
 Nullstellen $-1$ und $-1$ und Scheitelpunkt $(1.0|0.0) \Rightarrow f(x)=-4x^2 + 8x - 4 ; f(x)=-4(x-1)(x-1) ; f(x)=-4(x+1.0)^2$%
\item%
 Nullstellen $-5$ und $3$ und Scheitelpunkt $(1.0|-16.0) \Rightarrow f(x)=x^2 - 2x - 15 ; f(x)=(x-5)(x+3) ; f(x)=(x+1.0)^2 -16.0$%
\item%
 Punkt Scheitelpunkt $(5.0|-100.0)$ und Y-Achsenabschnitt $0 \Rightarrow f(x)=4x^2 - 40x ; f(x)=4(x-10)(x) ; f(x)=4(x+5.0)^2 -100.0$%
\item%
 Nullstellen $7$ und $-1$ und Scheitelpunkt $(-3.0|-32.0) \Rightarrow f(x)=2x^2 + 12x - 14 ; f(x)=2(x+7)(x-1) ; f(x)=2(x-3.0)^2 -32.0$%
\item%
 Nullstellen $5$ und $3$ und Scheitelpunkt $(-4.0|-1.0) \Rightarrow f(x)=x^2 + 8x + 15 ; f(x)=(x+5)(x+3) ; f(x)=(x-4.0)^2 -1.0$%
\end{enumerate}

%
\end{document}