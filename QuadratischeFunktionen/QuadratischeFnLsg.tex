\documentclass{article}%
\usepackage[T1]{fontenc}%
\usepackage[utf8]{inputenc}%
\usepackage{lmodern}%
\usepackage{textcomp}%
\usepackage{lastpage}%
\usepackage{xcolor}%
\usepackage{enumitem}%
%
\title{Quadratische Funktionen - Lösungen \newline 79a1cfb4-0229-4423-a3a2-f5f020ebaf9d}%
\date{\today}%
%
\begin{document}%
\normalsize%
\maketitle%
\section{Normalform zu Scheitelpunktsform}%
\label{sec:NormalformzuScheitelpunktsform}%
Für die Normalform ... ist die Scheitelpunktsform ....%
\begin{enumerate}[label=\alph*)]%
\item%
\newline\vspace{0.5cm}$f(x)=-4x^2 + 64x\Leftrightarrow f(x)=-4(x-8.0)^2 +256.0$%
\item%
\newline\vspace{0.5cm}$f(x)=-4x^2 - 48x + 4\Leftrightarrow f(x)=-4(x+6.0)^2 +148.0$%
\item%
\newline\vspace{0.5cm}$f(x)=-3x^2 + 5\Leftrightarrow f(x)=-3(x)^2 +5.0$%
\item%
\newline\vspace{0.5cm}$f(x)=-3x^2 + 6x + 4\Leftrightarrow f(x)=-3(x-1.0)^2 +7.0$%
\item%
\newline\vspace{0.5cm}$f(x)=2x^2 + 28x - 10\Leftrightarrow f(x)=2(x+7.0)^2 -108.0$%
\item%
\newline\vspace{0.5cm}$f(x)=-4x^2 + 24x - 2\Leftrightarrow f(x)=-4(x-3.0)^2 +34.0$%
\item%
\newline\vspace{0.5cm}$f(x)=x^2 - 8x - 9\Leftrightarrow f(x)=(x-4.0)^2 -25.0$%
\item%
\newline\vspace{0.5cm}$f(x)=-2x^2 - 32x\Leftrightarrow f(x)=-2(x+8.0)^2 +128.0$%
\item%
\newline\vspace{0.5cm}$f(x)=-x^2 + 4x + 8\Leftrightarrow f(x)=-(x-2.0)^2 +12.0$%
\item%
\newline\vspace{0.5cm}$f(x)=-2x^2 + 24x + 1\Leftrightarrow f(x)=-2(x-6.0)^2 +73.0$%
\item%
\newline\vspace{0.5cm}$f(x)=-2x^2 - 28x + 10\Leftrightarrow f(x)=-2(x+7.0)^2 +108.0$%
\item%
\newline\vspace{0.5cm}$f(x)=-2x^2 - 32x - 2\Leftrightarrow f(x)=-2(x+8.0)^2 +126.0$%
\item%
\newline\vspace{0.5cm}$f(x)=x^2 - 18x + 2\Leftrightarrow f(x)=(x-9.0)^2 -79.0$%
\item%
\newline\vspace{0.5cm}$f(x)=2x^2 - 36x - 4\Leftrightarrow f(x)=2(x-9.0)^2 -166.0$%
\item%
\newline\vspace{0.5cm}$f(x)=x^2 + 12x\Leftrightarrow f(x)=(x+6.0)^2 -36.0$%
\item%
\newline\vspace{0.5cm}$f(x)=-x^2 + 10x + 3\Leftrightarrow f(x)=-(x-5.0)^2 +28.0$%
\item%
\newline\vspace{0.5cm}$f(x)=-4x^2 + 72x + 4\Leftrightarrow f(x)=-4(x-9.0)^2 +328.0$%
\item%
\newline\vspace{0.5cm}$f(x)=4x^2 + 32x - 10\Leftrightarrow f(x)=4(x+4.0)^2 -74.0$%
\item%
\newline\vspace{0.5cm}$f(x)=-x^2 + 16x - 5\Leftrightarrow f(x)=-(x-8.0)^2 +59.0$%
\item%
\newline\vspace{0.5cm}$f(x)=-2x^2 + 12x - 8\Leftrightarrow f(x)=-2(x-3.0)^2 +10.0$%
\end{enumerate}

%
\section{Scheitelpunktsform zu Faktorisierten Form}%
\label{sec:ScheitelpunktsformzuFaktorisiertenForm}%
Für die Scheitelpunktsform ... ist die Faktorisierte Form ....%
\begin{enumerate}[label=\alph*)]%
\item%
\newline\vspace{0.5cm}$f(x)=-(x-3.5)^2 +42.25\Leftrightarrow f(x)=-(x-10)(x+3)$%
\item%
\newline\vspace{0.5cm}$f(x)=3(x-4.0)^2 -108.0\Leftrightarrow f(x)=3(x-10)(x+2)$%
\item%
\newline\vspace{0.5cm}$f(x)=-2(x+4.5)^2 +24.5\Leftrightarrow f(x)=-2(x+8)(x+1)$%
\item%
\newline\vspace{0.5cm}$f(x)=-(x-5.5)^2 +6.25\Leftrightarrow f(x)=-(x-8)(x-3)$%
\item%
\newline\vspace{0.5cm}$f(x)=2(x+9.0)^2 -2.0\Leftrightarrow f(x)=2(x+8)(x+10)$%
\item%
\newline\vspace{0.5cm}$f(x)=2(x-3.5)^2 -12.5\Leftrightarrow f(x)=2(x-6)(x-1)$%
\item%
\newline\vspace{0.5cm}$f(x)=-2(x-1.0)^2 +18.0\Leftrightarrow f(x)=-2(x-4)(x+2)$%
\item%
\newline\vspace{0.5cm}$f(x)=-4(x-5.5)^2 +9.0\Leftrightarrow f(x)=-4(x-7)(x-4)$%
\item%
\newline\vspace{0.5cm}$f(x)=-2(x+2.5)^2 +112.5\Leftrightarrow f(x)=-2(x-5)(x+10)$%
\item%
\newline\vspace{0.5cm}$f(x)=4(x-1.5)^2 -9.0\Leftrightarrow f(x)=4(x-3)(x)$%
\item%
\newline\vspace{0.5cm}$f(x)=(x+6.5)^2 -12.25\Leftrightarrow f(x)=(x+3)(x+10)$%
\item%
\newline\vspace{0.5cm}$f(x)=-3(x-1.5)^2 +216.75\Leftrightarrow f(x)=-3(x-10)(x+7)$%
\item%
\newline\vspace{0.5cm}$f(x)=-(x+10.0)^2\Leftrightarrow f(x)=-(x+10)(x+10)$%
\item%
\newline\vspace{0.5cm}$f(x)=-3(x-2.0)^2 +12.0\Leftrightarrow f(x)=-3(x)(x-4)$%
\item%
\newline\vspace{0.5cm}$f(x)=4(x)^2 -196.0\Leftrightarrow f(x)=4(x+7)(x-7)$%
\item%
\newline\vspace{0.5cm}$f(x)=(x-1.0)^2 -1.0\Leftrightarrow f(x)=(x-2)(x)$%
\item%
\newline\vspace{0.5cm}$f(x)=-4(x+1.5)^2 +25.0\Leftrightarrow f(x)=-4(x-1)(x+4)$%
\item%
\newline\vspace{0.5cm}$f(x)=-4(x+4.0)^2 +64.0\Leftrightarrow f(x)=-4(x+8)(x)$%
\item%
\newline\vspace{0.5cm}$f(x)=(x+6.5)^2 -0.25\Leftrightarrow f(x)=(x+6)(x+7)$%
\item%
\newline\vspace{0.5cm}$f(x)=-4(x-1.0)^2 +144.0\Leftrightarrow f(x)=-4(x-7)(x+5)$%
\end{enumerate}

%
\section{Faktorisierte Form zu Normalform}%
\label{sec:FaktorisierteFormzuNormalform}%
Für die Faktorisierte Form ... ist die Normalform ....%
\begin{enumerate}[label=\alph*)]%
\item%
\newline\vspace{0.5cm}$f(x)=-(x-1)(x+3)\Leftrightarrow f(x)=-x^2 - 2x + 3$%
\item%
\newline\vspace{0.5cm}$f(x)=4(x+9)(x-8)\Leftrightarrow f(x)=4x^2 + 4x - 288$%
\item%
\newline\vspace{0.5cm}$f(x)=-(x-5)(x-7)\Leftrightarrow f(x)=-x^2 + 12x - 35$%
\item%
\newline\vspace{0.5cm}$f(x)=4(x)(x-1)\Leftrightarrow f(x)=4x^2 - 4x$%
\item%
\newline\vspace{0.5cm}$f(x)=-4(x-3)(x)\Leftrightarrow f(x)=-4x^2 + 12x$%
\item%
\newline\vspace{0.5cm}$f(x)=-2(x-7)(x-3)\Leftrightarrow f(x)=-2x^2 + 20x - 42$%
\item%
\newline\vspace{0.5cm}$f(x)=-(x-4)(x+10)\Leftrightarrow f(x)=-x^2 - 6x + 40$%
\item%
\newline\vspace{0.5cm}$f(x)=-(x+8)(x-6)\Leftrightarrow f(x)=-x^2 - 2x + 48$%
\item%
\newline\vspace{0.5cm}$f(x)=3(x)(x+6)\Leftrightarrow f(x)=3x^2 + 18x$%
\item%
\newline\vspace{0.5cm}$f(x)=(x-3)(x-1)\Leftrightarrow f(x)=x^2 - 4x + 3$%
\item%
\newline\vspace{0.5cm}$f(x)=-2(x-5)(x-2)\Leftrightarrow f(x)=-2x^2 + 14x - 20$%
\item%
\newline\vspace{0.5cm}$f(x)=3(x-10)(x-1)\Leftrightarrow f(x)=3x^2 - 33x + 30$%
\item%
\newline\vspace{0.5cm}$f(x)=3(x+8)(x-4)\Leftrightarrow f(x)=3x^2 + 12x - 96$%
\item%
\newline\vspace{0.5cm}$f(x)=4(x+6)(x-8)\Leftrightarrow f(x)=4x^2 - 8x - 192$%
\item%
\newline\vspace{0.5cm}$f(x)=(x+7)(x-3)\Leftrightarrow f(x)=x^2 + 4x - 21$%
\item%
\newline\vspace{0.5cm}$f(x)=4(x-8)(x-1)\Leftrightarrow f(x)=4x^2 - 36x + 32$%
\item%
\newline\vspace{0.5cm}$f(x)=-2(x)(x-8)\Leftrightarrow f(x)=-2x^2 + 16x$%
\item%
\newline\vspace{0.5cm}$f(x)=-3(x)(x)\Leftrightarrow f(x)=-3x^2$%
\item%
\newline\vspace{0.5cm}$f(x)=-(x-5)(x)\Leftrightarrow f(x)=-x^2 + 5x$%
\item%
\newline\vspace{0.5cm}$f(x)=-4(x-8)(x+4)\Leftrightarrow f(x)=-4x^2 + 16x + 128$%
\end{enumerate}

%
\section{Normalform zu Faktorisierter Form}%
\label{sec:NormalformzuFaktorisierterForm}%
Für die Normalform ... ist die Faktorisierte Form ....%
\begin{enumerate}[label=\alph*)]%
\item%
\newline\vspace{0.5cm}$f(x)=2x^2 + 8x - 120\Leftrightarrow f(x)=2(x-6)(x+10)$%
\item%
\newline\vspace{0.5cm}$f(x)=3x^2 - 3x\Leftrightarrow f(x)=3(x-1)(x)$%
\item%
\newline\vspace{0.5cm}$f(x)=-x^2 + x\Leftrightarrow f(x)=-(x)(x-1)$%
\item%
\newline\vspace{0.5cm}$f(x)=x^2 - 12x + 20\Leftrightarrow f(x)=(x-10)(x-2)$%
\item%
\newline\vspace{0.5cm}$f(x)=-3x^2 + 18x + 81\Leftrightarrow f(x)=-3(x+3)(x-9)$%
\item%
\newline\vspace{0.5cm}$f(x)=4x^2 - 8x - 320\Leftrightarrow f(x)=4(x-10)(x+8)$%
\item%
\newline\vspace{0.5cm}$f(x)=-4x^2\Leftrightarrow f(x)=-4(x)(x)$%
\item%
\newline\vspace{0.5cm}$f(x)=2x^2 + 22x + 36\Leftrightarrow f(x)=2(x+2)(x+9)$%
\item%
\newline\vspace{0.5cm}$f(x)=-3x^2 - 12x + 135\Leftrightarrow f(x)=-3(x-5)(x+9)$%
\item%
\newline\vspace{0.5cm}$f(x)=3x^2 - 27x + 42\Leftrightarrow f(x)=3(x-7)(x-2)$%
\item%
\newline\vspace{0.5cm}$f(x)=2x^2 - 6x - 80\Leftrightarrow f(x)=2(x-8)(x+5)$%
\item%
\newline\vspace{0.5cm}$f(x)=-4x^2 + 196\Leftrightarrow f(x)=-4(x-7)(x+7)$%
\item%
\newline\vspace{0.5cm}$f(x)=2x^2 + 4x - 160\Leftrightarrow f(x)=2(x-8)(x+10)$%
\item%
\newline\vspace{0.5cm}$f(x)=3x^2 + 6x\Leftrightarrow f(x)=3(x+2)(x)$%
\item%
\newline\vspace{0.5cm}$f(x)=x^2 - 4x - 5\Leftrightarrow f(x)=(x+1)(x-5)$%
\item%
\newline\vspace{0.5cm}$f(x)=-2x^2 + 14x - 24\Leftrightarrow f(x)=-2(x-4)(x-3)$%
\item%
\newline\vspace{0.5cm}$f(x)=-x^2 - 2x\Leftrightarrow f(x)=-(x)(x+2)$%
\item%
\newline\vspace{0.5cm}$f(x)=-4x^2 - 4x + 360\Leftrightarrow f(x)=-4(x+10)(x-9)$%
\item%
\newline\vspace{0.5cm}$f(x)=-x^2\Leftrightarrow f(x)=-(x)(x)$%
\item%
\newline\vspace{0.5cm}$f(x)=4x^2 - 8x - 32\Leftrightarrow f(x)=4(x+2)(x-4)$%
\end{enumerate}

%
\section{Scheitelpunktform zu Normalform}%
\label{sec:ScheitelpunktformzuNormalform}%
Für die Scheitelpunktform ... ist die Normalform ....%
\begin{enumerate}[label=\alph*)]%
\item%
\newline\vspace{0.5cm}$f(x)=3(x+2)^2 +9\Leftrightarrow f(x)=3x^2 + 12x + 21$%
\item%
\newline\vspace{0.5cm}$f(x)=-2(x-2)^2\Leftrightarrow f(x)=-2x^2 + 8x - 8$%
\item%
\newline\vspace{0.5cm}$f(x)=-4(x-6)^2 -3\Leftrightarrow f(x)=-4x^2 + 48x - 147$%
\item%
\newline\vspace{0.5cm}$f(x)=2(x)^2\Leftrightarrow f(x)=2x^2$%
\item%
\newline\vspace{0.5cm}$f(x)=-3(x-5)^2 +4\Leftrightarrow f(x)=-3x^2 + 30x - 71$%
\item%
\newline\vspace{0.5cm}$f(x)=-4(x+9)^2 +8\Leftrightarrow f(x)=-4x^2 - 72x - 316$%
\item%
\newline\vspace{0.5cm}$f(x)=-3(x+3)^2 -2\Leftrightarrow f(x)=-3x^2 - 18x - 29$%
\item%
\newline\vspace{0.5cm}$f(x)=2(x-5)^2 -2\Leftrightarrow f(x)=2x^2 - 20x + 48$%
\item%
\newline\vspace{0.5cm}$f(x)=4(x-3)^2 -9\Leftrightarrow f(x)=4x^2 - 24x + 27$%
\item%
\newline\vspace{0.5cm}$f(x)=4(x)^2 -6\Leftrightarrow f(x)=4x^2 - 6$%
\item%
\newline\vspace{0.5cm}$f(x)=-2(x-9)^2 -6\Leftrightarrow f(x)=-2x^2 + 36x - 168$%
\item%
\newline\vspace{0.5cm}$f(x)=2(x+2)^2 -9\Leftrightarrow f(x)=2x^2 + 8x - 1$%
\item%
\newline\vspace{0.5cm}$f(x)=2(x-10)^2 +10\Leftrightarrow f(x)=2x^2 - 40x + 210$%
\item%
\newline\vspace{0.5cm}$f(x)=2(x-2)^2 -3\Leftrightarrow f(x)=2x^2 - 8x + 5$%
\item%
\newline\vspace{0.5cm}$f(x)=4(x+1)^2 +1\Leftrightarrow f(x)=4x^2 + 8x + 5$%
\item%
\newline\vspace{0.5cm}$f(x)=3(x+6)^2 -4\Leftrightarrow f(x)=3x^2 + 36x + 104$%
\item%
\newline\vspace{0.5cm}$f(x)=(x+3)^2 -10\Leftrightarrow f(x)=x^2 + 6x - 1$%
\item%
\newline\vspace{0.5cm}$f(x)=-2(x-4)^2 -7\Leftrightarrow f(x)=-2x^2 + 16x - 39$%
\item%
\newline\vspace{0.5cm}$f(x)=3(x)^2 -10\Leftrightarrow f(x)=3x^2 - 10$%
\item%
\newline\vspace{0.5cm}$f(x)=-(x+4)^2 +4\Leftrightarrow f(x)=-x^2 - 8x - 12$%
\end{enumerate}

%
\section{Faktorisierte Form zu Scheitelpunktform}%
\label{sec:FaktorisierteFormzuScheitelpunktform}%
Für die Faktorisierte Form ... ist die Scheitelpunktform ....%
\begin{enumerate}[label=\alph*)]%
\item%
\newline\vspace{0.5cm}$f(x)=(x+9)(x+4)\Leftrightarrow f(x)=(x+6.5)^2 -6.25$%
\item%
\newline\vspace{0.5cm}$f(x)=-3(x+6)(x+10)\Leftrightarrow f(x)=-3(x+8.0)^2 +12.0$%
\item%
\newline\vspace{0.5cm}$f(x)=-2(x-3)(x+10)\Leftrightarrow f(x)=-2(x+3.5)^2 +84.5$%
\item%
\newline\vspace{0.5cm}$f(x)=3(x+6)(x+8)\Leftrightarrow f(x)=3(x+7.0)^2 -3.0$%
\item%
\newline\vspace{0.5cm}$f(x)=-(x+9)(x+3)\Leftrightarrow f(x)=-(x+6.0)^2 +9.0$%
\item%
\newline\vspace{0.5cm}$f(x)=-2(x+3)(x-6)\Leftrightarrow f(x)=-2(x-1.5)^2 +40.5$%
\item%
\newline\vspace{0.5cm}$f(x)=-2(x-5)(x-2)\Leftrightarrow f(x)=-2(x-3.5)^2 +4.5$%
\item%
\newline\vspace{0.5cm}$f(x)=-(x+5)(x)\Leftrightarrow f(x)=-(x+2.5)^2 +6.25$%
\item%
\newline\vspace{0.5cm}$f(x)=-(x-7)(x+10)\Leftrightarrow f(x)=-(x+1.5)^2 +72.25$%
\item%
\newline\vspace{0.5cm}$f(x)=4(x-10)(x-1)\Leftrightarrow f(x)=4(x-5.5)^2 -81.0$%
\item%
\newline\vspace{0.5cm}$f(x)=3(x-10)(x-1)\Leftrightarrow f(x)=3(x-5.5)^2 -60.75$%
\item%
\newline\vspace{0.5cm}$f(x)=4(x+10)(x-2)\Leftrightarrow f(x)=4(x+4.0)^2 -144.0$%
\item%
\newline\vspace{0.5cm}$f(x)=3(x+6)(x+10)\Leftrightarrow f(x)=3(x+8.0)^2 -12.0$%
\item%
\newline\vspace{0.5cm}$f(x)=3(x-2)(x-9)\Leftrightarrow f(x)=3(x-5.5)^2 -36.75$%
\item%
\newline\vspace{0.5cm}$f(x)=4(x-10)(x-5)\Leftrightarrow f(x)=4(x-7.5)^2 -25.0$%
\item%
\newline\vspace{0.5cm}$f(x)=2(x-6)(x+2)\Leftrightarrow f(x)=2(x-2.0)^2 -32.0$%
\item%
\newline\vspace{0.5cm}$f(x)=4(x+7)(x-9)\Leftrightarrow f(x)=4(x-1.0)^2 -256.0$%
\item%
\newline\vspace{0.5cm}$f(x)=-4(x-8)(x+4)\Leftrightarrow f(x)=-4(x-2.0)^2 +144.0$%
\item%
\newline\vspace{0.5cm}$f(x)=4(x+8)(x+7)\Leftrightarrow f(x)=4(x+7.5)^2 -1.0$%
\item%
\newline\vspace{0.5cm}$f(x)=-(x+8)(x+1)\Leftrightarrow f(x)=-(x+4.5)^2 +12.25$%
\end{enumerate}

%
\section{Scheitelpunktform: Bestimme den Scheitelpunkt}%
\label{sec:ScheitelpunktformBestimmedenScheitelpunkt}%
Für die Scheitelpunktform ... ist der Scheitelpunkt ....%
\begin{enumerate}[label=\alph*)]%
\item%
\newline\vspace{0.5cm}$f(x)=-2(x+3)^2 +1 \Rightarrow SP(-3|1) $%
\item%
\newline\vspace{0.5cm}$f(x)=2(x-4)^2 +8 \Rightarrow SP(4|8) $%
\item%
\newline\vspace{0.5cm}$f(x)=-(x)^2 -2 \Rightarrow SP(0|-2) $%
\item%
\newline\vspace{0.5cm}$f(x)=3(x-3)^2 -4 \Rightarrow SP(3|-4) $%
\item%
\newline\vspace{0.5cm}$f(x)=2(x-8)^2 \Rightarrow SP(8|0) $%
\item%
\newline\vspace{0.5cm}$f(x)=-2(x-9)^2 +9 \Rightarrow SP(9|9) $%
\item%
\newline\vspace{0.5cm}$f(x)=-(x+10)^2 -9 \Rightarrow SP(-10|-9) $%
\item%
\newline\vspace{0.5cm}$f(x)=4(x+1)^2 -8 \Rightarrow SP(-1|-8) $%
\item%
\newline\vspace{0.5cm}$f(x)=3(x+7)^2 \Rightarrow SP(-7|0) $%
\item%
\newline\vspace{0.5cm}$f(x)=4(x-6)^2 -4 \Rightarrow SP(6|-4) $%
\item%
\newline\vspace{0.5cm}$f(x)=-(x+5)^2 -2 \Rightarrow SP(-5|-2) $%
\item%
\newline\vspace{0.5cm}$f(x)=2(x-1)^2 +7 \Rightarrow SP(1|7) $%
\item%
\newline\vspace{0.5cm}$f(x)=-(x)^2 -6 \Rightarrow SP(0|-6) $%
\item%
\newline\vspace{0.5cm}$f(x)=-3(x-1)^2 +8 \Rightarrow SP(1|8) $%
\item%
\newline\vspace{0.5cm}$f(x)=3(x+4)^2 +2 \Rightarrow SP(-4|2) $%
\item%
\newline\vspace{0.5cm}$f(x)=(x-10)^2 -6 \Rightarrow SP(10|-6) $%
\item%
\newline\vspace{0.5cm}$f(x)=3(x+1)^2 +8 \Rightarrow SP(-1|8) $%
\item%
\newline\vspace{0.5cm}$f(x)=4(x+4)^2 +3 \Rightarrow SP(-4|3) $%
\item%
\newline\vspace{0.5cm}$f(x)=-4(x-7)^2 +10 \Rightarrow SP(7|10) $%
\item%
\newline\vspace{0.5cm}$f(x)=-2(x+3)^2 +3 \Rightarrow SP(-3|3) $%
\end{enumerate}

%
\section{Normalform: Bestimme den Scheitelpunkt}%
\label{sec:NormalformBestimmedenScheitelpunkt}%
Für die Normalform ... ist der Scheitelpunkt ....%
\begin{enumerate}[label=\alph*)]%
\item%
\newline\vspace{0.5cm}$f(x)=3x^2 + 24x + 40 \Rightarrow SP(-4|-8) $%
\item%
\newline\vspace{0.5cm}$f(x)=-2x^2 - 12x - 11 \Rightarrow SP(-3|7) $%
\item%
\newline\vspace{0.5cm}$f(x)=3x^2 - 2 \Rightarrow SP(0|-2) $%
\item%
\newline\vspace{0.5cm}$f(x)=4x^2 + 24x + 36 \Rightarrow SP(-3|0) $%
\item%
\newline\vspace{0.5cm}$f(x)=-2x^2 + 10 \Rightarrow SP(0|10) $%
\item%
\newline\vspace{0.5cm}$f(x)=-4x^2 + 16x - 23 \Rightarrow SP(2|-7) $%
\item%
\newline\vspace{0.5cm}$f(x)=-4x^2 + 72x - 324 \Rightarrow SP(9|0) $%
\item%
\newline\vspace{0.5cm}$f(x)=2x^2 - 36x + 160 \Rightarrow SP(9|-2) $%
\item%
\newline\vspace{0.5cm}$f(x)=-x^2 + 6x - 18 \Rightarrow SP(3|-9) $%
\item%
\newline\vspace{0.5cm}$f(x)=-4x^2 - 48x - 151 \Rightarrow SP(-6|-7) $%
\item%
\newline\vspace{0.5cm}$f(x)=-3x^2 - 48x - 198 \Rightarrow SP(-8|-6) $%
\item%
\newline\vspace{0.5cm}$f(x)=2x^2 + 12x + 24 \Rightarrow SP(-3|6) $%
\item%
\newline\vspace{0.5cm}$f(x)=x^2 + 2x - 7 \Rightarrow SP(-1|-8) $%
\item%
\newline\vspace{0.5cm}$f(x)=-2x^2 + 16x - 40 \Rightarrow SP(4|-8) $%
\item%
\newline\vspace{0.5cm}$f(x)=3x^2 + 48x + 195 \Rightarrow SP(-8|3) $%
\item%
\newline\vspace{0.5cm}$f(x)=-2x^2 - 32x - 128 \Rightarrow SP(-8|0) $%
\item%
\newline\vspace{0.5cm}$f(x)=x^2 - 14x + 57 \Rightarrow SP(7|8) $%
\item%
\newline\vspace{0.5cm}$f(x)=2x^2 - 40x + 200 \Rightarrow SP(10|0) $%
\item%
\newline\vspace{0.5cm}$f(x)=-2x^2 - 4x - 8 \Rightarrow SP(-1|-6) $%
\item%
\newline\vspace{0.5cm}$f(x)=2x^2 \Rightarrow SP(0|0) $%
\end{enumerate}

%
\section{Faktorisierte Form: Bestimme den Scheitelpunkt}%
\label{sec:FaktorisierteFormBestimmedenScheitelpunkt}%
Für die Faktorisierte Form ... ist der Scheitelpunkt ....%
\begin{enumerate}[label=\alph*)]%
\item%
\newline\vspace{0.5cm}$f(x)=-(x+6)(x-2) \Rightarrow SP(-2.0|16.0) $%
\item%
\newline\vspace{0.5cm}$f(x)=-2(x-1)(x+7) \Rightarrow SP(-3.0|32.0) $%
\item%
\newline\vspace{0.5cm}$f(x)=-2(x-2)(x-1) \Rightarrow SP(1.5|0.5) $%
\item%
\newline\vspace{0.5cm}$f(x)=-2(x+3)(x-3) \Rightarrow SP(0.0|18.0) $%
\item%
\newline\vspace{0.5cm}$f(x)=-(x+2)(x+4) \Rightarrow SP(-3.0|1.0) $%
\item%
\newline\vspace{0.5cm}$f(x)=-2(x+6)(x+6) \Rightarrow SP(-6.0|0.0) $%
\item%
\newline\vspace{0.5cm}$f(x)=-2(x-5)(x+7) \Rightarrow SP(-1.0|72.0) $%
\item%
\newline\vspace{0.5cm}$f(x)=4(x-2)(x-6) \Rightarrow SP(4.0|-16.0) $%
\item%
\newline\vspace{0.5cm}$f(x)=-4(x+3)(x-9) \Rightarrow SP(3.0|144.0) $%
\item%
\newline\vspace{0.5cm}$f(x)=2(x+6)(x-6) \Rightarrow SP(0.0|-72.0) $%
\item%
\newline\vspace{0.5cm}$f(x)=-2(x-4)(x+7) \Rightarrow SP(-1.5|60.5) $%
\item%
\newline\vspace{0.5cm}$f(x)=4(x-4)(x+2) \Rightarrow SP(1.0|-36.0) $%
\item%
\newline\vspace{0.5cm}$f(x)=-3(x-4)(x) \Rightarrow SP(2.0|12.0) $%
\item%
\newline\vspace{0.5cm}$f(x)=-2(x-6)(x+5) \Rightarrow SP(0.5|60.5) $%
\item%
\newline\vspace{0.5cm}$f(x)=-3(x)(x-8) \Rightarrow SP(4.0|48.0) $%
\item%
\newline\vspace{0.5cm}$f(x)=(x+7)(x+3) \Rightarrow SP(-5.0|-4.0) $%
\item%
\newline\vspace{0.5cm}$f(x)=2(x)(x+4) \Rightarrow SP(-2.0|-8.0) $%
\item%
\newline\vspace{0.5cm}$f(x)=-4(x)(x+10) \Rightarrow SP(-5.0|100.0) $%
\item%
\newline\vspace{0.5cm}$f(x)=3(x+7)(x-4) \Rightarrow SP(-1.5|-90.75) $%
\item%
\newline\vspace{0.5cm}$f(x)=4(x+2)(x+1) \Rightarrow SP(-1.5|-1.0) $%
\end{enumerate}

%
\section{Faktorisierte Form: Bestimme die Nullstellen}%
\label{sec:FaktorisierteFormBestimmedieNullstellen}%
Für die Faktorisierte Form ... sind die Nullstellen ....%
\begin{enumerate}[label=\alph*)]%
\item%
\newline\vspace{0.5cm}$f(x)=-3(x+8)(x+7) \Rightarrow $ Zwei Nullstellen $: (-8|0) $ und $ (-7|0) $%
\item%
\newline\vspace{0.5cm}$f(x)=2(x-2)(x+6) \Rightarrow $ Zwei Nullstellen $: (2|0) $ und $ (-6|0) $%
\item%
\newline\vspace{0.5cm}$f(x)=-4(x-2)(x-4) \Rightarrow $ Zwei Nullstellen $: (2|0) $ und $ (4|0) $%
\item%
\newline\vspace{0.5cm}$f(x)=-3(x-4)(x+1) \Rightarrow $ Zwei Nullstellen $: (4|0) $ und $ (-1|0) $%
\item%
\newline\vspace{0.5cm}$f(x)=-4(x+9)(x-9) \Rightarrow $ Zwei Nullstellen $: (-9|0) $ und $ (9|0) $%
\item%
\newline\vspace{0.5cm}$f(x)=-3(x-5)(x-5) \Rightarrow $ Zwei Nullstellen $: (5|0) $ und $ (5|0) $%
\item%
\newline\vspace{0.5cm}$f(x)=-4(x+10)(x) \Rightarrow $ Zwei Nullstellen $: (-10|0) $ und $ (0|0) $%
\item%
\newline\vspace{0.5cm}$f(x)=-3(x-9)(x+1) \Rightarrow $ Zwei Nullstellen $: (9|0) $ und $ (-1|0) $%
\item%
\newline\vspace{0.5cm}$f(x)=3(x+5)(x-8) \Rightarrow $ Zwei Nullstellen $: (-5|0) $ und $ (8|0) $%
\item%
\newline\vspace{0.5cm}$f(x)=-(x-8)(x+6) \Rightarrow $ Zwei Nullstellen $: (8|0) $ und $ (-6|0) $%
\item%
\newline\vspace{0.5cm}$f(x)=-2(x-9)(x+2) \Rightarrow $ Zwei Nullstellen $: (9|0) $ und $ (-2|0) $%
\item%
\newline\vspace{0.5cm}$f(x)=-2(x-1)(x) \Rightarrow $ Zwei Nullstellen $: (1|0) $ und $ (0|0) $%
\item%
\newline\vspace{0.5cm}$f(x)=(x-4)(x-5) \Rightarrow $ Zwei Nullstellen $: (4|0) $ und $ (5|0) $%
\item%
\newline\vspace{0.5cm}$f(x)=-3(x)(x+1) \Rightarrow $ Zwei Nullstellen $: (0|0) $ und $ (-1|0) $%
\item%
\newline\vspace{0.5cm}$f(x)=2(x-10)(x) \Rightarrow $ Zwei Nullstellen $: (10|0) $ und $ (0|0) $%
\item%
\newline\vspace{0.5cm}$f(x)=2(x-1)(x) \Rightarrow $ Zwei Nullstellen $: (1|0) $ und $ (0|0) $%
\item%
\newline\vspace{0.5cm}$f(x)=(x-1)(x-3) \Rightarrow $ Zwei Nullstellen $: (1|0) $ und $ (3|0) $%
\item%
\newline\vspace{0.5cm}$f(x)=4(x-4)(x+9) \Rightarrow $ Zwei Nullstellen $: (4|0) $ und $ (-9|0) $%
\item%
\newline\vspace{0.5cm}$f(x)=(x+5)(x+5) \Rightarrow $ Zwei Nullstellen $: (-5|0) $ und $ (-5|0) $%
\item%
\newline\vspace{0.5cm}$f(x)=(x)(x) \Rightarrow $ Zwei Nullstellen $: (0|0) $ und $ (0|0) $%
\end{enumerate}

%
\section{Normalform: Bestimme die Nullstellen}%
\label{sec:NormalformBestimmedieNullstellen}%
Für die Normalform ... sind die Nullstellen ....%
\begin{enumerate}[label=\alph*)]%
\item%
\newline\vspace{0.5cm}$f(x)=-2x^2 + 28x - 10 \Rightarrow $ Zwei Nullstellen $: (13.6332495807108|0) $ und $ (0.3667504192892004|0) $%
\item%
\newline\vspace{0.5cm}$f(x)=-x^2 + 14x - 3 \Rightarrow $ Zwei Nullstellen $: (13.782329983125269|0) $ und $ (0.21767001687473186|0) $%
\item%
\newline\vspace{0.5cm}$f(x)=4x^2 - 40x - 4 \Rightarrow $ Zwei Nullstellen $: (10.099019513592784|0) $ und $ (-0.09901951359278449|0) $%
\item%
\newline\vspace{0.5cm}$f(x)=4x^2 + 4 \Rightarrow  $ Keine Lösung/Keine Nullstellen $ $%
\item%
\newline\vspace{0.5cm}$f(x)=-4x^2 + 24x - 3 \Rightarrow $ Zwei Nullstellen $: (5.872281323269014|0) $ und $ (0.1277186767309857|0) $%
\item%
\newline\vspace{0.5cm}$f(x)=-4x^2 - 64x - 8 \Rightarrow $ Zwei Nullstellen $: (-0.12599212598818887|0) $ und $ (-15.874007874011811|0) $%
\item%
\newline\vspace{0.5cm}$f(x)=-2x^2 + 20x - 4 \Rightarrow $ Zwei Nullstellen $: (9.79583152331272|0) $ und $ (0.2041684766872809|0) $%
\item%
\newline\vspace{0.5cm}$f(x)=3x^2 \Rightarrow $ Eine Nullstelle $: (0.0|0) $%
\item%
\newline\vspace{0.5cm}$f(x)=-3x^2 - 42x - 6 \Rightarrow $ Zwei Nullstellen $: (-0.1443453995989561|0) $ und $ (-13.855654600401044|0) $%
\item%
\newline\vspace{0.5cm}$f(x)=-3x^2 - 18x + 9 \Rightarrow $ Zwei Nullstellen $: (0.4641016151377544|0) $ und $ (-6.464101615137754|0) $%
\item%
\newline\vspace{0.5cm}$f(x)=-x^2 + 4x + 4 \Rightarrow $ Zwei Nullstellen $: (4.82842712474619|0) $ und $ (-0.8284271247461903|0) $%
\item%
\newline\vspace{0.5cm}$f(x)=3x^2 + 54x + 5 \Rightarrow $ Zwei Nullstellen $: (-0.09307385607507612|0) $ und $ (-17.906926143924924|0) $%
\item%
\newline\vspace{0.5cm}$f(x)=4x^2 - 8x - 9 \Rightarrow $ Zwei Nullstellen $: (2.802775637731995|0) $ und $ (-0.8027756377319946|0) $%
\item%
\newline\vspace{0.5cm}$f(x)=3x^2 + 54x + 7 \Rightarrow $ Zwei Nullstellen $: (-0.13057686956661918|0) $ und $ (-17.869423130433383|0) $%
\item%
\newline\vspace{0.5cm}$f(x)=-x^2 + 14x + 1 \Rightarrow $ Zwei Nullstellen $: (14.071067811865476|0) $ und $ (-0.0710678118654755|0) $%
\item%
\newline\vspace{0.5cm}$f(x)=3x^2 - 30x + 8 \Rightarrow $ Zwei Nullstellen $: (9.725815626252608|0) $ und $ (0.27418437374739213|0) $%
\item%
\newline\vspace{0.5cm}$f(x)=3x^2 - 48x - 1 \Rightarrow $ Zwei Nullstellen $: (16.020806277010642|0) $ und $ (-0.020806277010642305|0) $%
\item%
\newline\vspace{0.5cm}$f(x)=-3x^2 - 24x + 10 \Rightarrow $ Zwei Nullstellen $: (0.39696865275763926|0) $ und $ (-8.39696865275764|0) $%
\item%
\newline\vspace{0.5cm}$f(x)=-3x^2 - 54x + 2 \Rightarrow $ Zwei Nullstellen $: (0.03696114115063942|0) $ und $ (-18.03696114115064|0) $%
\item%
\newline\vspace{0.5cm}$f(x)=x^2 + 6x - 5 \Rightarrow $ Zwei Nullstellen $: (0.7416573867739413|0) $ und $ (-6.741657386773941|0) $%
\item%
\newline\vspace{0.5cm}$f(x)=-2x^2 - 16x - 2 \Rightarrow $ Zwei Nullstellen $: (-0.12701665379258298|0) $ und $ (-7.872983346207417|0) $%
\item%
\newline\vspace{0.5cm}$f(x)=3x^2 - 30x - 1 \Rightarrow $ Zwei Nullstellen $: (10.033222956847165|0) $ und $ (-0.03322295684716625|0) $%
\item%
\newline\vspace{0.5cm}$f(x)=x^2 - 14x \Rightarrow $ Zwei Nullstellen $: (14.0|0) $ und $ (0.0|0) $%
\item%
\newline\vspace{0.5cm}$f(x)=4x^2 - 24x - 6 \Rightarrow $ Zwei Nullstellen $: (6.24037034920393|0) $ und $ (-0.2403703492039302|0) $%
\item%
\newline\vspace{0.5cm}$f(x)=x^2 + 8x + 5 \Rightarrow $ Zwei Nullstellen $: (-0.6833752096446002|0) $ und $ (-7.3166247903554|0) $%
\item%
\newline\vspace{0.5cm}$f(x)=x^2 - 2x - 8 \Rightarrow $ Zwei Nullstellen $: (4.0|0) $ und $ (-2.0|0) $%
\item%
\newline\vspace{0.5cm}$f(x)=-x^2 - 4x + 2 \Rightarrow $ Zwei Nullstellen $: (0.4494897427831779|0) $ und $ (-4.449489742783178|0) $%
\item%
\newline\vspace{0.5cm}$f(x)=-3x^2 - 60x - 7 \Rightarrow $ Zwei Nullstellen $: (-0.11735527975093696|0) $ und $ (-19.88264472024906|0) $%
\item%
\newline\vspace{0.5cm}$f(x)=-2x^2 - 3 \Rightarrow  $ Keine Lösung/Keine Nullstellen $ $%
\item%
\newline\vspace{0.5cm}$f(x)=-4x^2 + 56x + 10 \Rightarrow $ Zwei Nullstellen $: (14.176350047203663|0) $ und $ (-0.17635004720366165|0) $%
\item%
\newline\vspace{0.5cm}$f(x)=-x^2 + 14x + 9 \Rightarrow $ Zwei Nullstellen $: (14.615773105863909|0) $ und $ (-0.6157731058639087|0) $%
\item%
\newline\vspace{0.5cm}$f(x)=-x^2 + 18x + 8 \Rightarrow $ Zwei Nullstellen $: (18.4339811320566|0) $ und $ (-0.43398113205660316|0) $%
\item%
\newline\vspace{0.5cm}$f(x)=-3x^2 - 42x + 5 \Rightarrow $ Zwei Nullstellen $: (0.11805216802087415|0) $ und $ (-14.118052168020874|0) $%
\item%
\newline\vspace{0.5cm}$f(x)=-x^2 + 12x - 7 \Rightarrow $ Zwei Nullstellen $: (11.385164807134505|0) $ und $ (0.6148351928654963|0) $%
\item%
\newline\vspace{0.5cm}$f(x)=-4x^2 - 5 \Rightarrow  $ Keine Lösung/Keine Nullstellen $ $%
\item%
\newline\vspace{0.5cm}$f(x)=-2x^2 - 12x \Rightarrow $ Zwei Nullstellen $: (0.0|0) $ und $ (-6.0|0) $%
\item%
\newline\vspace{0.5cm}$f(x)=-4x^2 + 16x + 2 \Rightarrow $ Zwei Nullstellen $: (4.121320343559642|0) $ und $ (-0.12132034355964239|0) $%
\item%
\newline\vspace{0.5cm}$f(x)=-2x^2 - 12x + 1 \Rightarrow $ Zwei Nullstellen $: (0.08220700148448801|0) $ und $ (-6.082207001484488|0) $%
\item%
\newline\vspace{0.5cm}$f(x)=2x^2 - 8x - 10 \Rightarrow $ Zwei Nullstellen $: (5.0|0) $ und $ (-1.0|0) $%
\item%
\newline\vspace{0.5cm}$f(x)=3x^2 + 6x \Rightarrow $ Zwei Nullstellen $: (0.0|0) $ und $ (-2.0|0) $%
\end{enumerate}

%
\section{Scheitelpunktform: Bestimme die Nullstellen}%
\label{sec:ScheitelpunktformBestimmedieNullstellen}%
Für die Scheitelpunktform ... sind die Nullstellen ....%
\begin{enumerate}[label=\alph*)]%
\item%
\newline\vspace{0.5cm}$f(x)=3(x+1)^2 -6 \Rightarrow $ Zwei Nullstellen $: (0.41421356237309515|0) $ und $ (-2.414213562373095|0) $%
\item%
\newline\vspace{0.5cm}$f(x)=-4(x+1)^2 -5 \Rightarrow  $ Keine Lösung/Keine Nullstellen $ $%
\item%
\newline\vspace{0.5cm}$f(x)=-2(x-8)^2 -5 \Rightarrow  $ Keine Lösung/Keine Nullstellen $ $%
\item%
\newline\vspace{0.5cm}$f(x)=3(x+2)^2 +1 \Rightarrow  $ Keine Lösung/Keine Nullstellen $ $%
\item%
\newline\vspace{0.5cm}$f(x)=-(x+5)^2 +1 \Rightarrow $ Zwei Nullstellen $: (-4.0|0) $ und $ (-6.0|0) $%
\item%
\newline\vspace{0.5cm}$f(x)=-3(x-1)^2 +7 \Rightarrow $ Zwei Nullstellen $: (2.5275252316519468|0) $ und $ (-0.5275252316519468|0) $%
\item%
\newline\vspace{0.5cm}$f(x)=-4(x+4)^2 -4 \Rightarrow  $ Keine Lösung/Keine Nullstellen $ $%
\item%
\newline\vspace{0.5cm}$f(x)=(x-4)^2 -7 \Rightarrow $ Zwei Nullstellen $: (6.645751311064591|0) $ und $ (1.3542486889354093|0) $%
\item%
\newline\vspace{0.5cm}$f(x)=4(x+1)^2 -10 \Rightarrow $ Zwei Nullstellen $: (0.5811388300841898|0) $ und $ (-2.58113883008419|0) $%
\item%
\newline\vspace{0.5cm}$f(x)=-3(x+7)^2 +10 \Rightarrow $ Zwei Nullstellen $: (-5.174258141649446|0) $ und $ (-8.825741858350554|0) $%
\item%
\newline\vspace{0.5cm}$f(x)=-2(x+2)^2 -3 \Rightarrow  $ Keine Lösung/Keine Nullstellen $ $%
\item%
\newline\vspace{0.5cm}$f(x)=-4(x-2)^2 +6 \Rightarrow $ Zwei Nullstellen $: (3.224744871391589|0) $ und $ (0.7752551286084111|0) $%
\item%
\newline\vspace{0.5cm}$f(x)=4(x-7)^2 \Rightarrow $ Eine Nullstelle $: (7|0) $%
\item%
\newline\vspace{0.5cm}$f(x)=2(x-9)^2 -5 \Rightarrow $ Zwei Nullstellen $: (10.581138830084189|0) $ und $ (7.41886116991581|0) $%
\item%
\newline\vspace{0.5cm}$f(x)=(x)^2 -2 \Rightarrow $ Zwei Nullstellen $: (1.4142135623730951|0) $ und $ (-1.4142135623730951|0) $%
\item%
\newline\vspace{0.5cm}$f(x)=(x+5)^2 +5 \Rightarrow  $ Keine Lösung/Keine Nullstellen $ $%
\item%
\newline\vspace{0.5cm}$f(x)=3(x-1)^2 +2 \Rightarrow  $ Keine Lösung/Keine Nullstellen $ $%
\item%
\newline\vspace{0.5cm}$f(x)=3(x+6)^2 -8 \Rightarrow $ Zwei Nullstellen $: (-4.3670068381445475|0) $ und $ (-7.6329931618554525|0) $%
\item%
\newline\vspace{0.5cm}$f(x)=-4(x-5)^2 +2 \Rightarrow $ Zwei Nullstellen $: (5.707106781186548|0) $ und $ (4.292893218813452|0) $%
\item%
\newline\vspace{0.5cm}$f(x)=-(x-8)^2 +2 \Rightarrow $ Zwei Nullstellen $: (9.414213562373096|0) $ und $ (6.585786437626905|0) $%
\item%
\newline\vspace{0.5cm}$f(x)=-4(x+10)^2 -4 \Rightarrow  $ Keine Lösung/Keine Nullstellen $ $%
\item%
\newline\vspace{0.5cm}$f(x)=3(x)^2 +2 \Rightarrow  $ Keine Lösung/Keine Nullstellen $ $%
\item%
\newline\vspace{0.5cm}$f(x)=-(x+3)^2 \Rightarrow $ Eine Nullstelle $: (-3|0) $%
\item%
\newline\vspace{0.5cm}$f(x)=-2(x)^2 +3 \Rightarrow $ Zwei Nullstellen $: (1.224744871391589|0) $ und $ (-1.224744871391589|0) $%
\item%
\newline\vspace{0.5cm}$f(x)=-(x-8)^2 -7 \Rightarrow  $ Keine Lösung/Keine Nullstellen $ $%
\item%
\newline\vspace{0.5cm}$f(x)=3(x+8)^2 +10 \Rightarrow  $ Keine Lösung/Keine Nullstellen $ $%
\item%
\newline\vspace{0.5cm}$f(x)=(x+5)^2 -1 \Rightarrow $ Zwei Nullstellen $: (-4.0|0) $ und $ (-6.0|0) $%
\item%
\newline\vspace{0.5cm}$f(x)=4(x)^2 +7 \Rightarrow  $ Keine Lösung/Keine Nullstellen $ $%
\item%
\newline\vspace{0.5cm}$f(x)=(x+7)^2 +9 \Rightarrow  $ Keine Lösung/Keine Nullstellen $ $%
\item%
\newline\vspace{0.5cm}$f(x)=3(x-8)^2 +3 \Rightarrow  $ Keine Lösung/Keine Nullstellen $ $%
\item%
\newline\vspace{0.5cm}$f(x)=4(x-8)^2 -2 \Rightarrow $ Zwei Nullstellen $: (8.707106781186548|0) $ und $ (7.292893218813452|0) $%
\item%
\newline\vspace{0.5cm}$f(x)=4(x-10)^2 +5 \Rightarrow  $ Keine Lösung/Keine Nullstellen $ $%
\item%
\newline\vspace{0.5cm}$f(x)=4(x+7)^2 -2 \Rightarrow $ Zwei Nullstellen $: (-6.292893218813452|0) $ und $ (-7.707106781186548|0) $%
\item%
\newline\vspace{0.5cm}$f(x)=3(x+3)^2 -6 \Rightarrow $ Zwei Nullstellen $: (-1.5857864376269049|0) $ und $ (-4.414213562373095|0) $%
\item%
\newline\vspace{0.5cm}$f(x)=2(x+2)^2 -10 \Rightarrow $ Zwei Nullstellen $: (0.2360679774997898|0) $ und $ (-4.23606797749979|0) $%
\item%
\newline\vspace{0.5cm}$f(x)=-4(x-9)^2 +10 \Rightarrow $ Zwei Nullstellen $: (10.581138830084189|0) $ und $ (7.41886116991581|0) $%
\item%
\newline\vspace{0.5cm}$f(x)=3(x-1)^2 -9 \Rightarrow $ Zwei Nullstellen $: (2.732050807568877|0) $ und $ (-0.7320508075688772|0) $%
\item%
\newline\vspace{0.5cm}$f(x)=-3(x+6)^2 -2 \Rightarrow  $ Keine Lösung/Keine Nullstellen $ $%
\item%
\newline\vspace{0.5cm}$f(x)=3(x+5)^2 -9 \Rightarrow $ Zwei Nullstellen $: (-3.267949192431123|0) $ und $ (-6.732050807568877|0) $%
\item%
\newline\vspace{0.5cm}$f(x)=4(x+6)^2 +6 \Rightarrow  $ Keine Lösung/Keine Nullstellen $ $%
\end{enumerate}

%
\section{Normalform: Bestimme den Y{-}Achsenabschnitt}%
\label{sec:NormalformBestimmedenY{-}Achsenabschnitt}%
Für die Normalform ...ist der Y{-}Achsenabschnitt ....%
\begin{enumerate}[label=\alph*)]%
\item%
\newline\vspace{0.5cm}$f(x)=x^2 + 10x + 4 \Rightarrow $ Y-Achsenabschnitt: $ (0|4) $%
\item%
\newline\vspace{0.5cm}$f(x)=2x^2 - 8 \Rightarrow $ Y-Achsenabschnitt: $ (0|-8) $%
\item%
\newline\vspace{0.5cm}$f(x)=-x^2 + 8x + 2 \Rightarrow $ Y-Achsenabschnitt: $ (0|2) $%
\item%
\newline\vspace{0.5cm}$f(x)=x^2 + 18x + 2 \Rightarrow $ Y-Achsenabschnitt: $ (0|2) $%
\item%
\newline\vspace{0.5cm}$f(x)=3x^2 + 48x + 4 \Rightarrow $ Y-Achsenabschnitt: $ (0|4) $%
\item%
\newline\vspace{0.5cm}$f(x)=3x^2 + 36x \Rightarrow $ Y-Achsenabschnitt: $ (0|0) $%
\item%
\newline\vspace{0.5cm}$f(x)=-x^2 - 6x \Rightarrow $ Y-Achsenabschnitt: $ (0|0) $%
\item%
\newline\vspace{0.5cm}$f(x)=2x^2 - 5 \Rightarrow $ Y-Achsenabschnitt: $ (0|-5) $%
\item%
\newline\vspace{0.5cm}$f(x)=3x^2 + 60x + 3 \Rightarrow $ Y-Achsenabschnitt: $ (0|3) $%
\item%
\newline\vspace{0.5cm}$f(x)=-3x^2 - 12x + 7 \Rightarrow $ Y-Achsenabschnitt: $ (0|7) $%
\item%
\newline\vspace{0.5cm}$f(x)=-2x^2 + 32x - 7 \Rightarrow $ Y-Achsenabschnitt: $ (0|-7) $%
\item%
\newline\vspace{0.5cm}$f(x)=-2x^2 + 12x - 2 \Rightarrow $ Y-Achsenabschnitt: $ (0|-2) $%
\item%
\newline\vspace{0.5cm}$f(x)=3x^2 - 54x \Rightarrow $ Y-Achsenabschnitt: $ (0|0) $%
\item%
\newline\vspace{0.5cm}$f(x)=-2x^2 + 24x - 2 \Rightarrow $ Y-Achsenabschnitt: $ (0|-2) $%
\item%
\newline\vspace{0.5cm}$f(x)=-2x^2 + 2 \Rightarrow $ Y-Achsenabschnitt: $ (0|2) $%
\item%
\newline\vspace{0.5cm}$f(x)=3x^2 + 7 \Rightarrow $ Y-Achsenabschnitt: $ (0|7) $%
\item%
\newline\vspace{0.5cm}$f(x)=-x^2 - 2x + 8 \Rightarrow $ Y-Achsenabschnitt: $ (0|8) $%
\item%
\newline\vspace{0.5cm}$f(x)=4x^2 - 1 \Rightarrow $ Y-Achsenabschnitt: $ (0|-1) $%
\item%
\newline\vspace{0.5cm}$f(x)=3x^2 + 48x + 3 \Rightarrow $ Y-Achsenabschnitt: $ (0|3) $%
\item%
\newline\vspace{0.5cm}$f(x)=3x^2 + 60x - 4 \Rightarrow $ Y-Achsenabschnitt: $ (0|-4) $%
\end{enumerate}

%
\section{Scheitelpunktform: Bestimme den Y{-}Achsenabschnitt}%
\label{sec:ScheitelpunktformBestimmedenY{-}Achsenabschnitt}%
Für die Scheitelpunktform ...ist der Y{-}Achsenabschnitt ....%
\begin{enumerate}[label=\alph*)]%
\item%
\newline\vspace{0.5cm}$f(x)=-4(x+8)^2 -2 \Rightarrow $ Y-Achsenabschnitt: $ (0|-258) $%
\item%
\newline\vspace{0.5cm}$f(x)=-2(x+7)^2 +7 \Rightarrow $ Y-Achsenabschnitt: $ (0|-91) $%
\item%
\newline\vspace{0.5cm}$f(x)=-3(x-3)^2 -5 \Rightarrow $ Y-Achsenabschnitt: $ (0|-32) $%
\item%
\newline\vspace{0.5cm}$f(x)=3(x+7)^2 +6 \Rightarrow $ Y-Achsenabschnitt: $ (0|153) $%
\item%
\newline\vspace{0.5cm}$f(x)=3(x+9)^2 \Rightarrow $ Y-Achsenabschnitt: $ (0|243) $%
\item%
\newline\vspace{0.5cm}$f(x)=2(x-3)^2 +4 \Rightarrow $ Y-Achsenabschnitt: $ (0|22) $%
\item%
\newline\vspace{0.5cm}$f(x)=-4(x)^2 -1 \Rightarrow $ Y-Achsenabschnitt: $ (0|-1) $%
\item%
\newline\vspace{0.5cm}$f(x)=2(x-5)^2 +6 \Rightarrow $ Y-Achsenabschnitt: $ (0|56) $%
\item%
\newline\vspace{0.5cm}$f(x)=2(x+6)^2 -3 \Rightarrow $ Y-Achsenabschnitt: $ (0|69) $%
\item%
\newline\vspace{0.5cm}$f(x)=-(x+2)^2 +7 \Rightarrow $ Y-Achsenabschnitt: $ (0|3) $%
\item%
\newline\vspace{0.5cm}$f(x)=-(x+9)^2 +9 \Rightarrow $ Y-Achsenabschnitt: $ (0|-72) $%
\item%
\newline\vspace{0.5cm}$f(x)=-(x+5)^2 +2 \Rightarrow $ Y-Achsenabschnitt: $ (0|-23) $%
\item%
\newline\vspace{0.5cm}$f(x)=3(x)^2 +5 \Rightarrow $ Y-Achsenabschnitt: $ (0|5) $%
\item%
\newline\vspace{0.5cm}$f(x)=(x-8)^2 +9 \Rightarrow $ Y-Achsenabschnitt: $ (0|73) $%
\item%
\newline\vspace{0.5cm}$f(x)=-2(x)^2 +3 \Rightarrow $ Y-Achsenabschnitt: $ (0|3) $%
\item%
\newline\vspace{0.5cm}$f(x)=3(x+7)^2 +7 \Rightarrow $ Y-Achsenabschnitt: $ (0|154) $%
\item%
\newline\vspace{0.5cm}$f(x)=(x+3)^2 -5 \Rightarrow $ Y-Achsenabschnitt: $ (0|4) $%
\item%
\newline\vspace{0.5cm}$f(x)=2(x+7)^2 \Rightarrow $ Y-Achsenabschnitt: $ (0|98) $%
\item%
\newline\vspace{0.5cm}$f(x)=(x-1)^2 +2 \Rightarrow $ Y-Achsenabschnitt: $ (0|3) $%
\item%
\newline\vspace{0.5cm}$f(x)=2(x-1)^2 +4 \Rightarrow $ Y-Achsenabschnitt: $ (0|6) $%
\end{enumerate}

%
\section{Faktorisierte Form: Bestimme den Y{-}Achsenabschnitt}%
\label{sec:FaktorisierteFormBestimmedenY{-}Achsenabschnitt}%
Für die Faktorisierte Form ...ist der Y{-}Achsenabschnitt ....%
\begin{enumerate}[label=\alph*)]%
\item%
\newline\vspace{0.5cm}$f(x)=4(x+5)(x+4) \Rightarrow $ Y-Achsenabschnitt: $ (0|80) $%
\item%
\newline\vspace{0.5cm}$f(x)=4(x-10)(x-2) \Rightarrow $ Y-Achsenabschnitt: $ (0|80) $%
\item%
\newline\vspace{0.5cm}$f(x)=-4(x-2)(x+7) \Rightarrow $ Y-Achsenabschnitt: $ (0|56) $%
\item%
\newline\vspace{0.5cm}$f(x)=-(x-6)(x) \Rightarrow $ Y-Achsenabschnitt: $ (0|0) $%
\item%
\newline\vspace{0.5cm}$f(x)=-3(x-1)(x+10) \Rightarrow $ Y-Achsenabschnitt: $ (0|30) $%
\item%
\newline\vspace{0.5cm}$f(x)=2(x+4)(x+7) \Rightarrow $ Y-Achsenabschnitt: $ (0|56) $%
\item%
\newline\vspace{0.5cm}$f(x)=4(x)(x+1) \Rightarrow $ Y-Achsenabschnitt: $ (0|0) $%
\item%
\newline\vspace{0.5cm}$f(x)=3(x-10)(x-7) \Rightarrow $ Y-Achsenabschnitt: $ (0|210) $%
\item%
\newline\vspace{0.5cm}$f(x)=-3(x+8)(x+3) \Rightarrow $ Y-Achsenabschnitt: $ (0|-72) $%
\item%
\newline\vspace{0.5cm}$f(x)=2(x+6)(x-9) \Rightarrow $ Y-Achsenabschnitt: $ (0|-108) $%
\item%
\newline\vspace{0.5cm}$f(x)=-(x+7)(x+10) \Rightarrow $ Y-Achsenabschnitt: $ (0|-70) $%
\item%
\newline\vspace{0.5cm}$f(x)=4(x)(x-4) \Rightarrow $ Y-Achsenabschnitt: $ (0|0) $%
\item%
\newline\vspace{0.5cm}$f(x)=4(x+10)(x+5) \Rightarrow $ Y-Achsenabschnitt: $ (0|200) $%
\item%
\newline\vspace{0.5cm}$f(x)=2(x-1)(x+2) \Rightarrow $ Y-Achsenabschnitt: $ (0|-4) $%
\item%
\newline\vspace{0.5cm}$f(x)=-(x+4)(x) \Rightarrow $ Y-Achsenabschnitt: $ (0|0) $%
\item%
\newline\vspace{0.5cm}$f(x)=2(x+9)(x+3) \Rightarrow $ Y-Achsenabschnitt: $ (0|54) $%
\item%
\newline\vspace{0.5cm}$f(x)=-(x-6)(x+9) \Rightarrow $ Y-Achsenabschnitt: $ (0|54) $%
\item%
\newline\vspace{0.5cm}$f(x)=2(x)(x-1) \Rightarrow $ Y-Achsenabschnitt: $ (0|0) $%
\item%
\newline\vspace{0.5cm}$f(x)=2(x-10)(x+3) \Rightarrow $ Y-Achsenabschnitt: $ (0|-60) $%
\item%
\newline\vspace{0.5cm}$f(x)=2(x+10)(x+9) \Rightarrow $ Y-Achsenabschnitt: $ (0|180) $%
\end{enumerate}

%
\section{Finde die Funktionsgleichung}%
\label{sec:FindedieFunktionsgleichung}%
Die Funktionsgleichung ist ...%
\begin{enumerate}[label=\alph*)]%
\item%
 Punkt Scheitelpunkt $(6.0|-4.0)$ und Y-Achsenabschnitt $140 \Rightarrow f(x)=4x^2 - 48x + 140 ; f(x)=4(x-7)(x-5) ; f(x)=4(x-6.0)^2 -4.0$%
\item%
 Nullstellen $-10$ und $-8$ und Scheitelpunkt $(-9.0|-3.0) \Rightarrow f(x)=3x^2 + 54x + 240 ; f(x)=3(x+10)(x+8) ; f(x)=3(x+9.0)^2 -3.0$%
\item%
 Nullstellen $7$ und $7$ und Scheitelpunkt $(7.0|0.0) \Rightarrow f(x)=x^2 - 14x + 49 ; f(x)=(x-7)(x-7) ; f(x)=(x-7.0)^2$%
\item%
 Punkt $(88|-23736)$ und Scheitelpunkt $(-1.0|27.0) \Rightarrow f(x)=-3x^2 - 6x + 24 ; f(x)=-3(x-2)(x+4) ; f(x)=-3(x+1.0)^2 +27.0$%
\item%
 Nullstellen $7$ und $-3$ und Scheitelpunkt $(2.0|50.0) \Rightarrow f(x)=-2x^2 + 8x + 42 ; f(x)=-2(x-7)(x+3) ; f(x)=-2(x-2.0)^2 +50.0$%
\item%
 Punkt $(42|4797)$ und Scheitelpunkt $(2.0|-3.0) \Rightarrow f(x)=3x^2 - 12x + 9 ; f(x)=3(x-1)(x-3) ; f(x)=3(x-2.0)^2 -3.0$%
\item%
 Die Funktion geht durch den Punkt $(-14|-84)$ und hat die Nullstellen $0$ und $-8 \Rightarrow f(x)=-x^2 - 8x ; f(x)=-(x)(x+8) ; f(x)=-(x+4.0)^2 +16.0$%
\item%
 Nullstellen $2$ und $0$ und Scheitelpunkt $(1.0|-2.0) \Rightarrow f(x)=2x^2 - 4x ; f(x)=2(x-2)(x) ; f(x)=2(x-1.0)^2 -2.0$%
\item%
 Die Funktion geht durch den Punkt $(-39|3861)$ und hat die Nullstellen $0$ und $-6 \Rightarrow f(x)=3x^2 + 18x ; f(x)=3(x)(x+6) ; f(x)=3(x+3.0)^2 -27.0$%
\item%
 Punkte $(64|-4355),(-39|-1368)$ und $(-9|-48) \Rightarrow f(x)=-x^2 - 4x - 3 ; f(x)=-(x+1)(x+3) ; f(x)=-(x+2.0)^2 +1.0$%
\item%
 Punkt Scheitelpunkt $(6.0|0.0)$ und Y-Achsenabschnitt $108 \Rightarrow f(x)=3x^2 - 36x + 108 ; f(x)=3(x-6)(x-6) ; f(x)=3(x-6.0)^2$%
\item%
 Nullstellen $4$ und $0$ und Scheitelpunkt $(2.0|8.0) \Rightarrow f(x)=-2x^2 + 8x ; f(x)=-2(x-4)(x) ; f(x)=-2(x-2.0)^2 +8.0$%
\item%
 Punkt Scheitelpunkt $(7.0|-2.0)$ und Y-Achsenabschnitt $96 \Rightarrow f(x)=2x^2 - 28x + 96 ; f(x)=2(x-8)(x-6) ; f(x)=2(x-7.0)^2 -2.0$%
\item%
 Nullstellen $-8$ und $4$ und Scheitelpunkt $(-2.0|72.0) \Rightarrow f(x)=-2x^2 - 8x + 64 ; f(x)=-2(x+8)(x-4) ; f(x)=-2(x+2.0)^2 +72.0$%
\item%
 Punkt Scheitelpunkt $(3.0|-18.0)$ und Y-Achsenabschnitt $0 \Rightarrow f(x)=2x^2 - 12x ; f(x)=2(x-6)(x) ; f(x)=2(x-3.0)^2 -18.0$%
\item%
 Punkt Scheitelpunkt $(-1.0|100.0)$ und Y-Achsenabschnitt $96 \Rightarrow f(x)=-4x^2 - 8x + 96 ; f(x)=-4(x+6)(x-4) ; f(x)=-4(x+1.0)^2 +100.0$%
\item%
 Punkt Scheitelpunkt $(2.0|-8.0)$ und Y-Achsenabschnitt $0 \Rightarrow f(x)=2x^2 - 8x ; f(x)=2(x)(x-4) ; f(x)=2(x-2.0)^2 -8.0$%
\item%
 Die Funktion geht durch den Punkt $(-94|-15136)$ und hat die Nullstellen $-8$ und $-6 \Rightarrow f(x)=-2x^2 - 28x - 96 ; f(x)=-2(x+8)(x+6) ; f(x)=-2(x+7.0)^2 +2.0$%
\item%
 Punkt Scheitelpunkt $(-1.0|-1.0)$ und Y-Achsenabschnitt $0 \Rightarrow f(x)=x^2 + 2x ; f(x)=(x+2)(x) ; f(x)=(x+1.0)^2 -1.0$%
\item%
 Nullstellen $-10$ und $0$ und Scheitelpunkt $(-5.0|-25.0) \Rightarrow f(x)=x^2 + 10x ; f(x)=(x+10)(x) ; f(x)=(x+5.0)^2 -25.0$%
\item%
 Nullstellen $-5$ und $7$ und Scheitelpunkt $(1.0|-72.0) \Rightarrow f(x)=2x^2 - 4x - 70 ; f(x)=2(x+5)(x-7) ; f(x)=2(x-1.0)^2 -72.0$%
\item%
 Nullstellen $-7$ und $-1$ und Scheitelpunkt $(-4.0|-18.0) \Rightarrow f(x)=2x^2 + 16x + 14 ; f(x)=2(x+7)(x+1) ; f(x)=2(x+4.0)^2 -18.0$%
\item%
 Nullstellen $8$ und $-10$ und Scheitelpunkt $(-1.0|324.0) \Rightarrow f(x)=-4x^2 - 8x + 320 ; f(x)=-4(x-8)(x+10) ; f(x)=-4(x+1.0)^2 +324.0$%
\item%
 Punkt Scheitelpunkt $(-3.0|-36.0)$ und Y-Achsenabschnitt $-27 \Rightarrow f(x)=x^2 + 6x - 27 ; f(x)=(x-3)(x+9) ; f(x)=(x+3.0)^2 -36.0$%
\item%
 Punkt $(21|1620)$ und Scheitelpunkt $(-3.0|-108.0) \Rightarrow f(x)=3x^2 + 18x - 81 ; f(x)=3(x-3)(x+9) ; f(x)=3(x+3.0)^2 -108.0$%
\item%
 Nullstellen $2$ und $-6$ und Scheitelpunkt $(-2.0|-48.0) \Rightarrow f(x)=3x^2 + 12x - 36 ; f(x)=3(x-2)(x+6) ; f(x)=3(x+2.0)^2 -48.0$%
\item%
 Nullstellen $-6$ und $-8$ und Scheitelpunkt $(-7.0|-3.0) \Rightarrow f(x)=3x^2 + 42x + 144 ; f(x)=3(x+6)(x+8) ; f(x)=3(x+7.0)^2 -3.0$%
\item%
 Punkte $(53|-2585),(47|-2009)$ und $(-75|-5913) \Rightarrow f(x)=-x^2 + 4x + 12 ; f(x)=-(x-6)(x+2) ; f(x)=-(x-2.0)^2 +16.0$%
\item%
 Nullstellen $-3$ und $-7$ und Scheitelpunkt $(-5.0|-4.0) \Rightarrow f(x)=x^2 + 10x + 21 ; f(x)=(x+3)(x+7) ; f(x)=(x+5.0)^2 -4.0$%
\item%
 Nullstellen $9$ und $-3$ und Scheitelpunkt $(3.0|72.0) \Rightarrow f(x)=-2x^2 + 12x + 54 ; f(x)=-2(x-9)(x+3) ; f(x)=-2(x-3.0)^2 +72.0$%
\item%
 Nullstellen $8$ und $2$ und Scheitelpunkt $(5.0|-27.0) \Rightarrow f(x)=3x^2 - 30x + 48 ; f(x)=3(x-8)(x-2) ; f(x)=3(x-5.0)^2 -27.0$%
\item%
 Die Funktion geht durch den Punkt $(52|-5074)$ und hat die Nullstellen $9$ und $-7 \Rightarrow f(x)=-2x^2 + 4x + 126 ; f(x)=-2(x-9)(x+7) ; f(x)=-2(x-1.0)^2 +128.0$%
\item%
 Punkt $(-9|98)$ und Scheitelpunkt $(-2.0|0.0) \Rightarrow f(x)=2x^2 + 8x + 8 ; f(x)=2(x+2)(x+2) ; f(x)=2(x+2.0)^2$%
\item%
 Nullstellen $0$ und $2$ und Scheitelpunkt $(1.0|-2.0) \Rightarrow f(x)=2x^2 - 4x ; f(x)=2(x)(x-2) ; f(x)=2(x-1.0)^2 -2.0$%
\item%
 Nullstellen $0$ und $-8$ und Scheitelpunkt $(-4.0|64.0) \Rightarrow f(x)=-4x^2 - 32x ; f(x)=-4(x)(x+8) ; f(x)=-4(x+4.0)^2 +64.0$%
\item%
 Nullstellen $-3$ und $5$ und Scheitelpunkt $(1.0|-32.0) \Rightarrow f(x)=2x^2 - 4x - 30 ; f(x)=2(x+3)(x-5) ; f(x)=2(x-1.0)^2 -32.0$%
\item%
 Nullstellen $-2$ und $0$ und Scheitelpunkt $(-1.0|-3.0) \Rightarrow f(x)=3x^2 + 6x ; f(x)=3(x+2)(x) ; f(x)=3(x+1.0)^2 -3.0$%
\item%
 Punkt Scheitelpunkt $(1.0|25.0)$ und Y-Achsenabschnitt $24 \Rightarrow f(x)=-x^2 + 2x + 24 ; f(x)=-(x+4)(x-6) ; f(x)=-(x-1.0)^2 +25.0$%
\item%
 Nullstellen $9$ und $5$ und Scheitelpunkt $(7.0|-16.0) \Rightarrow f(x)=4x^2 - 56x + 180 ; f(x)=4(x-9)(x-5) ; f(x)=4(x-7.0)^2 -16.0$%
\item%
 Punkt $(-32|-5376)$ und Scheitelpunkt $(5.0|100.0) \Rightarrow f(x)=-4x^2 + 40x ; f(x)=-4(x)(x-10) ; f(x)=-4(x-5.0)^2 +100.0$%
\end{enumerate}

%
\end{document}