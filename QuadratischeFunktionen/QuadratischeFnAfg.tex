\documentclass{article}%
\usepackage[T1]{fontenc}%
\usepackage[utf8]{inputenc}%
\usepackage{lmodern}%
\usepackage{textcomp}%
\usepackage{lastpage}%
\usepackage{xcolor}%
\usepackage{enumitem}%
%
\title{Quadratische Funktionen - Aufgaben \newline 79a1cfb4-0229-4423-a3a2-f5f020ebaf9d}%
\date{\today}%
%
\begin{document}%
\normalsize%
\maketitle%
\section{Normalform zu Scheitelpunktsform}%
\label{sec:NormalformzuScheitelpunktsform}%
Gegeben ist die Normalform. Bestimme die jeweilige Scheitelpunktsform.%
\begin{enumerate}[label=\alph*)]%
\item%
\newline\vspace{0.5cm} $f(x)=-4x^2 + 64x$%
\item%
\newline\vspace{0.5cm} $f(x)=-4x^2 - 48x + 4$%
\item%
\newline\vspace{0.5cm} $f(x)=-3x^2 + 5$%
\item%
\newline\vspace{0.5cm} $f(x)=-3x^2 + 6x + 4$%
\item%
\newline\vspace{0.5cm} $f(x)=2x^2 + 28x - 10$%
\item%
\newline\vspace{0.5cm} $f(x)=-4x^2 + 24x - 2$%
\item%
\newline\vspace{0.5cm} $f(x)=x^2 - 8x - 9$%
\item%
\newline\vspace{0.5cm} $f(x)=-2x^2 - 32x$%
\item%
\newline\vspace{0.5cm} $f(x)=-x^2 + 4x + 8$%
\item%
\newline\vspace{0.5cm} $f(x)=-2x^2 + 24x + 1$%
\item%
\newline\vspace{0.5cm} $f(x)=-2x^2 - 28x + 10$%
\item%
\newline\vspace{0.5cm} $f(x)=-2x^2 - 32x - 2$%
\item%
\newline\vspace{0.5cm} $f(x)=x^2 - 18x + 2$%
\item%
\newline\vspace{0.5cm} $f(x)=2x^2 - 36x - 4$%
\item%
\newline\vspace{0.5cm} $f(x)=x^2 + 12x$%
\item%
\newline\vspace{0.5cm} $f(x)=-x^2 + 10x + 3$%
\item%
\newline\vspace{0.5cm} $f(x)=-4x^2 + 72x + 4$%
\item%
\newline\vspace{0.5cm} $f(x)=4x^2 + 32x - 10$%
\item%
\newline\vspace{0.5cm} $f(x)=-x^2 + 16x - 5$%
\item%
\newline\vspace{0.5cm} $f(x)=-2x^2 + 12x - 8$%
\end{enumerate}

%
\section{Scheitelpunktsform zu Faktorisierten Form}%
\label{sec:ScheitelpunktsformzuFaktorisiertenForm}%
Gegeben ist die Scheitelpunktsform. Bestimme die jeweilige Faktorisierte Form.%
\begin{enumerate}[label=\alph*)]%
\item%
\newline\vspace{0.5cm} $f(x)=-(x-3.5)^2 +42.25$%
\item%
\newline\vspace{0.5cm} $f(x)=3(x-4.0)^2 -108.0$%
\item%
\newline\vspace{0.5cm} $f(x)=-2(x+4.5)^2 +24.5$%
\item%
\newline\vspace{0.5cm} $f(x)=-(x-5.5)^2 +6.25$%
\item%
\newline\vspace{0.5cm} $f(x)=2(x+9.0)^2 -2.0$%
\item%
\newline\vspace{0.5cm} $f(x)=2(x-3.5)^2 -12.5$%
\item%
\newline\vspace{0.5cm} $f(x)=-2(x-1.0)^2 +18.0$%
\item%
\newline\vspace{0.5cm} $f(x)=-4(x-5.5)^2 +9.0$%
\item%
\newline\vspace{0.5cm} $f(x)=-2(x+2.5)^2 +112.5$%
\item%
\newline\vspace{0.5cm} $f(x)=4(x-1.5)^2 -9.0$%
\item%
\newline\vspace{0.5cm} $f(x)=(x+6.5)^2 -12.25$%
\item%
\newline\vspace{0.5cm} $f(x)=-3(x-1.5)^2 +216.75$%
\item%
\newline\vspace{0.5cm} $f(x)=-(x+10.0)^2$%
\item%
\newline\vspace{0.5cm} $f(x)=-3(x-2.0)^2 +12.0$%
\item%
\newline\vspace{0.5cm} $f(x)=4(x)^2 -196.0$%
\item%
\newline\vspace{0.5cm} $f(x)=(x-1.0)^2 -1.0$%
\item%
\newline\vspace{0.5cm} $f(x)=-4(x+1.5)^2 +25.0$%
\item%
\newline\vspace{0.5cm} $f(x)=-4(x+4.0)^2 +64.0$%
\item%
\newline\vspace{0.5cm} $f(x)=(x+6.5)^2 -0.25$%
\item%
\newline\vspace{0.5cm} $f(x)=-4(x-1.0)^2 +144.0$%
\end{enumerate}

%
\section{Faktorisierte Form zu Normalform}%
\label{sec:FaktorisierteFormzuNormalform}%
Gegeben ist die Faktorisierte Form. Bestimme die jeweilige Normalform.%
\begin{enumerate}[label=\alph*)]%
\item%
\newline\vspace{0.5cm} $f(x)=-(x-1)(x+3)$%
\item%
\newline\vspace{0.5cm} $f(x)=4(x+9)(x-8)$%
\item%
\newline\vspace{0.5cm} $f(x)=-(x-5)(x-7)$%
\item%
\newline\vspace{0.5cm} $f(x)=4(x)(x-1)$%
\item%
\newline\vspace{0.5cm} $f(x)=-4(x-3)(x)$%
\item%
\newline\vspace{0.5cm} $f(x)=-2(x-7)(x-3)$%
\item%
\newline\vspace{0.5cm} $f(x)=-(x-4)(x+10)$%
\item%
\newline\vspace{0.5cm} $f(x)=-(x+8)(x-6)$%
\item%
\newline\vspace{0.5cm} $f(x)=3(x)(x+6)$%
\item%
\newline\vspace{0.5cm} $f(x)=(x-3)(x-1)$%
\item%
\newline\vspace{0.5cm} $f(x)=-2(x-5)(x-2)$%
\item%
\newline\vspace{0.5cm} $f(x)=3(x-10)(x-1)$%
\item%
\newline\vspace{0.5cm} $f(x)=3(x+8)(x-4)$%
\item%
\newline\vspace{0.5cm} $f(x)=4(x+6)(x-8)$%
\item%
\newline\vspace{0.5cm} $f(x)=(x+7)(x-3)$%
\item%
\newline\vspace{0.5cm} $f(x)=4(x-8)(x-1)$%
\item%
\newline\vspace{0.5cm} $f(x)=-2(x)(x-8)$%
\item%
\newline\vspace{0.5cm} $f(x)=-3(x)(x)$%
\item%
\newline\vspace{0.5cm} $f(x)=-(x-5)(x)$%
\item%
\newline\vspace{0.5cm} $f(x)=-4(x-8)(x+4)$%
\end{enumerate}

%
\section{Normalform zu Faktorisierter Form}%
\label{sec:NormalformzuFaktorisierterForm}%
Gegeben ist die Normalalsform. Bestimme die jeweilige Faktorisierte Form.%
\begin{enumerate}[label=\alph*)]%
\item%
\newline\vspace{0.5cm} $f(x)=2x^2 + 8x - 120$%
\item%
\newline\vspace{0.5cm} $f(x)=3x^2 - 3x$%
\item%
\newline\vspace{0.5cm} $f(x)=-x^2 + x$%
\item%
\newline\vspace{0.5cm} $f(x)=x^2 - 12x + 20$%
\item%
\newline\vspace{0.5cm} $f(x)=-3x^2 + 18x + 81$%
\item%
\newline\vspace{0.5cm} $f(x)=4x^2 - 8x - 320$%
\item%
\newline\vspace{0.5cm} $f(x)=-4x^2$%
\item%
\newline\vspace{0.5cm} $f(x)=2x^2 + 22x + 36$%
\item%
\newline\vspace{0.5cm} $f(x)=-3x^2 - 12x + 135$%
\item%
\newline\vspace{0.5cm} $f(x)=3x^2 - 27x + 42$%
\item%
\newline\vspace{0.5cm} $f(x)=2x^2 - 6x - 80$%
\item%
\newline\vspace{0.5cm} $f(x)=-4x^2 + 196$%
\item%
\newline\vspace{0.5cm} $f(x)=2x^2 + 4x - 160$%
\item%
\newline\vspace{0.5cm} $f(x)=3x^2 + 6x$%
\item%
\newline\vspace{0.5cm} $f(x)=x^2 - 4x - 5$%
\item%
\newline\vspace{0.5cm} $f(x)=-2x^2 + 14x - 24$%
\item%
\newline\vspace{0.5cm} $f(x)=-x^2 - 2x$%
\item%
\newline\vspace{0.5cm} $f(x)=-4x^2 - 4x + 360$%
\item%
\newline\vspace{0.5cm} $f(x)=-x^2$%
\item%
\newline\vspace{0.5cm} $f(x)=4x^2 - 8x - 32$%
\end{enumerate}

%
\section{Scheitelpunktform zu Normalform}%
\label{sec:ScheitelpunktformzuNormalform}%
Gegeben ist die Scheitelpunktform. Bestimme die jeweilige Normalform.%
\begin{enumerate}[label=\alph*)]%
\item%
\newline\vspace{0.5cm} $f(x)=3(x+2)^2 +9$%
\item%
\newline\vspace{0.5cm} $f(x)=-2(x-2)^2$%
\item%
\newline\vspace{0.5cm} $f(x)=-4(x-6)^2 -3$%
\item%
\newline\vspace{0.5cm} $f(x)=2(x)^2$%
\item%
\newline\vspace{0.5cm} $f(x)=-3(x-5)^2 +4$%
\item%
\newline\vspace{0.5cm} $f(x)=-4(x+9)^2 +8$%
\item%
\newline\vspace{0.5cm} $f(x)=-3(x+3)^2 -2$%
\item%
\newline\vspace{0.5cm} $f(x)=2(x-5)^2 -2$%
\item%
\newline\vspace{0.5cm} $f(x)=4(x-3)^2 -9$%
\item%
\newline\vspace{0.5cm} $f(x)=4(x)^2 -6$%
\item%
\newline\vspace{0.5cm} $f(x)=-2(x-9)^2 -6$%
\item%
\newline\vspace{0.5cm} $f(x)=2(x+2)^2 -9$%
\item%
\newline\vspace{0.5cm} $f(x)=2(x-10)^2 +10$%
\item%
\newline\vspace{0.5cm} $f(x)=2(x-2)^2 -3$%
\item%
\newline\vspace{0.5cm} $f(x)=4(x+1)^2 +1$%
\item%
\newline\vspace{0.5cm} $f(x)=3(x+6)^2 -4$%
\item%
\newline\vspace{0.5cm} $f(x)=(x+3)^2 -10$%
\item%
\newline\vspace{0.5cm} $f(x)=-2(x-4)^2 -7$%
\item%
\newline\vspace{0.5cm} $f(x)=3(x)^2 -10$%
\item%
\newline\vspace{0.5cm} $f(x)=-(x+4)^2 +4$%
\end{enumerate}

%
\section{Faktorisierte Form zu Scheitelpunktform}%
\label{sec:FaktorisierteFormzuScheitelpunktform}%
Gegeben ist die Faktorisierte Form. Bestimme die jeweilige Scheitelpunktform.%
\begin{enumerate}[label=\alph*)]%
\item%
\newline\vspace{0.5cm} $f(x)=(x+9)(x+4)$%
\item%
\newline\vspace{0.5cm} $f(x)=-3(x+6)(x+10)$%
\item%
\newline\vspace{0.5cm} $f(x)=-2(x-3)(x+10)$%
\item%
\newline\vspace{0.5cm} $f(x)=3(x+6)(x+8)$%
\item%
\newline\vspace{0.5cm} $f(x)=-(x+9)(x+3)$%
\item%
\newline\vspace{0.5cm} $f(x)=-2(x+3)(x-6)$%
\item%
\newline\vspace{0.5cm} $f(x)=-2(x-5)(x-2)$%
\item%
\newline\vspace{0.5cm} $f(x)=-(x+5)(x)$%
\item%
\newline\vspace{0.5cm} $f(x)=-(x-7)(x+10)$%
\item%
\newline\vspace{0.5cm} $f(x)=4(x-10)(x-1)$%
\item%
\newline\vspace{0.5cm} $f(x)=3(x-10)(x-1)$%
\item%
\newline\vspace{0.5cm} $f(x)=4(x+10)(x-2)$%
\item%
\newline\vspace{0.5cm} $f(x)=3(x+6)(x+10)$%
\item%
\newline\vspace{0.5cm} $f(x)=3(x-2)(x-9)$%
\item%
\newline\vspace{0.5cm} $f(x)=4(x-10)(x-5)$%
\item%
\newline\vspace{0.5cm} $f(x)=2(x-6)(x+2)$%
\item%
\newline\vspace{0.5cm} $f(x)=4(x+7)(x-9)$%
\item%
\newline\vspace{0.5cm} $f(x)=-4(x-8)(x+4)$%
\item%
\newline\vspace{0.5cm} $f(x)=4(x+8)(x+7)$%
\item%
\newline\vspace{0.5cm} $f(x)=-(x+8)(x+1)$%
\end{enumerate}

%
\section{Scheitelpunktform: Bestimme den Scheitelpunkt}%
\label{sec:ScheitelpunktformBestimmedenScheitelpunkt}%
Gegeben ist die Scheitelpunktform. Bestimme den Scheitelpunkt.%
\begin{enumerate}[label=\alph*)]%
\item%
\newline\vspace{0.5cm} $f(x)=-2(x+3)^2 +1$%
\item%
\newline\vspace{0.5cm} $f(x)=2(x-4)^2 +8$%
\item%
\newline\vspace{0.5cm} $f(x)=-(x)^2 -2$%
\item%
\newline\vspace{0.5cm} $f(x)=3(x-3)^2 -4$%
\item%
\newline\vspace{0.5cm} $f(x)=2(x-8)^2$%
\item%
\newline\vspace{0.5cm} $f(x)=-2(x-9)^2 +9$%
\item%
\newline\vspace{0.5cm} $f(x)=-(x+10)^2 -9$%
\item%
\newline\vspace{0.5cm} $f(x)=4(x+1)^2 -8$%
\item%
\newline\vspace{0.5cm} $f(x)=3(x+7)^2$%
\item%
\newline\vspace{0.5cm} $f(x)=4(x-6)^2 -4$%
\item%
\newline\vspace{0.5cm} $f(x)=-(x+5)^2 -2$%
\item%
\newline\vspace{0.5cm} $f(x)=2(x-1)^2 +7$%
\item%
\newline\vspace{0.5cm} $f(x)=-(x)^2 -6$%
\item%
\newline\vspace{0.5cm} $f(x)=-3(x-1)^2 +8$%
\item%
\newline\vspace{0.5cm} $f(x)=3(x+4)^2 +2$%
\item%
\newline\vspace{0.5cm} $f(x)=(x-10)^2 -6$%
\item%
\newline\vspace{0.5cm} $f(x)=3(x+1)^2 +8$%
\item%
\newline\vspace{0.5cm} $f(x)=4(x+4)^2 +3$%
\item%
\newline\vspace{0.5cm} $f(x)=-4(x-7)^2 +10$%
\item%
\newline\vspace{0.5cm} $f(x)=-2(x+3)^2 +3$%
\end{enumerate}

%
\section{Normalform: Bestimme den Scheitelpunkt}%
\label{sec:NormalformBestimmedenScheitelpunkt}%
Gegeben ist die Normalform. Bestimme den Scheitelpunkt.%
\begin{enumerate}[label=\alph*)]%
\item%
\newline\vspace{0.5cm} $f(x)=3x^2 + 24x + 40$%
\item%
\newline\vspace{0.5cm} $f(x)=-2x^2 - 12x - 11$%
\item%
\newline\vspace{0.5cm} $f(x)=3x^2 - 2$%
\item%
\newline\vspace{0.5cm} $f(x)=4x^2 + 24x + 36$%
\item%
\newline\vspace{0.5cm} $f(x)=-2x^2 + 10$%
\item%
\newline\vspace{0.5cm} $f(x)=-4x^2 + 16x - 23$%
\item%
\newline\vspace{0.5cm} $f(x)=-4x^2 + 72x - 324$%
\item%
\newline\vspace{0.5cm} $f(x)=2x^2 - 36x + 160$%
\item%
\newline\vspace{0.5cm} $f(x)=-x^2 + 6x - 18$%
\item%
\newline\vspace{0.5cm} $f(x)=-4x^2 - 48x - 151$%
\item%
\newline\vspace{0.5cm} $f(x)=-3x^2 - 48x - 198$%
\item%
\newline\vspace{0.5cm} $f(x)=2x^2 + 12x + 24$%
\item%
\newline\vspace{0.5cm} $f(x)=x^2 + 2x - 7$%
\item%
\newline\vspace{0.5cm} $f(x)=-2x^2 + 16x - 40$%
\item%
\newline\vspace{0.5cm} $f(x)=3x^2 + 48x + 195$%
\item%
\newline\vspace{0.5cm} $f(x)=-2x^2 - 32x - 128$%
\item%
\newline\vspace{0.5cm} $f(x)=x^2 - 14x + 57$%
\item%
\newline\vspace{0.5cm} $f(x)=2x^2 - 40x + 200$%
\item%
\newline\vspace{0.5cm} $f(x)=-2x^2 - 4x - 8$%
\item%
\newline\vspace{0.5cm} $f(x)=2x^2$%
\end{enumerate}

%
\section{Faktorisierte Form: Bestimme den Scheitelpunkt}%
\label{sec:FaktorisierteFormBestimmedenScheitelpunkt}%
Gegeben ist die Faktorisierte Form. Bestimme den Scheitelpunkt.%
\begin{enumerate}[label=\alph*)]%
\item%
\newline\vspace{0.5cm} $f(x)=-(x+6)(x-2)$%
\item%
\newline\vspace{0.5cm} $f(x)=-2(x-1)(x+7)$%
\item%
\newline\vspace{0.5cm} $f(x)=-2(x-2)(x-1)$%
\item%
\newline\vspace{0.5cm} $f(x)=-2(x+3)(x-3)$%
\item%
\newline\vspace{0.5cm} $f(x)=-(x+2)(x+4)$%
\item%
\newline\vspace{0.5cm} $f(x)=-2(x+6)(x+6)$%
\item%
\newline\vspace{0.5cm} $f(x)=-2(x-5)(x+7)$%
\item%
\newline\vspace{0.5cm} $f(x)=4(x-2)(x-6)$%
\item%
\newline\vspace{0.5cm} $f(x)=-4(x+3)(x-9)$%
\item%
\newline\vspace{0.5cm} $f(x)=2(x+6)(x-6)$%
\item%
\newline\vspace{0.5cm} $f(x)=-2(x-4)(x+7)$%
\item%
\newline\vspace{0.5cm} $f(x)=4(x-4)(x+2)$%
\item%
\newline\vspace{0.5cm} $f(x)=-3(x-4)(x)$%
\item%
\newline\vspace{0.5cm} $f(x)=-2(x-6)(x+5)$%
\item%
\newline\vspace{0.5cm} $f(x)=-3(x)(x-8)$%
\item%
\newline\vspace{0.5cm} $f(x)=(x+7)(x+3)$%
\item%
\newline\vspace{0.5cm} $f(x)=2(x)(x+4)$%
\item%
\newline\vspace{0.5cm} $f(x)=-4(x)(x+10)$%
\item%
\newline\vspace{0.5cm} $f(x)=3(x+7)(x-4)$%
\item%
\newline\vspace{0.5cm} $f(x)=4(x+2)(x+1)$%
\end{enumerate}

%
\section{Faktorisierte Form: Bestimme die Nullstellen}%
\label{sec:FaktorisierteFormBestimmedieNullstellen}%
Gegeben ist die Faktorisierte Form. Bestimme die Nullstellen.%
\begin{enumerate}[label=\alph*)]%
\item%
\newline\vspace{0.5cm} $f(x)=-3(x+8)(x+7)$%
\item%
\newline\vspace{0.5cm} $f(x)=2(x-2)(x+6)$%
\item%
\newline\vspace{0.5cm} $f(x)=-4(x-2)(x-4)$%
\item%
\newline\vspace{0.5cm} $f(x)=-3(x-4)(x+1)$%
\item%
\newline\vspace{0.5cm} $f(x)=-4(x+9)(x-9)$%
\item%
\newline\vspace{0.5cm} $f(x)=-3(x-5)(x-5)$%
\item%
\newline\vspace{0.5cm} $f(x)=-4(x+10)(x)$%
\item%
\newline\vspace{0.5cm} $f(x)=-3(x-9)(x+1)$%
\item%
\newline\vspace{0.5cm} $f(x)=3(x+5)(x-8)$%
\item%
\newline\vspace{0.5cm} $f(x)=-(x-8)(x+6)$%
\item%
\newline\vspace{0.5cm} $f(x)=-2(x-9)(x+2)$%
\item%
\newline\vspace{0.5cm} $f(x)=-2(x-1)(x)$%
\item%
\newline\vspace{0.5cm} $f(x)=(x-4)(x-5)$%
\item%
\newline\vspace{0.5cm} $f(x)=-3(x)(x+1)$%
\item%
\newline\vspace{0.5cm} $f(x)=2(x-10)(x)$%
\item%
\newline\vspace{0.5cm} $f(x)=2(x-1)(x)$%
\item%
\newline\vspace{0.5cm} $f(x)=(x-1)(x-3)$%
\item%
\newline\vspace{0.5cm} $f(x)=4(x-4)(x+9)$%
\item%
\newline\vspace{0.5cm} $f(x)=(x+5)(x+5)$%
\item%
\newline\vspace{0.5cm} $f(x)=(x)(x)$%
\end{enumerate}

%
\section{Normalform: Bestimme die Nullstellen}%
\label{sec:NormalformBestimmedieNullstellen}%
Gegeben ist die Normalform. Bestimme die Nullstellen.%
\begin{enumerate}[label=\alph*)]%
\item%
\newline\vspace{0.5cm} $f(x)=-2x^2 + 28x - 10$%
\item%
\newline\vspace{0.5cm} $f(x)=-x^2 + 14x - 3$%
\item%
\newline\vspace{0.5cm} $f(x)=4x^2 - 40x - 4$%
\item%
\newline\vspace{0.5cm} $f(x)=4x^2 + 4$%
\item%
\newline\vspace{0.5cm} $f(x)=-4x^2 + 24x - 3$%
\item%
\newline\vspace{0.5cm} $f(x)=-4x^2 - 64x - 8$%
\item%
\newline\vspace{0.5cm} $f(x)=-2x^2 + 20x - 4$%
\item%
\newline\vspace{0.5cm} $f(x)=3x^2$%
\item%
\newline\vspace{0.5cm} $f(x)=-3x^2 - 42x - 6$%
\item%
\newline\vspace{0.5cm} $f(x)=-3x^2 - 18x + 9$%
\item%
\newline\vspace{0.5cm} $f(x)=-x^2 + 4x + 4$%
\item%
\newline\vspace{0.5cm} $f(x)=3x^2 + 54x + 5$%
\item%
\newline\vspace{0.5cm} $f(x)=4x^2 - 8x - 9$%
\item%
\newline\vspace{0.5cm} $f(x)=3x^2 + 54x + 7$%
\item%
\newline\vspace{0.5cm} $f(x)=-x^2 + 14x + 1$%
\item%
\newline\vspace{0.5cm} $f(x)=3x^2 - 30x + 8$%
\item%
\newline\vspace{0.5cm} $f(x)=3x^2 - 48x - 1$%
\item%
\newline\vspace{0.5cm} $f(x)=-3x^2 - 24x + 10$%
\item%
\newline\vspace{0.5cm} $f(x)=-3x^2 - 54x + 2$%
\item%
\newline\vspace{0.5cm} $f(x)=x^2 + 6x - 5$%
\item%
\newline\vspace{0.5cm} $f(x)=-2x^2 - 16x - 2$%
\item%
\newline\vspace{0.5cm} $f(x)=3x^2 - 30x - 1$%
\item%
\newline\vspace{0.5cm} $f(x)=x^2 - 14x$%
\item%
\newline\vspace{0.5cm} $f(x)=4x^2 - 24x - 6$%
\item%
\newline\vspace{0.5cm} $f(x)=x^2 + 8x + 5$%
\item%
\newline\vspace{0.5cm} $f(x)=x^2 - 2x - 8$%
\item%
\newline\vspace{0.5cm} $f(x)=-x^2 - 4x + 2$%
\item%
\newline\vspace{0.5cm} $f(x)=-3x^2 - 60x - 7$%
\item%
\newline\vspace{0.5cm} $f(x)=-2x^2 - 3$%
\item%
\newline\vspace{0.5cm} $f(x)=-4x^2 + 56x + 10$%
\item%
\newline\vspace{0.5cm} $f(x)=-x^2 + 14x + 9$%
\item%
\newline\vspace{0.5cm} $f(x)=-x^2 + 18x + 8$%
\item%
\newline\vspace{0.5cm} $f(x)=-3x^2 - 42x + 5$%
\item%
\newline\vspace{0.5cm} $f(x)=-x^2 + 12x - 7$%
\item%
\newline\vspace{0.5cm} $f(x)=-4x^2 - 5$%
\item%
\newline\vspace{0.5cm} $f(x)=-2x^2 - 12x$%
\item%
\newline\vspace{0.5cm} $f(x)=-4x^2 + 16x + 2$%
\item%
\newline\vspace{0.5cm} $f(x)=-2x^2 - 12x + 1$%
\item%
\newline\vspace{0.5cm} $f(x)=2x^2 - 8x - 10$%
\item%
\newline\vspace{0.5cm} $f(x)=3x^2 + 6x$%
\end{enumerate}

%
\section{Scheitelpunktform: Bestimme die Nullstellen}%
\label{sec:ScheitelpunktformBestimmedieNullstellen}%
Gegeben ist die Scheitelpunktform. Bestimme die Nullstellen.%
\begin{enumerate}[label=\alph*)]%
\item%
\newline\vspace{0.5cm} $f(x)=3(x+1)^2 -6$%
\item%
\newline\vspace{0.5cm} $f(x)=-4(x+1)^2 -5$%
\item%
\newline\vspace{0.5cm} $f(x)=-2(x-8)^2 -5$%
\item%
\newline\vspace{0.5cm} $f(x)=3(x+2)^2 +1$%
\item%
\newline\vspace{0.5cm} $f(x)=-(x+5)^2 +1$%
\item%
\newline\vspace{0.5cm} $f(x)=-3(x-1)^2 +7$%
\item%
\newline\vspace{0.5cm} $f(x)=-4(x+4)^2 -4$%
\item%
\newline\vspace{0.5cm} $f(x)=(x-4)^2 -7$%
\item%
\newline\vspace{0.5cm} $f(x)=4(x+1)^2 -10$%
\item%
\newline\vspace{0.5cm} $f(x)=-3(x+7)^2 +10$%
\item%
\newline\vspace{0.5cm} $f(x)=-2(x+2)^2 -3$%
\item%
\newline\vspace{0.5cm} $f(x)=-4(x-2)^2 +6$%
\item%
\newline\vspace{0.5cm} $f(x)=4(x-7)^2$%
\item%
\newline\vspace{0.5cm} $f(x)=2(x-9)^2 -5$%
\item%
\newline\vspace{0.5cm} $f(x)=(x)^2 -2$%
\item%
\newline\vspace{0.5cm} $f(x)=(x+5)^2 +5$%
\item%
\newline\vspace{0.5cm} $f(x)=3(x-1)^2 +2$%
\item%
\newline\vspace{0.5cm} $f(x)=3(x+6)^2 -8$%
\item%
\newline\vspace{0.5cm} $f(x)=-4(x-5)^2 +2$%
\item%
\newline\vspace{0.5cm} $f(x)=-(x-8)^2 +2$%
\item%
\newline\vspace{0.5cm} $f(x)=-4(x+10)^2 -4$%
\item%
\newline\vspace{0.5cm} $f(x)=3(x)^2 +2$%
\item%
\newline\vspace{0.5cm} $f(x)=-(x+3)^2$%
\item%
\newline\vspace{0.5cm} $f(x)=-2(x)^2 +3$%
\item%
\newline\vspace{0.5cm} $f(x)=-(x-8)^2 -7$%
\item%
\newline\vspace{0.5cm} $f(x)=3(x+8)^2 +10$%
\item%
\newline\vspace{0.5cm} $f(x)=(x+5)^2 -1$%
\item%
\newline\vspace{0.5cm} $f(x)=4(x)^2 +7$%
\item%
\newline\vspace{0.5cm} $f(x)=(x+7)^2 +9$%
\item%
\newline\vspace{0.5cm} $f(x)=3(x-8)^2 +3$%
\item%
\newline\vspace{0.5cm} $f(x)=4(x-8)^2 -2$%
\item%
\newline\vspace{0.5cm} $f(x)=4(x-10)^2 +5$%
\item%
\newline\vspace{0.5cm} $f(x)=4(x+7)^2 -2$%
\item%
\newline\vspace{0.5cm} $f(x)=3(x+3)^2 -6$%
\item%
\newline\vspace{0.5cm} $f(x)=2(x+2)^2 -10$%
\item%
\newline\vspace{0.5cm} $f(x)=-4(x-9)^2 +10$%
\item%
\newline\vspace{0.5cm} $f(x)=3(x-1)^2 -9$%
\item%
\newline\vspace{0.5cm} $f(x)=-3(x+6)^2 -2$%
\item%
\newline\vspace{0.5cm} $f(x)=3(x+5)^2 -9$%
\item%
\newline\vspace{0.5cm} $f(x)=4(x+6)^2 +6$%
\end{enumerate}

%
\section{Normalform: Bestimme den Y{-}Achsenabschnitt}%
\label{sec:NormalformBestimmedenY{-}Achsenabschnitt}%
Gegeben ist die Normalform. Bestimme den Y{-}Achsenabschnitt.%
\begin{enumerate}[label=\alph*)]%
\item%
\newline\vspace{0.5cm} $f(x)=x^2 + 10x + 4$%
\item%
\newline\vspace{0.5cm} $f(x)=2x^2 - 8$%
\item%
\newline\vspace{0.5cm} $f(x)=-x^2 + 8x + 2$%
\item%
\newline\vspace{0.5cm} $f(x)=x^2 + 18x + 2$%
\item%
\newline\vspace{0.5cm} $f(x)=3x^2 + 48x + 4$%
\item%
\newline\vspace{0.5cm} $f(x)=3x^2 + 36x$%
\item%
\newline\vspace{0.5cm} $f(x)=-x^2 - 6x$%
\item%
\newline\vspace{0.5cm} $f(x)=2x^2 - 5$%
\item%
\newline\vspace{0.5cm} $f(x)=3x^2 + 60x + 3$%
\item%
\newline\vspace{0.5cm} $f(x)=-3x^2 - 12x + 7$%
\item%
\newline\vspace{0.5cm} $f(x)=-2x^2 + 32x - 7$%
\item%
\newline\vspace{0.5cm} $f(x)=-2x^2 + 12x - 2$%
\item%
\newline\vspace{0.5cm} $f(x)=3x^2 - 54x$%
\item%
\newline\vspace{0.5cm} $f(x)=-2x^2 + 24x - 2$%
\item%
\newline\vspace{0.5cm} $f(x)=-2x^2 + 2$%
\item%
\newline\vspace{0.5cm} $f(x)=3x^2 + 7$%
\item%
\newline\vspace{0.5cm} $f(x)=-x^2 - 2x + 8$%
\item%
\newline\vspace{0.5cm} $f(x)=4x^2 - 1$%
\item%
\newline\vspace{0.5cm} $f(x)=3x^2 + 48x + 3$%
\item%
\newline\vspace{0.5cm} $f(x)=3x^2 + 60x - 4$%
\end{enumerate}

%
\section{Scheitelpunktform: Bestimme den Y{-}Achsenabschnitt}%
\label{sec:ScheitelpunktformBestimmedenY{-}Achsenabschnitt}%
Gegeben ist die Scheitelpunktform. Bestimme den Y{-}Achsenabschnitt.%
\begin{enumerate}[label=\alph*)]%
\item%
\newline\vspace{0.5cm} $f(x)=-4(x+8)^2 -2$%
\item%
\newline\vspace{0.5cm} $f(x)=-2(x+7)^2 +7$%
\item%
\newline\vspace{0.5cm} $f(x)=-3(x-3)^2 -5$%
\item%
\newline\vspace{0.5cm} $f(x)=3(x+7)^2 +6$%
\item%
\newline\vspace{0.5cm} $f(x)=3(x+9)^2$%
\item%
\newline\vspace{0.5cm} $f(x)=2(x-3)^2 +4$%
\item%
\newline\vspace{0.5cm} $f(x)=-4(x)^2 -1$%
\item%
\newline\vspace{0.5cm} $f(x)=2(x-5)^2 +6$%
\item%
\newline\vspace{0.5cm} $f(x)=2(x+6)^2 -3$%
\item%
\newline\vspace{0.5cm} $f(x)=-(x+2)^2 +7$%
\item%
\newline\vspace{0.5cm} $f(x)=-(x+9)^2 +9$%
\item%
\newline\vspace{0.5cm} $f(x)=-(x+5)^2 +2$%
\item%
\newline\vspace{0.5cm} $f(x)=3(x)^2 +5$%
\item%
\newline\vspace{0.5cm} $f(x)=(x-8)^2 +9$%
\item%
\newline\vspace{0.5cm} $f(x)=-2(x)^2 +3$%
\item%
\newline\vspace{0.5cm} $f(x)=3(x+7)^2 +7$%
\item%
\newline\vspace{0.5cm} $f(x)=(x+3)^2 -5$%
\item%
\newline\vspace{0.5cm} $f(x)=2(x+7)^2$%
\item%
\newline\vspace{0.5cm} $f(x)=(x-1)^2 +2$%
\item%
\newline\vspace{0.5cm} $f(x)=2(x-1)^2 +4$%
\end{enumerate}

%
\section{Faktorisierte Form: Bestimme den Y{-}Achsenabschnitt}%
\label{sec:FaktorisierteFormBestimmedenY{-}Achsenabschnitt}%
Gegeben ist die Faktorisierte Form. Bestimme den Y{-}Achsenabschnitt.%
\begin{enumerate}[label=\alph*)]%
\item%
\newline\vspace{0.5cm} $f(x)=4(x+5)(x+4)$%
\item%
\newline\vspace{0.5cm} $f(x)=4(x-10)(x-2)$%
\item%
\newline\vspace{0.5cm} $f(x)=-4(x-2)(x+7)$%
\item%
\newline\vspace{0.5cm} $f(x)=-(x-6)(x)$%
\item%
\newline\vspace{0.5cm} $f(x)=-3(x-1)(x+10)$%
\item%
\newline\vspace{0.5cm} $f(x)=2(x+4)(x+7)$%
\item%
\newline\vspace{0.5cm} $f(x)=4(x)(x+1)$%
\item%
\newline\vspace{0.5cm} $f(x)=3(x-10)(x-7)$%
\item%
\newline\vspace{0.5cm} $f(x)=-3(x+8)(x+3)$%
\item%
\newline\vspace{0.5cm} $f(x)=2(x+6)(x-9)$%
\item%
\newline\vspace{0.5cm} $f(x)=-(x+7)(x+10)$%
\item%
\newline\vspace{0.5cm} $f(x)=4(x)(x-4)$%
\item%
\newline\vspace{0.5cm} $f(x)=4(x+10)(x+5)$%
\item%
\newline\vspace{0.5cm} $f(x)=2(x-1)(x+2)$%
\item%
\newline\vspace{0.5cm} $f(x)=-(x+4)(x)$%
\item%
\newline\vspace{0.5cm} $f(x)=2(x+9)(x+3)$%
\item%
\newline\vspace{0.5cm} $f(x)=-(x-6)(x+9)$%
\item%
\newline\vspace{0.5cm} $f(x)=2(x)(x-1)$%
\item%
\newline\vspace{0.5cm} $f(x)=2(x-10)(x+3)$%
\item%
\newline\vspace{0.5cm} $f(x)=2(x+10)(x+9)$%
\end{enumerate}

%
\section{Finde die Funktionsgleichung}%
\label{sec:FindedieFunktionsgleichung}%
Finde die Funktionsgleichung.%
\begin{enumerate}[label=\alph*)]%
\item%
 Die Funktion geht durch den Scheitelpunkt $(6.0|-4.0)$ und hat den Y-Achsenabschnitt $140$%
\item%
 Die Funktion hat die Nullstellen $-10$ und $-8$ und den Scheitelpunkt $(-9.0|-3.0)$%
\item%
 Die Funktion hat die Nullstellen $7$ und $7$ und den Scheitelpunkt $(7.0|0.0)$%
\item%
 Die Funktion geht durch den Punkt $(88|-23736)$ und hat den Scheitelpunkt $(-1.0|27.0)$%
\item%
 Die Funktion hat die Nullstellen $7$ und $-3$ und den Scheitelpunkt $(2.0|50.0)$%
\item%
 Die Funktion geht durch den Punkt $(42|4797)$ und hat den Scheitelpunkt $(2.0|-3.0)$%
\item%
 Die Funktion geht durch den Punkt $(-14|-84)$ und hat die Nullstellen $0$ und $-8$%
\item%
 Die Funktion hat die Nullstellen $2$ und $0$ und den Scheitelpunkt $(1.0|-2.0)$%
\item%
 Die Funktion geht durch den Punkt $(-39|3861)$ und hat die Nullstellen $0$ und $-6$%
\item%
 Die Funktion geht durch die Punkte $(64|-4355),(-39|-1368)$ und $(-9|-48)$%
\item%
 Die Funktion geht durch den Scheitelpunkt $(6.0|0.0)$ und hat den Y-Achsenabschnitt $108$%
\item%
 Die Funktion hat die Nullstellen $4$ und $0$ und den Scheitelpunkt $(2.0|8.0)$%
\item%
 Die Funktion geht durch den Scheitelpunkt $(7.0|-2.0)$ und hat den Y-Achsenabschnitt $96$%
\item%
 Die Funktion hat die Nullstellen $-8$ und $4$ und den Scheitelpunkt $(-2.0|72.0)$%
\item%
 Die Funktion geht durch den Scheitelpunkt $(3.0|-18.0)$ und hat den Y-Achsenabschnitt $0$%
\item%
 Die Funktion geht durch den Scheitelpunkt $(-1.0|100.0)$ und hat den Y-Achsenabschnitt $96$%
\item%
 Die Funktion geht durch den Scheitelpunkt $(2.0|-8.0)$ und hat den Y-Achsenabschnitt $0$%
\item%
 Die Funktion geht durch den Punkt $(-94|-15136)$ und hat die Nullstellen $-8$ und $-6$%
\item%
 Die Funktion geht durch den Scheitelpunkt $(-1.0|-1.0)$ und hat den Y-Achsenabschnitt $0$%
\item%
 Die Funktion hat die Nullstellen $-10$ und $0$ und den Scheitelpunkt $(-5.0|-25.0)$%
\item%
 Die Funktion hat die Nullstellen $-5$ und $7$ und den Scheitelpunkt $(1.0|-72.0)$%
\item%
 Die Funktion hat die Nullstellen $-7$ und $-1$ und den Scheitelpunkt $(-4.0|-18.0)$%
\item%
 Die Funktion hat die Nullstellen $8$ und $-10$ und den Scheitelpunkt $(-1.0|324.0)$%
\item%
 Die Funktion geht durch den Scheitelpunkt $(-3.0|-36.0)$ und hat den Y-Achsenabschnitt $-27$%
\item%
 Die Funktion geht durch den Punkt $(21|1620)$ und hat den Scheitelpunkt $(-3.0|-108.0)$%
\item%
 Die Funktion hat die Nullstellen $2$ und $-6$ und den Scheitelpunkt $(-2.0|-48.0)$%
\item%
 Die Funktion hat die Nullstellen $-6$ und $-8$ und den Scheitelpunkt $(-7.0|-3.0)$%
\item%
 Die Funktion geht durch die Punkte $(53|-2585),(47|-2009)$ und $(-75|-5913)$%
\item%
 Die Funktion hat die Nullstellen $-3$ und $-7$ und den Scheitelpunkt $(-5.0|-4.0)$%
\item%
 Die Funktion hat die Nullstellen $9$ und $-3$ und den Scheitelpunkt $(3.0|72.0)$%
\item%
 Die Funktion hat die Nullstellen $8$ und $2$ und den Scheitelpunkt $(5.0|-27.0)$%
\item%
 Die Funktion geht durch den Punkt $(52|-5074)$ und hat die Nullstellen $9$ und $-7$%
\item%
 Die Funktion geht durch den Punkt $(-9|98)$ und hat den Scheitelpunkt $(-2.0|0.0)$%
\item%
 Die Funktion hat die Nullstellen $0$ und $2$ und den Scheitelpunkt $(1.0|-2.0)$%
\item%
 Die Funktion hat die Nullstellen $0$ und $-8$ und den Scheitelpunkt $(-4.0|64.0)$%
\item%
 Die Funktion hat die Nullstellen $-3$ und $5$ und den Scheitelpunkt $(1.0|-32.0)$%
\item%
 Die Funktion hat die Nullstellen $-2$ und $0$ und den Scheitelpunkt $(-1.0|-3.0)$%
\item%
 Die Funktion geht durch den Scheitelpunkt $(1.0|25.0)$ und hat den Y-Achsenabschnitt $24$%
\item%
 Die Funktion hat die Nullstellen $9$ und $5$ und den Scheitelpunkt $(7.0|-16.0)$%
\item%
 Die Funktion geht durch den Punkt $(-32|-5376)$ und hat den Scheitelpunkt $(5.0|100.0)$%
\end{enumerate}

%
\end{document}