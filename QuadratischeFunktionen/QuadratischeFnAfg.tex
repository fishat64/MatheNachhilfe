\documentclass{article}%
\usepackage[T1]{fontenc}%
\usepackage[utf8]{inputenc}%
\usepackage{lmodern}%
\usepackage{textcomp}%
\usepackage{lastpage}%
\usepackage{xcolor}%
\usepackage{enumitem}%
%
\title{Quadratische Funktionen - Aufgaben \newline 4b8f0d73-64db-4785-a782-230215b40f77}%
\date{\today}%
%
\begin{document}%
\normalsize%
\maketitle%
\section{Normalform zu Scheitelpunktsform}%
\label{sec:NormalformzuScheitelpunktsform}%
Gegeben ist die Normalform. Bestimme die jeweilige Scheitelpunktsform.%
\begin{enumerate}[label=\alph*)]%
\item%
\newline\vspace{0.5cm} $f(x)=-4x^2 + 16x$%
\item%
\newline\vspace{0.5cm} $f(x)=3x^2 - 48x + 10$%
\item%
\newline\vspace{0.5cm} $f(x)=-x^2 + 10$%
\end{enumerate}

%
\section{Scheitelpunktsform zu Faktorisierten Form}%
\label{sec:ScheitelpunktsformzuFaktorisiertenForm}%
Gegeben ist die Scheitelpunktsform. Bestimme die jeweilige Faktorisierte Form.%
\begin{enumerate}[label=\alph*)]%
\item%
\newline\vspace{0.5cm} $f(x)=-4(x-1.0)^2 -256.0$%
\item%
\newline\vspace{0.5cm} $f(x)=3(x+3.5)^2 +36.75$%
\item%
\newline\vspace{0.5cm} $f(x)=3(x+4.0)^2 +12.0$%
\end{enumerate}

%
\section{Faktorisierte Form zu Normalform}%
\label{sec:FaktorisierteFormzuNormalform}%
Gegeben ist die Faktorisierte Form. Bestimme die jeweilige Normalform.%
\begin{enumerate}[label=\alph*)]%
\item%
\newline\vspace{0.5cm} $f(x)=(x+5)(x+4)$%
\item%
\newline\vspace{0.5cm} $f(x)=2(x-7)(x+1)$%
\item%
\newline\vspace{0.5cm} $f(x)=-2(x+6)(x-9)$%
\end{enumerate}

%
\section{Normalform zu Faktorisierter Form}%
\label{sec:NormalformzuFaktorisierterForm}%
Gegeben ist die Normalalsform. Bestimme die jeweilige Faktorisierte Form.%
\begin{enumerate}[label=\alph*)]%
\item%
\newline\vspace{0.5cm} $f(x)=-4x^2 - 60x - 224$%
\item%
\newline\vspace{0.5cm} $f(x)=x^2 - 5x + 6$%
\item%
\newline\vspace{0.5cm} $f(x)=-2x^2 + 28x - 90$%
\end{enumerate}

%
\section{Scheitelpunktform zu Normalform}%
\label{sec:ScheitelpunktformzuNormalform}%
Gegeben ist die Scheitelpunktform. Bestimme die jeweilige Normalform.%
\begin{enumerate}[label=\alph*)]%
\item%
\newline\vspace{0.5cm} $f(x)=2(x-10)^2 +2$%
\item%
\newline\vspace{0.5cm} $f(x)=-(x+2)^2 +7$%
\item%
\newline\vspace{0.5cm} $f(x)=-(x-9)^2 -7$%
\end{enumerate}

%
\section{Faktorisierte Form zu Scheitelpunktform}%
\label{sec:FaktorisierteFormzuScheitelpunktform}%
Gegeben ist die Faktorisierte Form. Bestimme die jeweilige Scheitelpunktform.%
\begin{enumerate}[label=\alph*)]%
\item%
\newline\vspace{0.5cm} $f(x)=2(x-7)(x-7)$%
\item%
\newline\vspace{0.5cm} $f(x)=(x-4)(x+1)$%
\item%
\newline\vspace{0.5cm} $f(x)=2(x)(x+8)$%
\end{enumerate}

%
\section{Scheitelpunktform: Bestimme den Scheitelpunkt}%
\label{sec:ScheitelpunktformBestimmedenScheitelpunkt}%
Gegeben ist die Scheitelpunktform. Bestimme den Scheitelpunkt.%
\begin{enumerate}[label=\alph*)]%
\item%
\newline\vspace{0.5cm} $f(x)=4(x+9)^2 +2$%
\item%
\newline\vspace{0.5cm} $f(x)=-2(x-9)^2 -3$%
\item%
\newline\vspace{0.5cm} $f(x)=-2(x-8)^2$%
\end{enumerate}

%
\section{Normalform: Bestimme den Scheitelpunkt}%
\label{sec:NormalformBestimmedenScheitelpunkt}%
Gegeben ist die Normalform. Bestimme den Scheitelpunkt.%
\begin{enumerate}[label=\alph*)]%
\item%
\newline\vspace{0.5cm} $f(x)=-2x^2 - 4$%
\item%
\newline\vspace{0.5cm} $f(x)=3x^2 + 18x + 23$%
\item%
\newline\vspace{0.5cm} $f(x)=2x^2 - 24x + 65$%
\end{enumerate}

%
\section{Faktorisierte Form: Bestimme den Scheitelpunkt}%
\label{sec:FaktorisierteFormBestimmedenScheitelpunkt}%
Gegeben ist die Faktorisierte Form. Bestimme den Scheitelpunkt.%
\begin{enumerate}[label=\alph*)]%
\item%
\newline\vspace{0.5cm} $f(x)=-4(x+4)(x-4)$%
\item%
\newline\vspace{0.5cm} $f(x)=-2(x+2)(x-5)$%
\item%
\newline\vspace{0.5cm} $f(x)=-4(x+2)(x+8)$%
\end{enumerate}

%
\section{Faktorisierte Form: Bestimme die Nullstellen}%
\label{sec:FaktorisierteFormBestimmedieNullstellen}%
Gegeben ist die Faktorisierte Form. Bestimme die Nullstellen.%
\begin{enumerate}[label=\alph*)]%
\item%
\newline\vspace{0.5cm} $f(x)=-4(x-7)(x-2)$%
\item%
\newline\vspace{0.5cm} $f(x)=-3(x+4)(x-1)$%
\item%
\newline\vspace{0.5cm} $f(x)=2(x-3)(x+8)$%
\end{enumerate}

%
\section{Normalform: Bestimme die Nullstellen}%
\label{sec:NormalformBestimmedieNullstellen}%
Gegeben ist die Normalform. Bestimme die Nullstellen.%
\begin{enumerate}[label=\alph*)]%
\item%
\newline\vspace{0.5cm} $f(x)=-x^2 + 16x - 2$%
\item%
\newline\vspace{0.5cm} $f(x)=2x^2 - 16x + 9$%
\item%
\newline\vspace{0.5cm} $f(x)=-x^2 + 8x$%
\item%
\newline\vspace{0.5cm} $f(x)=-3x^2 + 24x - 8$%
\item%
\newline\vspace{0.5cm} $f(x)=3x^2 + 42x + 7$%
\item%
\newline\vspace{0.5cm} $f(x)=-4x^2 - 8x - 10$%
\end{enumerate}

%
\section{Scheitelpunktform: Bestimme die Nullstellen}%
\label{sec:ScheitelpunktformBestimmedieNullstellen}%
Gegeben ist die Scheitelpunktform. Bestimme die Nullstellen.%
\begin{enumerate}[label=\alph*)]%
\item%
\newline\vspace{0.5cm} $f(x)=(x+9)^2 -2$%
\item%
\newline\vspace{0.5cm} $f(x)=2(x-2)^2 +2$%
\item%
\newline\vspace{0.5cm} $f(x)=-(x)^2 +9$%
\item%
\newline\vspace{0.5cm} $f(x)=-(x+2)^2 +7$%
\item%
\newline\vspace{0.5cm} $f(x)=(x+5)^2 +5$%
\item%
\newline\vspace{0.5cm} $f(x)=-3(x)^2 -4$%
\end{enumerate}

%
\section{Normalform: Bestimme den Y{-}Achsenabschnitt}%
\label{sec:NormalformBestimmedenY{-}Achsenabschnitt}%
Gegeben ist die Normalform. Bestimme den Y{-}Achsenabschnitt.%
\begin{enumerate}[label=\alph*)]%
\item%
\newline\vspace{0.5cm} $f(x)=4x^2 + 80x + 8$%
\item%
\newline\vspace{0.5cm} $f(x)=2x^2 - 36x - 4$%
\item%
\newline\vspace{0.5cm} $f(x)=-2x^2 - 12x$%
\end{enumerate}

%
\section{Scheitelpunktform: Bestimme den Y{-}Achsenabschnitt}%
\label{sec:ScheitelpunktformBestimmedenY{-}Achsenabschnitt}%
Gegeben ist die Scheitelpunktform. Bestimme den Y{-}Achsenabschnitt.%
\begin{enumerate}[label=\alph*)]%
\item%
\newline\vspace{0.5cm} $f(x)=-3(x)^2 -6$%
\item%
\newline\vspace{0.5cm} $f(x)=3(x-4)^2$%
\item%
\newline\vspace{0.5cm} $f(x)=-3(x+9)^2 -5$%
\end{enumerate}

%
\section{Faktorisierte Form: Bestimme den Y{-}Achsenabschnitt}%
\label{sec:FaktorisierteFormBestimmedenY{-}Achsenabschnitt}%
Gegeben ist die Faktorisierte Form. Bestimme den Y{-}Achsenabschnitt.%
\begin{enumerate}[label=\alph*)]%
\item%
\newline\vspace{0.5cm} $f(x)=3(x+6)(x-7)$%
\item%
\newline\vspace{0.5cm} $f(x)=-2(x)(x)$%
\item%
\newline\vspace{0.5cm} $f(x)=-(x-8)(x-4)$%
\end{enumerate}

%
\section{Finde die Funktionsgleichung}%
\label{sec:FindedieFunktionsgleichung}%
Finde die Funktionsgleichung.%
\begin{enumerate}[label=\alph*)]%
\item%
 Die Funktion geht durch den Punkt $(59|7198)$ und hat den Scheitelpunkt $(-1.0|-2.0)$%
\item%
 Die Funktion geht durch die Punkte $(31|-1353),(65|-5025)$ und $(-42|-1280)$%
\item%
 Die Funktion hat die Nullstellen $-1$ und $7$ und den Scheitelpunkt $(-3.0|16.0)$%
\item%
 Die Funktion geht durch den Scheitelpunkt $(-1.0|2.0)$ und hat den Y-Achsenabschnitt $0$%
\item%
 Die Funktion hat die Nullstellen $-1$ und $-1$ und den Scheitelpunkt $(1.0|0.0)$%
\item%
 Die Funktion hat die Nullstellen $-5$ und $3$ und den Scheitelpunkt $(1.0|-16.0)$%
\item%
 Die Funktion geht durch den Scheitelpunkt $(5.0|-100.0)$ und hat den Y-Achsenabschnitt $0$%
\item%
 Die Funktion hat die Nullstellen $7$ und $-1$ und den Scheitelpunkt $(-3.0|-32.0)$%
\item%
 Die Funktion hat die Nullstellen $5$ und $3$ und den Scheitelpunkt $(-4.0|-1.0)$%
\end{enumerate}

%
\end{document}