\documentclass{article}%
\usepackage[T1]{fontenc}%
\usepackage[utf8]{inputenc}%
\usepackage{lmodern}%
\usepackage{textcomp}%
\usepackage{lastpage}%
\usepackage{xcolor}%
\usepackage{enumitem}%
%
\title{Quadratische Funktionen - Aufgaben \newline a2fad196-ecbe-4644-84ec-e6fdb2eec452}%
\date{\today}%
%
\begin{document}%
\normalsize%
\maketitle%
\section{Normalform zu Scheitelpunktsform}%
\label{sec:NormalformzuScheitelpunktsform}%
Gegeben ist die Normalform. Bestimme die jeweilige Scheitelpunktsform.%
\begin{enumerate}[label=\alph*)]%
\item%
\newline\vspace{0.5cm} $f(x)=2x^2 + 4$%
\item%
\newline\vspace{0.5cm} $f(x)=-x^2 + 12x - 4$%
\item%
\newline\vspace{0.5cm} $f(x)=-3x^2 + 30x + 2$%
\item%
\newline\vspace{0.5cm} $f(x)=-4x^2 - 64x + 7$%
\item%
\newline\vspace{0.5cm} $f(x)=2x^2 + 12x + 4$%
\item%
\newline\vspace{0.5cm} $f(x)=3x^2 - 18x - 8$%
\item%
\newline\vspace{0.5cm} $f(x)=x^2 - 12x - 9$%
\item%
\newline\vspace{0.5cm} $f(x)=-2x^2 + 24x - 8$%
\item%
\newline\vspace{0.5cm} $f(x)=2x^2 - 20x - 5$%
\item%
\newline\vspace{0.5cm} $f(x)=3x^2 + 12x - 9$%
\end{enumerate}

%
\section{Scheitelpunktsform zu Faktorisierten Form}%
\label{sec:ScheitelpunktsformzuFaktorisiertenForm}%
Gegeben ist die Scheitelpunktsform. Bestimme die jeweilige Faktorisierte Form.%
\begin{enumerate}[label=\alph*)]%
\item%
\newline\vspace{0.5cm} $f(x)=-(x+1.0)^2 +36.0$%
\item%
\newline\vspace{0.5cm} $f(x)=2(x-5.0)^2 -8.0$%
\item%
\newline\vspace{0.5cm} $f(x)=3(x-1.5)^2 -126.75$%
\item%
\newline\vspace{0.5cm} $f(x)=-4(x-8.0)^2 +16.0$%
\item%
\newline\vspace{0.5cm} $f(x)=4(x-5.0)^2 -16.0$%
\item%
\newline\vspace{0.5cm} $f(x)=3(x)^2 -12.0$%
\item%
\newline\vspace{0.5cm} $f(x)=(x+3.0)^2 -25.0$%
\item%
\newline\vspace{0.5cm} $f(x)=3(x)^2$%
\item%
\newline\vspace{0.5cm} $f(x)=-3(x-1.0)^2 +75.0$%
\item%
\newline\vspace{0.5cm} $f(x)=2(x)^2 -98.0$%
\end{enumerate}

%
\section{Faktorisierte Form zu Normalform}%
\label{sec:FaktorisierteFormzuNormalform}%
Gegeben ist die Faktorisierte Form. Bestimme die jeweilige Normalform.%
\begin{enumerate}[label=\alph*)]%
\item%
\newline\vspace{0.5cm} $f(x)=-4(x)(x-9)$%
\item%
\newline\vspace{0.5cm} $f(x)=-3(x+2)(x+4)$%
\item%
\newline\vspace{0.5cm} $f(x)=4(x+10)(x)$%
\item%
\newline\vspace{0.5cm} $f(x)=3(x-9)(x)$%
\item%
\newline\vspace{0.5cm} $f(x)=2(x+8)(x-8)$%
\item%
\newline\vspace{0.5cm} $f(x)=-2(x+5)(x-5)$%
\item%
\newline\vspace{0.5cm} $f(x)=-2(x-5)(x-2)$%
\item%
\newline\vspace{0.5cm} $f(x)=3(x-9)(x-5)$%
\item%
\newline\vspace{0.5cm} $f(x)=-(x-4)(x-4)$%
\item%
\newline\vspace{0.5cm} $f(x)=-4(x+9)(x-1)$%
\end{enumerate}

%
\section{Normalform zu Faktorisierter Form}%
\label{sec:NormalformzuFaktorisierterForm}%
Gegeben ist die Normalalsform. Bestimme die jeweilige Faktorisierte Form.%
\begin{enumerate}[label=\alph*)]%
\item%
\newline\vspace{0.5cm} $f(x)=4x^2 + 52x + 160$%
\item%
\newline\vspace{0.5cm} $f(x)=-4x^2 - 56x - 180$%
\item%
\newline\vspace{0.5cm} $f(x)=2x^2 - 8x - 42$%
\item%
\newline\vspace{0.5cm} $f(x)=4x^2 - 80x + 400$%
\item%
\newline\vspace{0.5cm} $f(x)=2x^2 - 8$%
\item%
\newline\vspace{0.5cm} $f(x)=3x^2 + 9x$%
\item%
\newline\vspace{0.5cm} $f(x)=2x^2 + 6x + 4$%
\item%
\newline\vspace{0.5cm} $f(x)=-3x^2 - 9x + 162$%
\item%
\newline\vspace{0.5cm} $f(x)=x^2 - 4x - 12$%
\item%
\newline\vspace{0.5cm} $f(x)=2x^2 + 16x + 14$%
\end{enumerate}

%
\section{Scheitelpunktform zu Normalform}%
\label{sec:ScheitelpunktformzuNormalform}%
Gegeben ist die Scheitelpunktform. Bestimme die jeweilige Normalform.%
\begin{enumerate}[label=\alph*)]%
\item%
\newline\vspace{0.5cm} $f(x)=-3(x)^2 -3$%
\item%
\newline\vspace{0.5cm} $f(x)=3(x+1)^2 -7$%
\item%
\newline\vspace{0.5cm} $f(x)=-3(x+4)^2 -8$%
\item%
\newline\vspace{0.5cm} $f(x)=-4(x+10)^2 -5$%
\item%
\newline\vspace{0.5cm} $f(x)=(x)^2 +2$%
\item%
\newline\vspace{0.5cm} $f(x)=-3(x)^2 +3$%
\item%
\newline\vspace{0.5cm} $f(x)=4(x+8)^2 -3$%
\item%
\newline\vspace{0.5cm} $f(x)=-4(x)^2 +2$%
\item%
\newline\vspace{0.5cm} $f(x)=-3(x+6)^2 -1$%
\item%
\newline\vspace{0.5cm} $f(x)=-2(x)^2 -2$%
\end{enumerate}

%
\section{Faktorisierte Form zu Scheitelpunktform}%
\label{sec:FaktorisierteFormzuScheitelpunktform}%
Gegeben ist die Faktorisierte Form. Bestimme die jeweilige Scheitelpunktform.%
\begin{enumerate}[label=\alph*)]%
\item%
\newline\vspace{0.5cm} $f(x)=-3(x)(x+2)$%
\item%
\newline\vspace{0.5cm} $f(x)=(x+1)(x-1)$%
\item%
\newline\vspace{0.5cm} $f(x)=-3(x)(x+9)$%
\item%
\newline\vspace{0.5cm} $f(x)=(x-5)(x-5)$%
\item%
\newline\vspace{0.5cm} $f(x)=2(x+9)(x-9)$%
\item%
\newline\vspace{0.5cm} $f(x)=4(x-1)(x+3)$%
\item%
\newline\vspace{0.5cm} $f(x)=-4(x+7)(x)$%
\item%
\newline\vspace{0.5cm} $f(x)=-3(x+1)(x+1)$%
\item%
\newline\vspace{0.5cm} $f(x)=-3(x-5)(x-3)$%
\item%
\newline\vspace{0.5cm} $f(x)=-2(x+4)(x-2)$%
\end{enumerate}

%
\section{Scheitelpunktform: Bestimme den Scheitelpunkt}%
\label{sec:ScheitelpunktformBestimmedenScheitelpunkt}%
Gegeben ist die Scheitelpunktform. Bestimme den Scheitelpunkt.%
\begin{enumerate}[label=\alph*)]%
\item%
\newline\vspace{0.5cm} $f(x)=(x-5)^2 +1$%
\item%
\newline\vspace{0.5cm} $f(x)=3(x+7)^2 +2$%
\item%
\newline\vspace{0.5cm} $f(x)=-(x-6)^2 +10$%
\item%
\newline\vspace{0.5cm} $f(x)=-4(x-6)^2 +6$%
\item%
\newline\vspace{0.5cm} $f(x)=-3(x-3)^2 -8$%
\item%
\newline\vspace{0.5cm} $f(x)=2(x)^2 -6$%
\item%
\newline\vspace{0.5cm} $f(x)=-4(x-1)^2 -10$%
\item%
\newline\vspace{0.5cm} $f(x)=2(x-6)^2 -8$%
\item%
\newline\vspace{0.5cm} $f(x)=(x-6)^2$%
\item%
\newline\vspace{0.5cm} $f(x)=-3(x+7)^2 +7$%
\end{enumerate}

%
\section{Normalform: Bestimme den Scheitelpunkt}%
\label{sec:NormalformBestimmedenScheitelpunkt}%
Gegeben ist die Normalform. Bestimme den Scheitelpunkt.%
\begin{enumerate}[label=\alph*)]%
\item%
\newline\vspace{0.5cm} $f(x)=x^2 + 14x + 51$%
\item%
\newline\vspace{0.5cm} $f(x)=-4x^2 + 72x - 330$%
\item%
\newline\vspace{0.5cm} $f(x)=-x^2 - 8x - 14$%
\item%
\newline\vspace{0.5cm} $f(x)=-x^2 - 6x - 16$%
\item%
\newline\vspace{0.5cm} $f(x)=-2x^2 - 8x - 17$%
\item%
\newline\vspace{0.5cm} $f(x)=-2x^2 + 12x - 22$%
\item%
\newline\vspace{0.5cm} $f(x)=-4x^2 - 16x - 25$%
\item%
\newline\vspace{0.5cm} $f(x)=-3x^2 + 6x - 7$%
\item%
\newline\vspace{0.5cm} $f(x)=2x^2 + 12x + 17$%
\item%
\newline\vspace{0.5cm} $f(x)=-2x^2 + 24x - 69$%
\end{enumerate}

%
\section{Faktorisierte Form: Bestimme den Scheitelpunkt}%
\label{sec:FaktorisierteFormBestimmedenScheitelpunkt}%
Gegeben ist die Faktorisierte Form. Bestimme den Scheitelpunkt.%
\begin{enumerate}[label=\alph*)]%
\item%
\newline\vspace{0.5cm} $f(x)=(x+8)(x-3)$%
\item%
\newline\vspace{0.5cm} $f(x)=3(x+4)(x-5)$%
\item%
\newline\vspace{0.5cm} $f(x)=-4(x-5)(x-1)$%
\item%
\newline\vspace{0.5cm} $f(x)=2(x-5)(x-2)$%
\item%
\newline\vspace{0.5cm} $f(x)=4(x+8)(x-2)$%
\item%
\newline\vspace{0.5cm} $f(x)=3(x-3)(x+3)$%
\item%
\newline\vspace{0.5cm} $f(x)=-4(x+5)(x+3)$%
\item%
\newline\vspace{0.5cm} $f(x)=2(x-4)(x-10)$%
\item%
\newline\vspace{0.5cm} $f(x)=2(x+8)(x-5)$%
\item%
\newline\vspace{0.5cm} $f(x)=-3(x)(x+10)$%
\end{enumerate}

%
\section{Faktorisierte Form: Bestimme die Nullstellen}%
\label{sec:FaktorisierteFormBestimmedieNullstellen}%
Gegeben ist die Faktorisierte Form. Bestimme die Nullstellen.%
\begin{enumerate}[label=\alph*)]%
\item%
\newline\vspace{0.5cm} $f(x)=-(x+4)(x+2)$%
\item%
\newline\vspace{0.5cm} $f(x)=-(x-6)(x-6)$%
\item%
\newline\vspace{0.5cm} $f(x)=(x-4)(x+3)$%
\item%
\newline\vspace{0.5cm} $f(x)=-2(x-1)(x-7)$%
\item%
\newline\vspace{0.5cm} $f(x)=3(x-5)(x+4)$%
\item%
\newline\vspace{0.5cm} $f(x)=3(x+2)(x+1)$%
\item%
\newline\vspace{0.5cm} $f(x)=4(x+4)(x)$%
\item%
\newline\vspace{0.5cm} $f(x)=(x-3)(x+6)$%
\item%
\newline\vspace{0.5cm} $f(x)=3(x+7)(x+3)$%
\item%
\newline\vspace{0.5cm} $f(x)=2(x+6)(x+6)$%
\end{enumerate}

%
\section{Normalform: Bestimme die Nullstellen}%
\label{sec:NormalformBestimmedieNullstellen}%
Gegeben ist die Normalform. Bestimme die Nullstellen.%
\begin{enumerate}[label=\alph*)]%
\item%
\newline\vspace{0.5cm} $f(x)=3x^2 + 48x + 9$%
\item%
\newline\vspace{0.5cm} $f(x)=2x^2 - 28x + 1$%
\item%
\newline\vspace{0.5cm} $f(x)=4x^2 + 48x + 10$%
\item%
\newline\vspace{0.5cm} $f(x)=4x^2 - 9$%
\item%
\newline\vspace{0.5cm} $f(x)=2x^2 + 4x + 10$%
\item%
\newline\vspace{0.5cm} $f(x)=-4x^2 - 10$%
\item%
\newline\vspace{0.5cm} $f(x)=-3x^2 - 36x + 10$%
\item%
\newline\vspace{0.5cm} $f(x)=4x^2 + 80x + 3$%
\item%
\newline\vspace{0.5cm} $f(x)=3x^2 + 24x - 5$%
\item%
\newline\vspace{0.5cm} $f(x)=-4x^2 - 16x + 2$%
\item%
\newline\vspace{0.5cm} $f(x)=-3x^2 + 60x + 4$%
\item%
\newline\vspace{0.5cm} $f(x)=x^2 - 16x - 10$%
\item%
\newline\vspace{0.5cm} $f(x)=-x^2 - 10x + 5$%
\item%
\newline\vspace{0.5cm} $f(x)=x^2 + 8x + 6$%
\item%
\newline\vspace{0.5cm} $f(x)=-2x^2 - 8x + 1$%
\item%
\newline\vspace{0.5cm} $f(x)=-3x^2 - 12x + 7$%
\item%
\newline\vspace{0.5cm} $f(x)=-4x^2 + 8x$%
\item%
\newline\vspace{0.5cm} $f(x)=3x^2 + 12x - 9$%
\item%
\newline\vspace{0.5cm} $f(x)=-3x^2 + 24x + 1$%
\item%
\newline\vspace{0.5cm} $f(x)=3x^2 + 30x - 8$%
\end{enumerate}

%
\section{Scheitelpunktform: Bestimme die Nullstellen}%
\label{sec:ScheitelpunktformBestimmedieNullstellen}%
Gegeben ist die Scheitelpunktform. Bestimme die Nullstellen.%
\begin{enumerate}[label=\alph*)]%
\item%
\newline\vspace{0.5cm} $f(x)=-4(x+10)^2 -10$%
\item%
\newline\vspace{0.5cm} $f(x)=-3(x-1)^2 -2$%
\item%
\newline\vspace{0.5cm} $f(x)=-4(x-2)^2 -3$%
\item%
\newline\vspace{0.5cm} $f(x)=3(x+3)^2 -8$%
\item%
\newline\vspace{0.5cm} $f(x)=-2(x-6)^2 -4$%
\item%
\newline\vspace{0.5cm} $f(x)=4(x+2)^2 +1$%
\item%
\newline\vspace{0.5cm} $f(x)=4(x-10)^2 -8$%
\item%
\newline\vspace{0.5cm} $f(x)=-2(x+9)^2 -1$%
\item%
\newline\vspace{0.5cm} $f(x)=2(x+10)^2 -6$%
\item%
\newline\vspace{0.5cm} $f(x)=2(x-3)^2 +6$%
\item%
\newline\vspace{0.5cm} $f(x)=2(x+2)^2 -1$%
\item%
\newline\vspace{0.5cm} $f(x)=3(x+2)^2 +8$%
\item%
\newline\vspace{0.5cm} $f(x)=-(x+4)^2 +1$%
\item%
\newline\vspace{0.5cm} $f(x)=2(x-3)^2 -5$%
\item%
\newline\vspace{0.5cm} $f(x)=-3(x-5)^2 +5$%
\item%
\newline\vspace{0.5cm} $f(x)=-3(x-2)^2 +3$%
\item%
\newline\vspace{0.5cm} $f(x)=-4(x+5)^2 +4$%
\item%
\newline\vspace{0.5cm} $f(x)=-(x-5)^2 -1$%
\item%
\newline\vspace{0.5cm} $f(x)=-2(x-8)^2 -1$%
\item%
\newline\vspace{0.5cm} $f(x)=(x-9)^2 +9$%
\end{enumerate}

%
\section{Normalform: Bestimme den Y{-}Achsenabschnitt}%
\label{sec:NormalformBestimmedenY{-}Achsenabschnitt}%
Gegeben ist die Normalform. Bestimme den Y{-}Achsenabschnitt.%
\begin{enumerate}[label=\alph*)]%
\item%
\newline\vspace{0.5cm} $f(x)=3x^2 + 30x + 6$%
\item%
\newline\vspace{0.5cm} $f(x)=2x^2 - 4x - 6$%
\item%
\newline\vspace{0.5cm} $f(x)=-x^2 - 5$%
\item%
\newline\vspace{0.5cm} $f(x)=-3x^2 + 24x - 2$%
\item%
\newline\vspace{0.5cm} $f(x)=-4x^2 + 80x - 5$%
\item%
\newline\vspace{0.5cm} $f(x)=-4x^2 + 24x - 3$%
\item%
\newline\vspace{0.5cm} $f(x)=x^2 + 2x$%
\item%
\newline\vspace{0.5cm} $f(x)=4x^2 + 16x - 2$%
\item%
\newline\vspace{0.5cm} $f(x)=-3x^2 + 18x - 8$%
\item%
\newline\vspace{0.5cm} $f(x)=-x^2 + 8$%
\end{enumerate}

%
\section{Scheitelpunktform: Bestimme den Y{-}Achsenabschnitt}%
\label{sec:ScheitelpunktformBestimmedenY{-}Achsenabschnitt}%
Gegeben ist die Scheitelpunktform. Bestimme den Y{-}Achsenabschnitt.%
\begin{enumerate}[label=\alph*)]%
\item%
\newline\vspace{0.5cm} $f(x)=-2(x+8)^2 +6$%
\item%
\newline\vspace{0.5cm} $f(x)=-2(x-3)^2$%
\item%
\newline\vspace{0.5cm} $f(x)=3(x-8)^2 +10$%
\item%
\newline\vspace{0.5cm} $f(x)=3(x-6)^2 -2$%
\item%
\newline\vspace{0.5cm} $f(x)=(x-2)^2 -4$%
\item%
\newline\vspace{0.5cm} $f(x)=4(x-6)^2 +2$%
\item%
\newline\vspace{0.5cm} $f(x)=-3(x-9)^2 +7$%
\item%
\newline\vspace{0.5cm} $f(x)=-(x)^2 +4$%
\item%
\newline\vspace{0.5cm} $f(x)=-(x-10)^2 -4$%
\item%
\newline\vspace{0.5cm} $f(x)=3(x+2)^2 -2$%
\end{enumerate}

%
\section{Faktorisierte Form: Bestimme den Y{-}Achsenabschnitt}%
\label{sec:FaktorisierteFormBestimmedenY{-}Achsenabschnitt}%
Gegeben ist die Faktorisierte Form. Bestimme den Y{-}Achsenabschnitt.%
\begin{enumerate}[label=\alph*)]%
\item%
\newline\vspace{0.5cm} $f(x)=-(x-3)(x+1)$%
\item%
\newline\vspace{0.5cm} $f(x)=(x+1)(x+8)$%
\item%
\newline\vspace{0.5cm} $f(x)=-4(x+6)(x+9)$%
\item%
\newline\vspace{0.5cm} $f(x)=-2(x+1)(x-2)$%
\item%
\newline\vspace{0.5cm} $f(x)=-2(x-3)(x+2)$%
\item%
\newline\vspace{0.5cm} $f(x)=-3(x+10)(x+5)$%
\item%
\newline\vspace{0.5cm} $f(x)=2(x-8)(x+7)$%
\item%
\newline\vspace{0.5cm} $f(x)=4(x)(x+2)$%
\item%
\newline\vspace{0.5cm} $f(x)=3(x+8)(x)$%
\item%
\newline\vspace{0.5cm} $f(x)=3(x+8)(x-8)$%
\end{enumerate}

%
\section{Finde die Funktionsgleichung}%
\label{sec:FindedieFunktionsgleichung}%
Finde die Funktionsgleichung.%
\begin{enumerate}[label=\alph*)]%
\item%
 Die Funktion hat die Nullstellen $-10$ und $4$ und den Scheitelpunkt $(-3.0|-98.0)$%
\item%
 Die Funktion geht durch den Punkt $(28|-777)$ und hat den Scheitelpunkt $(-1.0|64.0)$%
\item%
 Die Funktion geht durch den Scheitelpunkt $(-10.0|0.0)$ und hat den Y-Achsenabschnitt $-100$%
\item%
 Die Funktion geht durch den Punkt $(-19|-798)$ und hat die Nullstellen $2$ und $0$%
\item%
 Die Funktion geht durch die Punkte $(-43|4624),(-44|4900)$ und $(16|2500)$%
\item%
 Die Funktion geht durch die Punkte $(95|19530),(-76|10296)$ und $(51|5978)$%
\item%
 Die Funktion geht durch die Punkte $(64|-9520),(20|-1248)$ und $(24|-1680)$%
\item%
 Die Funktion hat die Nullstellen $-8$ und $0$ und den Scheitelpunkt $(-4.0|-32.0)$%
\item%
 Die Funktion geht durch die Punkte $(-34|-1015),(75|-5920)$ und $(-73|-5032)$%
\item%
 Die Funktion geht durch den Scheitelpunkt $(-5.0|32.0)$ und hat den Y-Achsenabschnitt $-18$%
\item%
 Die Funktion hat die Nullstellen $0$ und $8$ und den Scheitelpunkt $(4.0|-48.0)$%
\item%
 Die Funktion hat die Nullstellen $-4$ und $-6$ und den Scheitelpunkt $(-5.0|1.0)$%
\item%
 Die Funktion hat die Nullstellen $0$ und $6$ und den Scheitelpunkt $(3.0|-36.0)$%
\item%
 Die Funktion hat die Nullstellen $5$ und $9$ und den Scheitelpunkt $(7.0|-8.0)$%
\item%
 Die Funktion geht durch die Punkte $(96|39008),(65|18300)$ und $(-26|1920)$%
\item%
 Die Funktion geht durch den Punkt $(56|11200)$ und hat den Scheitelpunkt $(3.0|-36.0)$%
\item%
 Die Funktion geht durch die Punkte $(-47|-3774),(-95|-16830)$ und $(0|80)$%
\item%
 Die Funktion geht durch die Punkte $(82|25920),(44|7072)$ und $(4|-288)$%
\item%
 Die Funktion geht durch den Scheitelpunkt $(-4.0|25.0)$ und hat den Y-Achsenabschnitt $9$%
\item%
 Die Funktion geht durch die Punkte $(-52|-2600),(10|-120)$ und $(75|-5775)$%
\end{enumerate}

%
\end{document}