\documentclass{article}%
\usepackage[T1]{fontenc}%
\usepackage[utf8]{inputenc}%
\usepackage{lmodern}%
\usepackage{textcomp}%
\usepackage{lastpage}%
\usepackage{xcolor}%
\usepackage{enumitem}%
%
\title{Lineare Gleichungssysteme - Aufgaben \newline 2f6ebe3c-5ae0-4ef6-8064-8d84bb8c56fb}%
\date{\today}%
%
\begin{document}%
\normalsize%
\maketitle%
\section{Lineare Gleichungen mit einer Variablen}%
\label{sec:LineareGleichungenmiteinerVariablen}%
Bestimme X.%
\begin{enumerate}[label=\alph*)]%
\item%
\newline\vspace{0.5cm} $7x+8=71$%
\item%
\newline\vspace{0.5cm} $1x+4=5$%
\item%
\newline\vspace{0.5cm} $8x+2=18$%
\item%
\newline\vspace{0.5cm} $8x+9=25$%
\item%
\newline\vspace{0.5cm} $2x+9=17$%
\item%
\newline\vspace{0.5cm} $5x+6=21$%
\item%
\newline\vspace{0.5cm} $2x+4=16$%
\item%
\newline\vspace{0.5cm} $2x+4=18$%
\item%
\newline\vspace{0.5cm} $6x+4=22$%
\item%
\newline\vspace{0.5cm} $5x+7=37$%
\item%
\newline\vspace{0.5cm} $6x+8=32$%
\item%
\newline\vspace{0.5cm} $6x+4=10$%
\item%
\newline\vspace{0.5cm} $1x+7=12$%
\item%
\newline\vspace{0.5cm} $9x+4=49$%
\item%
\newline\vspace{0.5cm} $3x+3=21$%
\end{enumerate}

%
\section{Lineare Gleichungssysteme mit zwei Variablen}%
\label{sec:LineareGleichungssystememitzweiVariablen}%
Bestimme die Unbekannten, wende das ... an.%
\begin{enumerate}[label=\alph*)]%
\item%
Verfahren deiner Wahl \newline\vspace{0.5cm} $5x+5y+2=37\newline5x+5y+2=37\newline$%
\item%
Einsetzungsverfahren \newline\vspace{0.5cm} $8x+7y+2=93\newline1x+7y+7=49\newline$%
\item%
Additionsverfahren \newline\vspace{0.5cm} $4x+6y+1=79\newline7x+6y+5=101\newline$%
\item%
Einsetzungsverfahren \newline\vspace{0.5cm} $5x+7y+8=62\newline4x+6y+5=49\newline$%
\item%
Gleichsetzungsverfahren \newline\vspace{0.5cm} $7x+5y+7=58\newline7x+7y+8=71\newline$%
\item%
Gleichsetzungsverfahren \newline\vspace{0.5cm} $7x+1y+5=67\newline1x+4y+1=33\newline$%
\item%
Einsetzungsverfahren \newline\vspace{0.5cm} $3x+8y+9=79\newline2x+3y+4=32\newline$%
\item%
Einsetzungsverfahren \newline\vspace{0.5cm} $8x+4y+8=64\newline5x+7y+1=63\newline$%
\item%
Gleichsetzungsverfahren \newline\vspace{0.5cm} $2x+8y+5=43\newline4x+7y+2=42\newline$%
\item%
Gleichsetzungsverfahren \newline\vspace{0.5cm} $3x+5y+5=33\newline9x+8y+7=56\newline$%
\item%
Additionsverfahren \newline\vspace{0.5cm} $6x+7y+3=93\newline2x+3y+3=37\newline$%
\item%
Einsetzungsverfahren \newline\vspace{0.5cm} $3x+8y+2=90\newline7x+2y+1=73\newline$%
\item%
Additionsverfahren \newline\vspace{0.5cm} $2x+4y+4=10\newline6x+6y+1=13\newline$%
\item%
Additionsverfahren \newline\vspace{0.5cm} $2x+7y+2=69\newline4x+6y+9=71\newline$%
\item%
Einsetzungsverfahren \newline\vspace{0.5cm} $7x+9y+1=120\newline5x+3y+9=70\newline$%
\end{enumerate}

%
\section{Lineare Gleichungssysteme mit drei Variablen}%
\label{sec:LineareGleichungssystememitdreiVariablen}%
Bestimme die Unbekannten.%
\begin{enumerate}[label=\alph*)]%
\item%
\newline\vspace{0.5cm} $2x+2y+6z+1=67\newline4x+7y+1z+2=69\newline4x+4y+5z+8=91\newline$%
\item%
\newline\vspace{0.5cm} $4x+8y+6z+1=63\newline1x+4y+1z+2=17\newline7x+2y+4z+4=68\newline$%
\item%
\newline\vspace{0.5cm} $7x+3y+2z+3=92\newline4x+9y+9z+9=149\newline9x+9y+6z+8=179\newline$%
\item%
\newline\vspace{0.5cm} $6x+5y+8z+3=123\newline6x+6y+4z+6=100\newline7x+3y+8z+2=127\newline$%
\item%
\newline\vspace{0.5cm} $8x+7y+1z+3=55\newline8x+1y+9z+7=93\newline8x+8y+2z+9=67\newline$%
\item%
\newline\vspace{0.5cm} $4x+3y+4z+4=79\newline1x+3y+7z+2=77\newline7x+1y+2z+2=84\newline$%
\item%
\newline\vspace{0.5cm} $5x+3y+9z+3=82\newline4x+3y+1z+6=56\newline8x+5y+9z+8=120\newline$%
\item%
\newline\vspace{0.5cm} $1x+8y+4z+7=55\newline7x+2y+9z+5=116\newline4x+9y+3z+9=61\newline$%
\item%
\newline\vspace{0.5cm} $7x+5y+4z+3=112\newline2x+4y+6z+8=74\newline5x+2y+9z+3=96\newline$%
\item%
\newline\vspace{0.5cm} $7x+5y+3z+2=118\newline7x+3y+4z+3=111\newline1x+7y+9z+2=114\newline$%
\item%
\newline\vspace{0.5cm} $1x+3y+8z+9=105\newline1x+9y+5z+7=133\newline7x+3y+4z+1=95\newline$%
\item%
\newline\vspace{0.5cm} $6x+2y+3z+3=55\newline7x+1y+3z+6=60\newline3x+4y+7z+2=78\newline$%
\item%
\newline\vspace{0.5cm} $4x+4y+1z+7=42\newline1x+9y+6z+1=74\newline6x+5y+9z+2=104\newline$%
\item%
\newline\vspace{0.5cm} $9x+1y+4z+5=96\newline2x+8y+4z+7=77\newline7x+6y+6z+1=106\newline$%
\item%
\newline\vspace{0.5cm} $9x+9y+1z+2=57\newline9x+5y+1z+9=56\newline2x+8y+3z+9=36\newline$%
\end{enumerate}

%
\end{document}